\chapter{学科特殊环境}

\section{代码}
% the listings package

\subsection{行内代码}

一共有三种方法可以生成行内代码:
\begin{itemize}
    \item 使用 \texttt{\textbackslash verb} 抄录环境:\verb!<Your code>!
    \item 使用 \texttt{\textbackslash lstinline} 行内代码环境:\lstinline[basicstyle=\tt]+<Your code>+
    \item 使用简写的行内代码环境:|<Your code>|
\end{itemize}
在前两者中,只要使用在代码中未出现的符号将代码包括在内即可。

\subsection{代码块}

代码示例。在构建 \textsf{njuvisual} 宏包时,使用 inkscape 程序从矢量图导出的 \textsf{tikz} 曲线代码并不美观,需要缩进、对齐,因而拜托鄢老师编写了如下的 Python 脚本。
\begin{lstlisting}[language=Python,morekeywords={startswith,endswith,split,strip,join,find,append,replace}]
def deal_tuple(s):
if s.startswith("("):
    # s be like (x,y)
    return "( {0:>9}, {1:>9} )".format(*s[1:-1].split(","))
else:
    return s


indent = len(".. controls (    0.0000,    0.0000 )") - 1


def deal_line(s):
    s = " ".join(map(deal_tuple, s.split()))
    if s.find(")") != -1:
        s = " " * (indent - s.find(")")) + s
    return s


try:
    while True:
        s = input().strip()
        if s.startswith("\path"):
            l = []
            while not s.endswith(";"):
                l.append(s)  # use this instead of += for speedup
                s = input().strip()
            l.append(s[:-1])  # remove trailing ;
            # NOTE: all strings in l should be newline-free
            head, rp, body = " ".join(l).replace(".. controls", "\n.. controls").replace(
                "--", "\n--").replace(")(", ")\n(").replace("cycle(", "cycle\n(").partition("]")

            # force the program to keep newlines added manually
            result = "\n".join(map(deal_line, body.splitlines()))

            print(head, rp, "\n", result, ";", sep="")
        else:
            print(s)
except EOFError as e:
    pass
\end{lstlisting}

\section{算法块}

使用 {algorithm} 和 {algorithmic} 的算法示例\footnote{两者均位于 \textsf{algorithms} 下}。

\begin{algorithm}[htbp]
    \caption{Lanczos Algorithm}
    \begin{algorithmic}
        \STATE - Choose an initial state $\vec{b}_0$ and a maximum iteration number $m$
        \FOR{$n=0,\ldots m$}
        \STATE - Construct the space $K = \spn\{\vec{b}_0 , \ldots \vec{b}_n \}$
        \STATE - Obtain $\vec{C} = \hat{H} \vec{b}_n$
        \IF{$\vec{C} \in K$}
        \STATE Set $\vec{b}_{n+1}$ to a random vector orthogonal to $K$
        \ELSE
        \STATE Orthogonalize $\vec{C}$ against $K$, yielding $\vec{b}_{n+1}$
        \ENDIF\ENDFOR
    \end{algorithmic}
\end{algorithm}


\section{物理}
% the physics package

\begin{equation}
    \expval{A}{\Psi}
\end{equation}


\section{化学}

\subsection{化学式}

1972年,R. J. Fiela、E. Koros、R. Noyes等通过实验对BZ振荡反应作了解释。其主要思想
是:系统中存在着两个受溴离子浓度控制的过程A和B,当$[\ce{Br-}]$高于临界浓度$[\ce{Br-}]_\text{crit}$时发生
A过程,当$[\ce{Br-}]$低于$[\ce{Br-}]_\text{crit}$时发生B过程。也就是说$[\ce{Br-}]$起着开关作用,它控制着从A到
B过程,再由B到A过程的转变。在A过程,由于化学反应$[\ce{Br-}]$降低,当$[\ce{Br-}]$低于$[\ce{Br-}]_\text{crit}$
时,B过程发生。在B过程中,\ce{Br-}再生,$[\ce{Br-}]$增加,当$[\ce{Br-}]$再次高于$[\ce{Br-}]_\text{crit}$时,A过程再次
发生。这样系统就在A过程与B过程间往复振荡。下面用\ce{BrO3- - Ce^{3+} - MA - H2SO4}系统为例
加以说明。\cite{qiujinhengetal2010}

% 当$[\ce{Br-}]$足够高时,发生下列A过程:
% \begin{equation}
%     \ce{BrO3- + Br- + 2H+ ->[k_1] HBrO2 +HOBr}
% \end{equation}
% \begin{equation}
%     \ce{HBrO2 + Br- + H+ ->[k_2] 2HOBr}
% \end{equation}
% 其中第一步是速率控制步,当达到准定态时,有
% \begin{equation*}
%     \left[\ce{HBrO2}\right] =\frac{k_1}{k_2}\left[\ce{BrO3-}\right]\left[\ce{H+}\right]
% \end{equation*}

% 当$[\ce{Br-}]$低时,\ce{Ce^{3+}}被氧化发生下列B过程:
% \begin{equation}
%     \ce{BrO3- + HBrO2 + 2H+ ->[k_3] 2HBrO2 + H2O}
% \end{equation}
% \begin{equation}
%     \ce{BrO2 + Ce^{3+} + H+ ->[k_4] HBrO2 + Ce^{4+}}
% \end{equation}
% \begin{equation}
%     \ce{2HBrO2 ->[k_5] BrO3- + HOBr + H+}
% \end{equation}

% 反应(3)是速率控制步,经反应(3)、反应(4)将自催化产生\ce{HBrO2},达到准定态
% 时,有
% \begin{equation*}
%     [\ce{HBrO2}] \approx \frac{k_3}{2k_5}[\ce{BrO3-}][\ce{H+}]
% \end{equation*}

% 由反应(2)和反应(3)可以看出:\ce{Br}和\ce{BrO3-}是竞争\ce{HBrO2}的。当$k_2[\ce{Br-}]>k_3[\ce{BrO3-}]$时,自催化过程不可能发生。自催化是BZ振荡反应中必不可少的步骤,否则该振荡不
% 能发生。\ce{Br-}的临界浓度为
% \begin{equation*}
%     [\ce{Br-}]_\text{crit} =\frac{k_3}{k_2}[\ce{BrO3-}] =\num{5e-6}[\ce{BrO3-}]
% \end{equation*}
% \ce{Br-}的再生可通过下列过程实现:
% \begin{equation}
%     \ce{4Ce^{4+} + BrCH(COOH)2 +H2O + HOBr ->[k_6] Br- + 4Ce^{3+} + 3CO2 + 6H+}
% \end{equation}
% 该系统的总反应为
% \begin{equation}
%     \ce{2H+ + 2BrO3- + 2CH2(COOH)2 ->  BrCH(COOH)2 + 3CO2 + 4H2O}
% \end{equation}
% 振荡的控制物种是\ce{Br-}。


\subsection{结构式}

% \begin{figure}[H]
%   \centering
%   \chemfig{[:18]*5(-=(-CHO)-N-=)}
% \end{figure}

\schemestart
  \setchemfig{chemfig style=blue} \chemname[3ex]{\chemfig{[:-60]*6(-=)([:90]*6(=-EDG))}}{Diene}
  \hspace{10mm}
  \setchemfig{chemfig style=red} \chemname[3ex]{\chemfig{[:-60]*6(=)([::90]-EWG)}}{Dienophile}
  \hspace{10mm}
  \arrow(.mid east--.mid west)[,1.5]
  \hspace{10mm}
  \chemfig{[:-60]*6(=[,,,,blue]-[,,,,blue]--[,,,,red](<[,,,,red]{\color{red}E}|{\color{red}WG})-(<[,,,,blue]{\color{blue}EDG})-[,,,,blue])}
\schemestop

% pkg modiagram
% \begin{modiagram}[names,distance=5cm,names-style={draw=black}]
% \atom[C]{left}{
%   2s = {3.6;pair},
%   2p = {6.3;up,up},
%   label = {
%     2sleft = {2s},
%     2pyleft = {2p}
%   }
% }
% \atom[O]{right}{
%   2s = {0;pair},
%   2p = {4.8;pair,up,up},
%   label = {
%     2sright = {2s},
%     2pyright = {2p}
%   }
% }
% \molecule[CO]{
%   2sMO = {0.4;pair,pair},
%   2pMO = {0.2/0.5,0.5/0.2;pair,pair,pair},
% }
% \node[right] at (2sigma.45) {$3\sigma$};
% \node[right] at (2sigma*.-45) {$4\sigma$};
% \node[right] at (2piz.45) { $1\pi$};
% \node[right] at (2psigma.45) {$5\sigma$};
% \node[right] at (2piz*.45) {$2\pi$};
% \node[right] at (2psigma*.45) {$6\sigma$};
% \EnergyAxis[title=Energy]
% \end{modiagram}
