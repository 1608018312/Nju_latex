\chapter{页面布局}

\textbf{特别提醒:本章文字仅供格式示例,内容已停止维护,请参考宏包手册进行设置}

本模板格式依照《11-南京大学毕业论文(设计)的撰写规范和装订要求》进行调整,文件内容详见\cref{chap:standard}

\section{封面页}


\texttt{cover.sty}中定义了生成封面的相关命令


\subsection{第二导师}

secondmentor 用于指定是否在封面打印第二导师

\subsection{国家图书馆封面}

对于研究生,本模板提供了nlcover 用于生成国家图书馆封面

\subsection{文档类型}

如果编写的是毕业设计,请参考\cref{sec:classoptions},将Type选项改为design。
\subsection{多行标题}

为了使较长的论文题目也能美观地呈现在封面页上,njuthesis类提供了\texttt{TitleLength}这一选项,用于控制封面标题的行数。该命令已于\cref{sec:classoptions}进行介绍,可以在\texttt{njuthesis.tex}文件开头的类定义中找到,可选值为1、2、3,缺省值为单行标题。


\subsection{输入个人信息}

\texttt{njusetup}定义了用于文档封面的诸多属性参数,
写作时修改相应字符串即可。注意不要有空行,否则可能报错

\begin{lstlisting}
\njusetup {
    info = {
        <type> = <myinfo>; 
    }
}
\end{lstlisting}

\subsubsection{论文标题}
\begin{description}
    \item[\texttt{TitleA}] 单行标题,或多行标题的第一行。关于是否应该折行,单行能容纳的最长标题为\emph{15个中文字符},请自行选择合适的截断处。
    \item[\texttt{TitleB}] 多行标题的第二行
    \item[\texttt{TitleC}] 多行标题的第三行
    \item[\texttt{TitleEN}] 英文标题,注意空格要用波浪线(\textasciitilde)替代
\end{description}

\subsubsection{个人年级、学号、姓名}
\begin{description}
    \item[\texttt{Grade}] 年级
    \item[\texttt{StudentID}] 本科生为9位数字学号,研究生为两位英文字母标识加8位数字学号,两位字母自动大写
    \item[\texttt{StudentName}] 姓名
    \item[\texttt{StudentNameEN}] 姓名拼音 
\end{description}

\subsubsection{就读院系专业}

本科生无需填写研究方向。
\begin{description}
    \item[\texttt{Department}] 学院名称
    \item[\texttt{DepartmentEN}] 学院英文名称
    \item[\texttt{Major}] 专业名称
    \item[\texttt{MajorEN}] 专业英文名称
    \item[\texttt{Field}] 研究方向
    \item[\texttt{FieldEN}] 研究方向英文名称
\end{description}

\subsubsection{导师信息}
注意标注A的为第一导师
\begin{description}
    \item[\texttt{Mentor<A/B>}] 导师姓名
    \item[\texttt{Mentor<A/B>EN}] 导师姓名的英文拼音  
    \item[\texttt{Mentor<A/B>Title}] 导师职称
    \item[\texttt{Mentor<A/B>TitleEN}] 导师职称英文
\end{description}

\subsubsection{提交日期}
\begin{description}
    \item[\texttt{SubmitDate}] 论文提交日期
\end{description}

\subsubsection{答辩信息}

除答辩日期以外,本部分内容仅用于国家图书馆封面。本科生忽略即可。
答辩委员会姓名与职称之间需使用波浪线连接。

\begin{description}
    \item[\texttt{DefendDate}] 答辩日期
    \item[\texttt{ReviewerChairman}] 答辩委员会主席的姓名及职称 
    \item[\texttt{Reviewer<A/B/C/D>}] 四位评阅人的姓名及职称  
\end{description}

\subsubsection{国家图书馆封面相关信息}

本部分内容仅用于国家图书馆封面。本科生忽略即可。

\begin{description}
    \item[\texttt{Classification}] 分类号
    \item[\texttt{SecurityLevel}] 限制  
    \item[\texttt{UDC}] UDC
    \item[\texttt{MentorInfo}] 指导教师职务、职称、学位、单位名称及地址
\end{description}

\section{摘要页}

\texttt{profile/abstract.sty}提供了摘要页格式的定义。

\begin{lstlisting}
\begin{abstract}
    <text>
    \keywords{<keywords>}
\end{abstract}

\begin{englishabstract}
    <text>
    \englishkeywords{<keywords>}
\end{englishabstract}
\end{lstlisting}

摘要页一般不插入目录,默认只添加pdf书签。如确实有插入目录的需求,请在\texttt{abstract.sty}文件中定位到如下语句
\begin{lstlisting}[language=TeX]
% \phantomsection\addcontentsline{toc}{chapter}{中文摘要}
\pdfbookmark[0]{中文摘要}{中文摘要}
\end{lstlisting}
将其修改为
\begin{lstlisting}[language=TeX]
\phantomsection\addcontentsline{toc}{chapter}{中文摘要}
% \pdfbookmark[0]{中文摘要}{中文摘要}
\end{lstlisting}

在使用\hologo{LuaLaTeX}编译时,研究生中文摘要页的标题会出现空格无下划线的问题,目前正在积极寻求解决方法。

\section{前言页}

使用preface环境定义

\begin{lstlisting}
\begin{preface}
    <text>
\end{preface}
\end{lstlisting}

\section{目录页}

目录页格式定制于\texttt{profile/page.sty}

\section{正文}

正文格式定制于\texttt{profile/page.sty},页边距在\texttt{profile/packages.sty}

本科生无页眉,页面编号居中位于页脚;研究生无页脚,页眉包括章节名和页面编号。

% 对中文加下划线请使用xeCJKfntef包的CJKunderline命令代替uline,以解决中文的换行问题

\section{参考文献页}

需要使用biber手动编译才会显示,具体内容参考\cref{chap:reference}

\section{致谢页}

同前言,使用acknowledgement环境

\begin{lstlisting}
\begin{acknowledgement}
    <text>
\end{acknowledgement}
\end{lstlisting}

\section{附录页}

附录放在\lstinline|\appendix|命令后,以英文字母进行编号
