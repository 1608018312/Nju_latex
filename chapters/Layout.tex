\chapter{页面布局}

本模板格式依照《11-南京大学毕业论文(设计)的撰写规范和装订要求》进行调整,文件内容详见\cref{chap:standard}

\section{封面页}

\subsection{封面格式}

\texttt{cover.sty}中定义了生成封面的相关命令



\subsection{文档类型}

如果编写的是毕业设计,请参考\cref{sec:classoptions},将Type选项改为design。
\subsection{多行标题}

为了使较长的论文题目也能美观地呈现在封面页上,njuthesis类提供了\texttt{TitleLength}这一选项,用于控制封面标题的行数。该命令已于\cref{sec:classoptions}进行介绍,可以在\texttt{njuthesis.tex}文件开头的类定义中找到,可选值为1、2、3,缺省值为单行标题。

\subsection{第二导师}

secondsupervisor用于指定是否在封面打印第二导师

\subsection{输入个人信息}

主目录下的\texttt{coverinfo.sty}文件定义了用于文档封面的诸多属性参数。
写作时在此文件中修改相应字符串即可。

\subsubsection{论文标题}
\begin{description}
    \item[\texttt{\textbackslash TitleA}] 单行标题,或多行标题的第一行。关于是否应该折行,单行能容纳的最长标题为\emph{15个中文字符},请自行选择合适的截断处。
    \item[\texttt{\textbackslash TitleB}] 多行标题的第二行
    \item[\texttt{\textbackslash TitleC}] 多行标题的第三行
    \item[\texttt{\textbackslash Title\textunderscore EN}] 英文标题,注意空格要用波浪线(\textasciitilde)替代
\end{description}

\subsubsection{个人年级、学号、姓名}
\begin{description}
    \item[\texttt{\textbackslash Grade}] 年级
    \item[\texttt{\textbackslash StudentID}] 9位数字学号
    \item[\texttt{\textbackslash StudentName}] 姓名
    \item[\texttt{\textbackslash StudentName\textunderscore EN}] 姓名拼音 
\end{description}

\subsubsection{就读院系专业}
\begin{description}
    \item[\texttt{\textbackslash Department}] 学院名称
    \item[\texttt{\textbackslash Department\textunderscore EN}] 学院英文名称
    \item[\texttt{\textbackslash Major}] 专业名称
    \item[\texttt{\textbackslash Major\textunderscore EN}] 专业英文名称
\end{description}

\subsubsection{导师信息}
注意标注A的为第一导师
\begin{description}
    \item[\texttt{\textbackslash Mentor<A/B>}] 导师姓名
    \item[\texttt{\textbackslash Mentor<A/B>\textunderscore EN}] 导师姓名的英文拼音  
    \item[\texttt{\textbackslash Mentor<A/B>Title}] 导师职称
    \item[\texttt{\textbackslash Mentor<A/B>Title\textunderscore EN}] 导师职称英文
\end{description}

\subsubsection{提交日期}
\begin{description}
    \item[\texttt{\textbackslash SubmitDate}] 论文提交日期
\end{description}

\section{摘要页}

\texttt{profile/abstract.sty}提供了摘要页格式的定义。

摘要页一般不插入目录,默认只添加pdf书签。如确实有插入目录的需求,请在\texttt{abstract.sty}文件中定位到如下语句
\begin{lstlisting}[language=TeX]
% \phantomsection\addcontentsline{toc}{chapter}{中文摘要}
\pdfbookmark[0]{中文摘要}{中文摘要}
\end{lstlisting}
将其修改为
\begin{lstlisting}[language=TeX]
\phantomsection\addcontentsline{toc}{chapter}{中文摘要}
% \pdfbookmark[0]{中文摘要}{中文摘要}
\end{lstlisting}

\section{目录页}

目录页格式定制于\texttt{profile/page.sty}

\section{正文}

正文格式定制于\texttt{profile/page.sty},页边距在\texttt{profile/packages.sty}

\section{参考文献页}

需要使用biber手动编译才会显示,具体内容参考\cref{chap:reference}

\section{致谢页}

致谢页采用无编号章节形式,需要手动插入目录
\begin{lstlisting}[language=TeX]
\chapter*{致谢}
\addcontentsline{toc}{chapter}{致谢}
\end{lstlisting}

\section{附录页}

附录放在\lstinline|\appendix|命令后,以英文字母进行编号
