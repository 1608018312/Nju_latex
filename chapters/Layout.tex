\chapter{页面布局}

\section{总体要求}

摘录自《11-南京大学毕业论文(设计)的撰写规范和装订要求》,内容详见\cref{chap:standard}

\begin{description}
    \item[论文题目] 三号宋体加粗
    \item[各部分标题] 四号黑体
    \item[中文摘要、关键词内容] 小四号楷体
    \item[英文摘要、关键词内容] 小四号新罗马体(Time New Roman)
    \item[目录标题] 三号宋体加粗
    \item[目录内容中章的标题] 四号黑体
    \item[目录中其他内容] 小四号宋体
    \item[正文] 小四号宋体(行距1.5倍)
    \item[参考文献标题] 四号黑体
    \item[参考文献内容] 小四号宋体
    \item[注释内容] 五号宋体
    \item[致谢、附录标题] 四号黑体
    \item[致谢、附录内容] 小四号宋体(行距1.5倍)
    \item[非正文部分的页码] 五号罗马数字(Ⅰ、Ⅱ……)
    \item[论文页码] 页脚居中、五号阿拉伯数字(新罗马体)连续编码
\end{description}

\section{封面页}

\subsection{封面格式}

\subsection{输入个人信息}

主目录下的\texttt{coverinfo.sty}文件定义了用于文档封面的诸多属性参数,包括

\begin{description}
    \item[\texttt{\textbackslash TitleA}] 单行标题,或多行标题的第一行。关于是否应该折行,单行能容纳的最长标题为\emph{15个中文字符},请自行选择合适的截断处。
    \item[\texttt{\textbackslash TitleB}] 多行标题的第二行
    \item[\texttt{\textbackslash TitleC}] 多行标题的第三行
    \item[\texttt{\textbackslash Title\textunderscore EN}] 英文标题,注意空格要用波浪线(\textasciitilde)替代
    \item[\texttt{\textbackslash Grade}] 年级
    \item[\texttt{\textbackslash StudentID}] 9位数字学号
    \item[\texttt{\textbackslash StudentName}] 姓名
    \item[\texttt{\textbackslash StudentName\textunderscore EN}] 姓名拼音 
    \item[\texttt{\textbackslash Department}] 学院名称
    \item[\texttt{\textbackslash Major}] 专业名称
    \item[\texttt{\textbackslash Major\textunderscore EN}] 专业英文名称
    \item[\texttt{\textbackslash Mentor}] 导师姓名
    \item[\texttt{\textbackslash Mentor\textunderscore EN}] 导师姓名的英文拼音  
    \item[\texttt{\textbackslash MentorTitle}] 导师职称
    \item[\texttt{\textbackslash MentorTitle\textunderscore EN}] 导师职称英文
    \item[\texttt{\textbackslash SubmitDate}] 论文提交日期
\end{description}

为了使较长的论文题目也能美观地呈现在封面页上,njuthesis类提供了\texttt{TitleLength}这一选项,用于控制封面标题的行数。该命令已于
\cref{sec:classoptions}进行介绍,可以在\texttt{njuthesis.tex}文件开头的类定义中找到,可选值为1、2、3,缺省值为单行标题。

如果编写的是毕业设计,请参考\cref{sec:classoptions},将选项改为design。

\section{摘要页}

摘要页一般不插入目录,如有需求请在abstract.sty文件中反注释相关代码

\section{目录页}

目录页格式定制于page.sty

\section{正文}

正文格式定制于page.sty,页边距在package.sty

\section{参考文献页}

需要使用biber手动编译才会显示

\section{致谢页}

致谢页采用无编号章节形式,需要手动插入目录
\begin{lstlisting}[language=TeX]
\chapter*{致谢}
\addcontentsline{toc}{chapter}{致谢}
\end{lstlisting}

\section{附录页}

附录放在最后,以英文字母进行编号
