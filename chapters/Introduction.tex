\chapter{绪论}

\section{简介}
南京大学学士学位论文\hologo{LaTeX}模板\cite{njuthesis}基于本科生院的论文撰写规范修改,用于生成符合南京大学学位论文排版要求和相应的国家规范、行业标准的学位论文。

此模板通过LPPL协议开源,旨在为同学提供毕业论文书写的方便,如有模板问题或者版权问题,请于Github上提出issue。

\section{开发环境}

本模板的开发维护主要在\hologo{MiKTeX}与\hologo{TeX}\,Live\,2021下进行,内部代码主要由\lstinline|expl3|宏包提供的\hologo{LaTeX3}语法实现。

\section{目录结构}
\label{sec:directory}

\begin{lstlisting}[language=bash]
    │  coverinfo.sty
    │  njuthesis.bib
    │  njuthesis.cls
    │  njuthesis.tex
    │
    ├─figure
    │      njulogo.pdf
    │      njuname.pdf
    │
    └─profile
           abstract.sty
           components.sty
           cover.sty
           font.sty
           packages.sty
           page.sty
           reference.sty
\end{lstlisting}

\subsection{文件说明}
\begin{description}
    \item [\texttt{abstract.sty}] 定义中英文摘要环境及相关格式
    \item [\texttt{components.sty}] 定义图表标题格式
    \item [\texttt{cover.sty}] 定义论文封面内容
    \item [\texttt{font.sty}] 负责重定义CTeX宏集的默认字体,因为CTeX字体加粗要求不满足论文规定
    \item [\texttt{package.sty}] 存放所有被引用的包
    \item [\texttt{page.sty}] 定义正文页面、标题以及目录条目格式
    \item [\texttt{reference.sty}] 定义引文格式
\end{description}

\section{自定义选项}
\label{sec:classoptions}

njuthesis类提供了三个可以自定义的选项,如下所示。
\begin{description}
    \item[\texttt{TitleLength}] 标题长度,用于控制封面标题的行数。可选值为1、2、3,缺省值为1。
    \item[\texttt{Degree}] 学位类型,本科生为UG,即Undergraduate的缩写。
    \item[\texttt{Type}] 用于选择文档类型并将相关字段输出在封面和摘要页。可选值包括thesis(表示毕业论文)和design(表示毕业设计),缺省值为thesis。
\end{description}

