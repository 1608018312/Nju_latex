\chapter{绪论}

\textbf{特别提醒:本章文字仅供格式示例,内容已停止维护,请参考宏包手册进行设置}

\section{简介}
南京大学学位论文\hologo{LaTeX}模板\cite{njuthesis}基于本科生院的论文撰写规范制作,同时参考研究生院提供的硕士、博士学位材料包,用于生成符合南京大学学位论文排版要求和相应的国家规范、行业标准的学位论文,旨在为同学提供毕业论文书写的方便。

此模板通过LPPL协议开源,目前已\href{https://ctan.org/pkg/njuthesis}{发布在CTAN}。

如有模板问题或者版权问题,请于Github项目主页上\href{https://github.com/nju-lug/NJUThesis/issues}{提出issue}。

\begin{quotation}
    请注意,本示例文档内容\textbf{存在较多过时之处},请参考.dtx文件生成的使用手册以获得准确的参考。
\end{quotation}

\section{开发环境}

本模板的开发维护主要在\hologo{MiKTeX}与\hologo{TeX}\,Live\,2021下进行,内部代码主要由\lstinline|expl3|宏包提供的\hologo{LaTeX3}语法实现。

\section{目录结构}
\label{sec:directory}

\begin{lstlisting}[language=bash]
    │  njuthesis.bib
    │  njuthesis.cls
    │  njuthesis.tex
    │
    ├─figure
    │      njulogo.pdf
    │      njuname.pdf
    │
    └─profile
           abstract.sty
           components.sty
           cover.sty
           font.sty
           packages.sty
           page.sty
           reference.sty
\end{lstlisting}

\subsection{文件说明}
\begin{description}
    \item [\texttt{abstract.sty}] 定义中英文摘要环境及相关格式
    \item [\texttt{components.sty}] 定义图片、表格、代码、数学环境样式
    \item [\texttt{cover.sty}] 定义论文封面内容
    \item [\texttt{font.sty}] 重定义CTeX宏集的默认字体,因为CTeX字体加粗要求不满足论文规定
    \item [\texttt{package.sty}] 存放所有被引用的包
    \item [\texttt{page.sty}] 定义正文页面、标题以及目录条目格式
    \item [\texttt{reference.sty}] 定义引文格式
\end{description}

\section{自定义选项}
\label{sec:classoptions}

njuthesis类提供了若干个可以自定义的选项,如下所示。
\begin{description}
    \item[\texttt{titlelength}] 标题长度,用于控制封面标题的行数。可选值为1、2、3,缺省值为1。
    \item[\texttt{secondmentor}] 是否有第二导师,用于在封面和摘要添加第二导师,缺省值为false。
    \item[\texttt{nlcover}] 是否需要国家图书馆封面,缺省值为false。
    \item[\texttt{degree}] 学位类型,本科学位为ug,即undergraduate的缩写;学术型硕士学位为mg;硕士专业学位为mf;博士学位为phd。
    \item[\texttt{type}] 用于选择文档类型并将相关字段输出在封面和摘要页。可选值包括thesis(表示毕业论文)和design(表示毕业设计),缺省值为thesis。
    \item[\texttt{systemfont}] 是否用系统字体,用于自动检测运行系统并使用预装字体,缺省值为true。
    \item[\texttt{customlatinfont}] 用于选择自定义英文字符集。可选值包括windows, macos, gyre, null,缺省值为windows。
    \item[\texttt{customchinesefont}] 用于选择自定义中文字符集。可选值包括windows, macos, fandol, founder, noto, null,缺省值为windows。
\end{description}
