\chapter{配置环境}

\section{概述}
下表是目前经过测试的环境。如果有其他可用不可用的环境,欢迎补充。

\begin{table}[ht]
    \caption{经过测试的环境}
    % \label{tab:1}
    \begin{tabular}{ccc}
        \toprule
        OS & Tex & 测试情况 \\
        \midrule
        Windows 10 & \hologo{TeX}\,Live 2021 & ✔ \\
        Windows 10 & \hologo{MiKTeX} & ✔ \\
        Windows 10 & \hologo{TeX}\,Live 2020 & ✔  \\
        Ubuntu 20.04 & \hologo{TeX}\,Live 2021 & ✔ \\
        南大\hologo{TeX}\footnote{由于未知原因,南大\hologo{TeX}不能使用\hologo{XeLaTeX}编译,请务必选择\hologo{LuaLaTeX}进行编译。} & Overleaf & ✔ \\
        \bottomrule
    \end{tabular}
\end{table}

\section{本地编译}

\subsection{安装\hologo{TeX}发行版}

首先需要下载\hologo{TeX}软件发行版,校园网环境中使用\href{https://mirror.nju.edu.cn/download/app/TeX%20%E6%8E%92%E7%89%88%E7%B3%BB%E7%BB%9F}{南大镜像站}可以获得最好的体验。\textbf{推荐使用最新的\hologo{TeX}\,Live 2021或者\hologo{MiKTeX} 21以避免潜在的兼容性问题。}

\begin{itemize}
    \item 为了避免不必要的麻烦,请尽可能下载 full 版本,如 texlive-full。简而言之,下载大的那个。
    \item 并且,尽可能使用最新版(截至目前是 2021)。2020 及之前版本使用 PDF 格式的图片可能会出现加粗问题。
\end{itemize}

\subsection{配置前端}

推荐使用 VSCode + LaTeX Workshop(VSCode插件)完成论文编写,也可以使用其他编辑器,如 texworks、texstudio。

若使用 LaTeX Workshop 插件,本项目在\lstinline|.vscode/|中提供一份简易配置,可以直接使用。

\subsection{编译顺序}
应采用以下顺序进行编译,以生成正确的参考文献目录和编号。
\begin{enumerate}
    \item \hologo{XeLaTeX}/\hologo{LuaLaTeX}
    \item \hologo{biber}
    \item \hologo{XeLaTeX}/\hologo{LuaLaTeX}
    \item \hologo{XeLaTeX}/\hologo{LuaLaTeX}
\end{enumerate}

编译产物为\lstinline|njuthesis.pdf|。

\section{南大\hologo{TeX}在线编译}

本地编译器占用空间过大,对于磁盘空间紧张或者处理器性能有限的同学,不妨尝试本节介绍的在线编译方法\footnote{理论上在\href{https://doc.nju.edu.cn/books/latex}{这个网站}能找到一段平台简介,实际上大家都有意无意地鸽了,下次一定补上。}。

\subsection{平台简介}

\href{https://tex.nju.edu.cn}{南大\hologo{TeX}}基于开源的ShareLaTeX平台,于2021年3月4日正式上线,面向南京大学全体师生开放。

\subsection{操作步骤}

\begin{enumerate}
    \item 下载\href{https://github.com/nju-lug/NJUThesisUndergraduate/archive/refs/heads/master.zip}{模板全部文件}
    \item 访问\href{https://tex.nju.edu.cn}{南大\hologo{TeX}},点击界面右上方Register,使用\emph{南京大学邮箱}注册账号并登录
    \item 点击New Project -> Upload Project上传刚刚得到的zip文件,上传后njuthesis.tex、njuthesis.cls等文件应在根目录,目录结构如\cref{{sec:directory}}所示
    \item 在南大\hologo{TeX}项目内页面左上角的Menu中,将编译器改为\hologo{XeLaTeX}
    \item 编写论文
    \item 点击Compile按钮进行预览
\end{enumerate}

\section{字体}

学校论文格式要求的字体一般已经在电脑上包含,
本模板已对不同平台进行了适配。

在\cref{tab:fontset}中可以看到,不同系统上使用的字体有所差别,实际输出结果可能存在细微不同, 使用时请注意。出于美观考虑\footnote{以及出于编译速度考虑},不推荐在Windows平台进行编译\footnote{因为SimSun没有原生粗体,AutoFakeBold效果实在难堪大用},在Linux平台或者使用了Ubuntu后端的南大\hologo{TeX}上效果更佳。

% \begin{enumerate}
%     \item Ubuntu 下遇到缺失字体 WenQuanYi Zen Hei Mono 或 Times New Roman的问题:
    
%        安装对应字体即可。使用以下指令下载:
       
%        \begin{lstlisting}
% sudo apt install fonts-wqy-zenhei ttf-mscorefonts-installer
%         \end{lstlisting}

%     \item macOS 下提示 Package fontspec Warning: Font "STSong" does not contain requested Script "CJK"

%        忽略即可,不影响使用。该警告已被抑制。
% \end{enumerate}

\subsection{字体列表}

各个系统的默认字体请参考\cref{tab:fontset}。指定字体的相关命令写于\texttt{profile/font.sty}。
该文件中也预留有使用方正字体或者思源字体的命令,可根据个人喜好进行修改选择。

\begin{table}[htbp]
    \caption{默认字体清单}
    \label{tab:fontset}
    \begin{tabular}{cccc}
        \toprule
        类型 & Windows & macOS & Linux \\
        \midrule
        西文衬线 & Times~New~Roman & Times~New~Roman & texgyretermes \\
        西文无衬线 & Arial & Arial & texgyreheros \\
        西文等宽 & Courier~New & Courier~New & texgyrecursor \\
        宋体 & SimSun & Songti~SC~Light &FandolSong-Regular \\
        黑体 & SimHei & Heiti~SC~Light & FandolHei-Regular \\
        仿宋 & FangSong & STFangsong & FandolFang-Regular \\
        楷体 & KaiTi & Kaiti~SC & FandolKai-Regular \\
        \bottomrule
    \end{tabular}
\end{table}
