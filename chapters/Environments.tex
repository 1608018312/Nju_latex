\chapter{配置环境}

下表是目前经过测试的环境。如果有其他可用不可用的环境,欢迎补充。

\begin{table}[ht]
    \caption{经过测试的环境}
    % \label{tab:1}
    \begin{tabular}{ccc}
        \toprule
        OS & TeX & 测试情况 \\
        \midrule
        Windows 10 & \hologo{TeX}\,Live 2021 & 通过 \\
        Windows 10 & \hologo{MiKTeX} & 通过 \\
        Windows 10 & \hologo{TeX}\,Live 2020 & cref存在格式问题  \\
        macOS 10.15 & \hologo{TeX}\,Live 2021 & 通过 \\
        Ubuntu 20.04 & \hologo{TeX}\,Live 2021 & 通过 \\
        南大\hologo{TeX} & \hologo{TeX}\,Live 2021 & 通过 \\
        Overleaf & \hologo{TeX}\,Live 2020 & cref存在格式问题  \\
        \bottomrule
    \end{tabular}
\end{table}

\section{本地编译}

\subsection{安装\hologo{TeX}发行版}

首先需要下载\hologo{TeX}软件发行版,校园网环境中使用\href{https://mirror.nju.edu.cn/download/app/TeX%20%E6%8E%92%E7%89%88%E7%B3%BB%E7%BB%9F}{南大镜像站}可以获得最好的体验。\textbf{推荐使用最新的\hologo{TeX}\,Live 2021或者\hologo{MiKTeX} 21以避免潜在的兼容性问题。}

\begin{itemize}
    \item 为了避免不必要的麻烦,请尽可能下载 full 版本,如 texlive-full。简而言之,下载大的那个。
    \item 并且,尽可能使用最新版(截至目前是 2021)。2020 及之前版本使用 PDF 格式的图片可能会出现加粗问题。
\end{itemize}

\subsection{选择编辑器}

配置完编译器后,还需要一个\textbf{文本编辑器}作为前端来完成\texttt{.tex}文件内容的写作。

至今仍有相当一部分人认为Windows自带的\textit{记事本}是最好的文本编辑器,但对于本项目而言,在此诚心诚意地推荐你使用\textbf{更现代更美观更多功能}的编辑器,譬如\emph{安装了 LaTeX Workshop 插件 的 \href{https://code.visualstudio.com/}{Visual Studio Code}},来完成论文编写。你也可以根据个人的喜好随便使用其他编辑器,如 TeXworks、TeX Studio 等,顺手就行。

若使用 LaTeX Workshop 插件,本项目在\lstinline|.vscode/|中提供一份简易配置,可以省略初始配置步骤直接使用。

\subsection{编译顺序}
应采用以下命令顺序进行编译,以生成正确的目录、编号和参考文献条目。
\begin{enumerate}
    \item \lstinline|xelatex| / \lstinline|lualatex|
    \item \lstinline|biber|
    \item \lstinline|xelatex| / \lstinline|lualatex|
    \item \lstinline|xelatex| / \lstinline|lualatex|
\end{enumerate}

编译产物\footnote{作为化学学生,俺认为用“产物”代替“编译生成的文件”是一个通俗易懂的说法}为\lstinline|njuthesis.pdf|,位于主目录下。此外还会生成一系列中间文件,可以选择使用\lstinline|latexmk -c|进行清理。

\section{在线编译}

相信你在接触了本地编译以后,很快就会意识到一些十分显然的事实,譬如\hologo{TeX}编译器安装过程较为漫长,占用空间过大,而且在一部分处理器性能不佳的电脑上需要较长编译时间\footnote{其实这三点都是对广大的Windows用户说的,同一个模板在Linux编译可以节省一半耗时}。拒绝接受这些麻烦的同学不妨尝试本节介绍的在线编译方法。

\subsection{南大\hologo{TeX}平台简介}

\href{https://tex.nju.edu.cn}{南大\hologo{TeX}}基于开源的ShareLaTeX平台\footnote{理论上在\href{https://doc.nju.edu.cn/books/latex}{这个网站}能找到一段平台简介,实际上大家都有意无意地鸽了,下次一定补上。},于2021年3月4日正式上线,面向南京大学全体师生开放,首次使用需凭学校邮箱自助注册账号。

\subsection{操作步骤}

\begin{enumerate}
    \item 下载\href{https://github.com/nju-lug/NJUThesisUndergraduate/archive/refs/heads/master.zip}{模板全部文件}
    \item 访问\href{https://tex.nju.edu.cn}{南大\hologo{TeX}},点击界面右上方Register,使用\emph{南京大学邮箱}注册账号并登录
    \item 点击New Project -> Upload Project上传刚刚得到的zip文件,上传后njuthesis.tex、njuthesis.cls等文件应在根目录,目录结构如\cref{{sec:directory}}所示
    \item 在项目页面左上角的Menu中,将编译器改为\hologo{XeLaTeX}或者\hologo{LuaLaTeX}
    \item 编写论文
    \item 点击Compile按钮进行编译和预览
    \item 点击编译按钮右侧第三个按钮下载产物
\end{enumerate}

\subsection{关于Overleaf平台}

由于\href{https://www.overleaf.com/}{Overleaf平台}的\hologo{TeX}\,Live版本停留在2020,\texttt{cleveref}包在引用章节时会生成错误的标签,引发格式错误;而南大\hologo{TeX}通过及时更新规避了这一问题。因此\emph{请务必不要使用Overleaf官网进行编译}。

\section{字体}

学校论文格式要求使用的字体一般已经预装在各个操作系统,本模板针对不同平台进行了自动检测适配,可以开箱即用。

各个系统的默认字体请参考\cref{tab:defaultfontset}。可以看到,不同系统上使用的字体有所差别,实际输出结果可能存在细微不同, 使用时请注意。例如,在Linux平台或者使用了Ubuntu后端的南大\hologo{TeX}上,宋体加粗效果更明显;另一方面,在Windows平台进行编译的效果更接近Word加粗\footnote{因为SimSun没有原生粗体,通过AutoFakeBold=2.17进行模仿}。

% \begin{enumerate}
%     \item Ubuntu 下遇到缺失字体 WenQuanYi Zen Hei Mono 或 Times New Roman的问题:
    
%        安装对应字体即可。使用以下指令下载:
       
%        \begin{lstlisting}
% sudo apt install fonts-wqy-zenhei ttf-mscorefonts-installer
%         \end{lstlisting}

%     \item macOS 下提示 Package fontspec Warning: Font "STSong" does not contain requested Script "CJK"

%        忽略即可,不影响使用。该警告已被抑制。
% \end{enumerate}


指定字体的相关命令写于\texttt{profile/font.sty}。
该文件中也预留有使用方正字体或者思源字体的命令,涉及的字体见\cref{tab:userfontset},可根据个人喜好进行修改选择。

\begin{table}[htbp]
    \caption{操作系统预装字体清单}
    \label{tab:defaultfontset}
    \begin{tabular}{cccc}
        \toprule
        类型 & Windows & macOS & Linux \\
        \midrule
        西文衬线 & Times~New~Roman & Times~New~Roman & TeX~Gyre~Termes \\
        西文无衬线 & Arial & Arial & TeX~Gyre~Heros \\
        西文等宽 & Courier~New & Menlo & TeX~Gyre~Cursor \\
        宋体 & SimSun & Songti~SC~Light &FandolSong-Regular \\
        黑体 & SimHei & Heiti~SC~Light & FandolHei-Regular \\
        仿宋 & FangSong & STFangsong & FandolFang-Regular \\
        楷体 & KaiTi & Kaiti~SC & FandolKai-Regular \\
        \bottomrule
    \end{tabular}
\end{table}

如果 Ubuntu 下遇到缺失字体 Times New Roman 的问题,安装对应字体即可。使用以下指令下载:
\begin{lstlisting}
sudo apt install ttf-mscorefonts-installer
\end{lstlisting}

\begin{table}[htbp]
    \caption{预留的自定义中文字体清单}
    \label{tab:userfontset}
    \begin{tabular}{ccc}
        \toprule
        类型 & 方正 & 思源 \\
        \midrule
        宋体 & FZSSK & Noto~Serif~CJK~SC \\
        黑体 & FZHTK & Noto~Sans~CJK~SC \\
        仿宋 & FZFSK & 方正仿宋简体 \\
        楷体 & FZKTK & 方正楷体简体 \\
        \bottomrule
    \end{tabular}
\end{table}

\subsection{修改字体配置}

如果希望覆盖检测系统字体的命令(譬如在Linux编译时使用Windows字体样式),在\texttt{profile/font.sty}中注释掉操作系统判断语句,并保留希望使用的字符集即可。全部字符集命令如下所示。

\begin{description}
    \item[\texttt{\textbackslash set\textunderscore latin\textunderscore fontset\textunderscore windows}] Windows英文字符集
    \item[\texttt{\textbackslash set\textunderscore chinese\textunderscore fontset\textunderscore windows}] Windows中文字符集
    \item[\texttt{\textbackslash set\textunderscore latin\textunderscore fontset\textunderscore macos}] macOS英文字符集
    \item[\texttt{\textbackslash set\textunderscore chinese\textunderscore fontset\textunderscore macos}] macOS中文字符集
    \item[\texttt{\textbackslash set\textunderscore latin\textunderscore fontset\textunderscore gyre}] Linux英文字符集
    \item[\texttt{\textbackslash set\textunderscore chinese\textunderscore fontset\textunderscore fandol}] Linux中文字符集
    \item[\texttt{\textbackslash set\textunderscore chinese\textunderscore fontset\textunderscore founder}] 方正中文字符集
    \item[\texttt{\textbackslash set\textunderscore chinese\textunderscore fontset\textunderscore noto}] 思源中文字符集
\end{description}
