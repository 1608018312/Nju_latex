\chapter{数学公式与定理}

\section{公式示例}
\begin{equation}\label{eq:dewitt}
    \int  e^{ax} \tanh bx\ dx = 
    \begin{cases}
    \displaystyle{ \frac{ e^{(a+2b)x}}{(a+2b)} 
    {_2F_1}\left[ 1+\frac{a}{2b},1,2+\frac{a}{2b}, -e^{2bx}\right] }& \\
    \displaystyle{
    \hspace{1cm}-\frac{1}{a}e^{ax}{_2F_1}\left[ 1, \frac{a}{2b},1+\frac{a}{2b}, -e^{2bx}\right]
    }
     & a\ne b \\
    \displaystyle{\frac{e^{ax}-2\tan^{-1}[e^{ax}]}{a} } & a = b
    \end{cases}
\end{equation}
    
你可以使用\lstinline|equation|环境插入公式,如\cref{eq:dewitt},代码如下:
\begin{lstlisting}[language=TeX]
\begin{equation}\label{eq:dewitt}
    \int  e^{ax} \tanh bx\ dx = 
    \begin{cases}
    \displaystyle{ \frac{ e^{(a+2b)x}}{(a+2b)} 
    {_2F_1}\left[ 1+\frac{a}{2b},1,2+\frac{a}{2b}, -e^{2bx}\right] }& \\
    \displaystyle{
    \hspace{1cm}-\frac{1}{a}e^{ax}{_2F_1}\left[ 1, \frac{a}{2b},1+\frac{a}{2b}, -e^{2bx}\right]
    }
        & a\ne b \\
    \displaystyle{\frac{e^{ax}-2\tan^{-1}[e^{ax}]}{a} } & a = b
    \end{cases}
\end{equation}
\end{lstlisting}