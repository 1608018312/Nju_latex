\chapter{数学公式与定理}

\section{公式示例}
\begin{equation}\label{eq:dewitt}
    \int  e^{ax} \tanh bx\ dx = 
    \begin{cases}
    \displaystyle{ \frac{ e^{(a+2b)x}}{(a+2b)} 
    {_2F_1}\left[ 1+\frac{a}{2b},1,2+\frac{a}{2b}, -e^{2bx}\right] }& \\
    \displaystyle{
    \hspace{1cm}-\frac{1}{a}e^{ax}{_2F_1}\left[ 1, \frac{a}{2b},1+\frac{a}{2b}, -e^{2bx}\right]
    }
     & a\ne b \\
    \displaystyle{\frac{e^{ax}-2\tan^{-1}[e^{ax}]}{a} } & a = b
    \end{cases}
\end{equation}
    
你可以使用\lstinline|equation|环境插入公式,如\cref{eq:dewitt},代码如下:
\begin{lstlisting}[language=TeX]
\begin{equation}\label{eq:dewitt}
    \int  e^{ax} \tanh bx\ dx = 
    \begin{cases}
    \displaystyle{ \frac{ e^{(a+2b)x}}{(a+2b)} 
    {_2F_1}\left[ 1+\frac{a}{2b},1,2+\frac{a}{2b}, -e^{2bx}\right] }& \\
    \displaystyle{
    \hspace{1cm}-\frac{1}{a}e^{ax}{_2F_1}\left[ 1, \frac{a}{2b},1+\frac{a}{2b}, -e^{2bx}\right]
    }
        & a\ne b \\
    \displaystyle{\frac{e^{ax}-2\tan^{-1}[e^{ax}]}{a} } & a = b
    \end{cases}
\end{equation}
\end{lstlisting}

\section{定理环境}

\begin{proof}
    证明我是我
\end{proof}

\begin{definition}[他人]
    定义他人即地狱
\end{definition}

全部数学环境如下所示

\begin{table}[htbp]
    \caption{数学环境}
    \label{tab:mathenv}
    \begin{tabular}{cc}
        \toprule
        标签 & 名称 \\
        \midrule
        algorithm & 算法 \\
        assumption & 假设 \\
        axiom & 公理 \\
        conclusion & 结论 \\
        condition & 条件 \\
        corollary & 推论 \\
        definition & 定义 \\
        proof & 证明 \\
        example & 例 \\ 
        lemma & 引理 \\
        property & 性质 \\
        proposition & 命题 \\
        remark & 注解 \\
        theorem & 定理 \\
        \bottomrule
    \end{tabular}
\end{table}