\chapter{数学公式与定理}

\section{公式示例}
\begin{equation}\label{eq:dewitt}
    \int \mathrm{e}^{ax} \tanh {bx} \, \mathrm{d}x =
    \begin{dcases}
        \begin{multlined}
            \frac{\mathrm{e}^{(a+2b)x}}{(a + 2b)} \,
            {{}_2F_1} \left( 1 + \frac{a}{2b}, 1, 2+\frac{a}{2b}, -\mathrm{e}^{2bx} \right) \\
            - \frac{1}{a} \mathrm{e}^{ax} \, {{}_2F_1} \left( 1, \frac{a}{2b}, 1 + \frac{a}{2b}, -\mathrm{e}^{2bx} \right)
        \end{multlined}
        & a \ne b \\
    \frac{e^{ax} - 2 \tan^{-1}(\mathrm{e}^{ax})}{a} & a = b
  \end{dcases}
\end{equation}

你可以使用\lstinline|equation|环境插入公式,如\cref{eq:dewitt},代码如下:
\begin{lstlisting}[language=TeX]
\begin{equation}\label{eq:dewitt}
    \int \mathrm{e}^{ax} \tanh {bx} \, \mathrm{d}x =
    \begin{dcases}
        \begin{multlined}
            \frac{\mathrm{e}^{(a+2b)x}}{(a + 2b)} \,
            {{}_2F_1} \left( 1 + \frac{a}{2b}, 1, 2+\frac{a}{2b}, -\mathrm{e}^{2bx} \right) \\
            - \frac{1}{a} \mathrm{e}^{ax} \, {{}_2F_1} \left( 1, \frac{a}{2b}, 1 + \frac{a}{2b}, -\mathrm{e}^{2bx} \right)
        \end{multlined}
        & a \ne b \\
    \frac{e^{ax} - 2 \tan^{-1}(\mathrm{e}^{ax})}{a} & a = b
    \end{dcases}
\end{equation}
\end{lstlisting}

使用mathtools包会与unicode-math包产生一些命令的冲突,直接引用会输出以下两条警告
\begin{lstlisting}
Using \overbracket and \underbracket from `mathtools' package. Use \Uoverbracket and \Uunderbracket for original `unicode-math' definition.
I'm going to overwrite the following commands from the `mathtools' package: \dblcolon, \coloneqq, \Coloneqq, \eqqcolon.
\end{lstlisting}
这些警告已被手动抑制。

\section{定理环境}

\begin{proof}
    证明我是我
\end{proof}

\begin{definition}[他人]
    定义他人即地狱
\end{definition}

全部数学环境如下所示

\begin{table}[htbp]
    \caption{数学环境}
    \label{tab:mathenv}
    \begin{tabular}{cc}
        \toprule
        标签 & 名称 \\
        \midrule
        algorithm & 算法 \\
        assumption & 假设 \\
        axiom & 公理 \\
        conclusion & 结论 \\
        condition & 条件 \\
        corollary & 推论 \\
        definition & 定义 \\
        example & 例 \\ 
        lemma & 引理 \\
        proof & 证明 \\
        property & 性质 \\
        proposition & 命题 \\
        remark & 注解 \\
        theorem & 定理 \\
        \bottomrule
    \end{tabular}
\end{table}
