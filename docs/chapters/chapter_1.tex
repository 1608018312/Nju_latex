\chapter{声学拓扑晶体绝缘体的理论建模}
\section{引言}

\section{电声类比方法}
\subsection{声容}
声容的推导从绝热近似的状态方程出发,假设声波传播过程为绝热过程,则

\[
pV^\gamma = \text{常数},
\]

其中 \(p\) 为气体压力,\(V\) 为气体体积,\(\gamma = \frac{C_p}{C_v}\) 为绝热指数。对该公式取对数后对时间求导,得到

\[
V^\gamma \frac{dp}{dt} + \gamma p V^{\gamma-1} \frac{dV}{dt} = 0,
\]

整理为

\[
\frac{dp}{p} + \gamma \frac{dV}{V} = 0.
\]

假设声波的压力和体积变化为微小量,即

\[
p = p_0 + \Delta p, \quad V = V_0 + \Delta V,
\]

代入后忽略高阶小量项,得

\[
\frac{\Delta p}{p_0} = -\gamma \frac{\Delta V}{V_0}.
\]

从中可得体积变化和压力变化之间的关系为

\[
\Delta V = -\frac{V_0}{\gamma p_0} \Delta p.
\]

声容 \(C_a\) 定义为单位压力变化引起的体积变化:

\[
C_a = \frac{\Delta V}{\Delta p}.
\]

将上述关系式代入后得到

\[
C_a = \frac{V_0}{\gamma p_0}.
\]

进一步结合声速的定义

\[
c^2 = \frac{\gamma p_0}{\rho_0},
\]

将 \(\gamma p_0\) 替换为 \(\rho_0 c^2\),最终得声容的表达式

\[
C_a = \frac{V}{\rho_0 c^2}.
\]

这一公式揭示了声容取决于腔体体积 \(V\)、介质密度 \(\rho_0\) 和声速 \(c\),其中声容越大表示腔体对压力变化的响应越显著。

\subsection{声质量}

声质量的推导从牛顿第二定律开始。假设在声学系统中,颈管内的空气柱被视为振动的质点,其运动遵循牛顿第二定律:

\[
F = M \frac{d^2 x}{dt^2},
\]

其中 \( F \) 为作用在空气柱上的力,\( M \) 为空气柱的质量,\( x \) 为空气柱的位移。对于声波传播,作用力由声压差产生,可以表示为:

\[
F = \Delta p \cdot A,
\]

其中 \( \Delta p \) 为两端声压差,\( A \) 为颈管的截面积。空气柱的质量 \( M \) 表示为:

\[
M = \rho_0 l_{\text{eff}} A,
\]

其中 \( \rho_0 \) 为空气的静态密度,\( l_{\text{eff}} \) 为颈管的有效长度,考虑了端点修正(通常为 \( l + 1.7r \),其中 \( r \) 是颈管半径)。将 \( F \) 和 \( M \) 表达式代入牛顿第二定律,得到:

\[
\Delta p \cdot A = \rho_0 l_{\text{eff}} A \frac{d^2 x}{dt^2}.
\]

简化为:

\[
\Delta p = \rho_0 l_{\text{eff}} \frac{d^2 x}{dt^2}.
\]

由体积流量 \( U \) 的定义,\( U = A \frac{dx}{dt} \),两次对时间求导后,得到空气柱的加速度与体积流量的关系:

\[
\frac{d^2 x}{dt^2} = \frac{1}{A} \frac{dU}{dt}.
\]

代入上述公式,得到:

\[
\Delta p = \rho_0 l_{\text{eff}} \frac{1}{A} \frac{dU}{dt}.
\]

整理后可得空气柱的声质量表达式:

\[
L_a = \frac{\rho_0 l_{\text{eff}}}{A}.
\]

这一公式表明,声质量 \( L_a \) 是由空气的静态密度 \(\rho_0\)、颈管的有效长度 \(l_{\text{eff}}\) 和截面积 \(A\) 决定的,其中 \( L_a \) 越大,表示空气柱对流量变化的惯性越强。

\subsection{声阻}
声阻的推导从牛顿第二定律和流体动力学的基本方程出发,描述了声波传播时介质中摩擦力引起的阻力。假设声波通过一个窄管时,声阻力来源于空气柱在运动过程中与管壁之间的黏性摩擦。牛顿第二定律为:

\[
F = M \frac{d^2 x}{dt^2},
\]

其中 \( F \) 是作用在空气柱上的总力,\( M \) 是空气柱的质量,\( x \) 是空气柱的位移。在考虑摩擦力的情况下,总力 \( F \) 包含两个分量:由压力差引起的驱动力 \( F_p = \Delta p \cdot A \) 和与速度成正比的阻力 \( F_r = R_a \frac{dx}{dt} \),其中 \( R_a \) 是声阻。由此,总力可以表示为:

\[
\Delta p \cdot A - R_a \frac{dx}{dt} = M \frac{d^2 x}{dt^2}.
\]

空气柱的质量 \( M \) 表达为:

\[
M = \rho_0 l A,
\]

其中 \( \rho_0 \) 为空气密度,\( l \) 为空气柱的长度,\( A \) 为管道的截面积。将体积流量 \( U = A \frac{dx}{dt} \) 代入,声压差与流量的关系变为:

\[
\Delta p = R_a \frac{U}{A} + \rho_0 l \frac{1}{A} \frac{dU}{dt}.
\]

忽略惯性项(即空气柱的质量效应较小的情况下),可以得到声阻的定义:

\[
R_a = \frac{\Delta p}{\frac{U}{A}}.
\]

对于圆形管道,声阻可以进一步结合流体动力学中的黏性摩擦公式推导,其值为:

\[
R_a = \frac{8 \mu l}{\pi r^4},
\]

其中 \( \mu \) 是空气的动态黏性系数,\( l \) 是管道的长度,\( r \) 是管道的半径。这一公式表明,声阻 \( R_a \) 与介质的黏性和管道的几何尺寸(长度和半径)直接相关,管道越窄,声阻越大,黏性越强,声阻也越大,从而对体积流量的变化产生显著的阻碍作用。

\section{零阶模式的声学哈密顿量}

在文章中,哈密顿量的推导较为详细,主要涉及以下几个步骤和公式:

文章中首先通过声学-电学类比来定义声学系统的 lumped 参数模型。在该模型中,腔体被表示为电容 \( C \),波导被表示为电感 \( L_v \) 和 \( L_w \)。具体公式如下:
\[
C = \frac{V}{\rho c^2}
\]
\[
L_v = \frac{\rho (l_v + 1.7r_v)}{\pi r_v^2}, \quad L_w = \frac{\rho (l_w + 1.7r_w)}{\pi r_w^2}
\]
其中,\( V \) 是腔体的体积,\( \rho \) 是空气的密度,\( c \) 是声速,\( r_v \) 和 \( r_w \) 分别是波导的半径,\( l_v \) 和 \( l_w \) 是它们的长度。


通过应用基尔霍夫电流定律,文章得到了描述声学波动的方程。对于周期性结构,腔体的声压在每个晶格点处满足如下方程:
\[
- (2w + 2v) u_1^0 + w u_2^0 + w u_3^0 + v u_2^6 + v u_3^1 = \omega^2 u_1^0
\]
\[
- (2w + 2v) u_2^0 + w u_3^0 + w u_1^0 + v u_3^2 + v u_1^3 = \omega^2 u_2^0
\]
\[
- (2w + 2v) u_3^0 + w u_1^0 + w u_2^0 + v u_1^4 + v u_2^5 = \omega^2 u_3^0
\]
其中,\( u_n^m \) 表示第 \( n \) 个腔体在第 \( m \) 个晶格上的声压,\( w = -\frac{1}{L_v C} \),\( v = -\frac{1}{L_w C} \) 是与腔体和波导的物理参数相关的系数,\( \omega \) 是角频率。


通过上述方程,可以将哈密顿量表示为矩阵形式。在周期性结构下,哈密顿量 \( H_0(k) \) 以矩阵的形式表示为:
\[
H_0(k) =
\begin{pmatrix}
-2w - 2v & w + v e^{i k \cdot (a_1 + a_2)} & w + v e^{i k \cdot a_1} \\
w + v e^{-i k \cdot (a_1 + a_2)} & -2w - 2v & w + v e^{-i k \cdot a_2} \\
w + v e^{-i k \cdot a_1} & w + v e^{i k \cdot a_2} & -2w - 2v
\end{pmatrix}
\]
这里,\( k \) 是布里渊区内的波矢,\( a_1 \) 和 \( a_2 \) 是晶格常数。


考虑到实际声学系统的边界条件,文章进一步推导了边界对哈密顿量的影响。对于有限结构,硬边界和软边界条件分别影响哈密顿量的对角项。

- **硬边界条件**:对于硬边界,腔体的对角项变为:
  \[
  - (2w + v)
  \]
  
- **软边界条件**:对于软边界,腔体的对角项变为:
  \[
  - (2w + 2v)
  \]


文章进一步通过计算系统的拓扑极化,来揭示哈密顿量与拓扑相之间的关系。极化 \( p \) 的计算通过下式给出:
\[
e^{-i \pi p} = \prod_{n \in \text{occ}} \theta_n(K) \theta_n(\Gamma)
\]
其中,\( \theta_n(k) = \langle u_n(k) | R_3 | u_n(k) \rangle \) 是通过三重对称操作 \( R_3 \) 计算得到的本征值,\( n \) 是占据的能带,\( K \) 和 \( \Gamma \) 是高对称点。


通过上述推导,文章揭示了软边界条件如何保留系统的拓扑特性,特别是角落态的存在,这些角落态与电子系统中的零能态相对应。

总结而言,哈密顿量的推导通过声学电学类比、基尔霍夫电流定律以及周期性结构的布洛赫波函数方法,详细描述了声学拓扑晶体绝缘体的物理特性,同时揭示了边界条件对拓扑态的关键作用。

\section{一阶模式的声学哈密顿量}

为了审慎地展示Kitaev链的严格声学对应性,我们详细展示了完整的推导过程。

我们从一个单声学腔开始,该腔体与单元腔中连接的管道相连,如主文图1(a)所示(图S1中也提供了更详细的信息)。该腔体对应的经典格林函数定义为 \( G(\vec{r}, \vec{r}') \)。

当声波传播频率 \( \omega \) 位于特定模式下(即 \( P_z \)-偶极模式),其声速势场的归一化分布为:
\[
\psi(\vec{r}) = \sqrt{\frac{2}{w^2 h}} \sin\left(\frac{\pi x}{h}\right),
\]
如主文所示,此时 \( G(\vec{r}, \vec{r}') \) 可以简化为:
\[
G(\vec{r}, \vec{r}') = \sum_j \frac{c_0^2}{\omega_j^2 - \omega^2} \psi_j(\vec{r}) \psi_j(\vec{r}')
\approx \frac{c_0^2}{\omega_0^2 - \omega^2} \psi(\vec{r}) \psi(\vec{r}'). 
\]

其中,\( \psi(\vec{r}) \) 是位置 \( \vec{r} \) 处的声速势,\( \omega_0 \) 是 \( P_z \)-偶极模式的固有频率,该模式表现为腔内的驻波。需要强调的是,当 \( \omega \) 接近 \( \omega_0 \) 时,上述近似是成立的。

因此,腔体中声压的场分布可以表示为:
\[
p(\vec{r}) = -i \rho \omega \oint_{\partial V} G(\vec{r}, \vec{r}') \vec{v}(\vec{r}') \mathrm{d}S'
= -\frac{i \rho c_0^2 \omega}{\omega_0^2 - \omega^2} \oint_{\partial V} \psi(\vec{r}') \vec{v}(\vec{r}') \mathrm{d}S', 
\]
其中 \( \vec{v}(\vec{r}') \) 表示声学速度,\( S' \) 表示腔体的表面积。

由于声学硬边界条件使得 \( \vec{v} = 0 \),方程(S2)中的积分仅在管道的两端不为零。因此,有效面积 \( S' \) 可表示为:
\[
S' = S^t + 2S^A + S^\mu,
\]
其中上标分别表示对应的管道区域,如图S1(b)中的虚线区域所示。

---

为了进一步区分单元腔体中的两个腔体的参数,我们引入下标 \( j = 1, 2 \),并定义Kitaev链中波函数的关键参数 \( \xi = [\xi_1, \xi_2]^T \)。它们与Kitaev链中的波函数相关联,可以定义为:
\[
\xi_1 = \frac{p_1(\vec{r})}{v_1(\vec{r})} = \frac{-i \rho c_0^2}{\omega_0^2 - \omega^2} \oint_{\partial V_1} \psi^*(\vec{r}') \vec{v}_1(\vec{r}') \mathrm{d}S',
\]
\[
\xi_2 = \frac{p_2(\vec{r})}{v_2(\vec{r})} = \frac{-i \rho c_0^2}{\omega_0^2 - \omega^2} \oint_{\partial V_2} \psi^*(\vec{r}') \vec{v}_2(\vec{r}') \mathrm{d}S'. 
\]

将方程(S2)代入并结合体积和表面特性,可以分别表示为:
\[
\xi_1 = -\frac{c_0^2 d^2}{2(\omega_0^2 - \omega^2)} \left[ -\nu_1(0) \psi_1^A + \nu_1(0) \psi_1^A e^{-ika} + \nu_1^A(\psi_2^A e^{-ika}) \right],
\]
\[
\xi_2 = -\frac{c_0^2 d^2}{2(\omega_0^2 - \omega^2)} \left[ -\nu_2(0) \psi_2^A + \nu_2(0) \psi_2^A e^{-ika} + \nu_2^A(\psi_1^A e^{-ika}) \right]. 
\]

其中 \( V = w^2 h \) 是腔体的体积,\( \bar{\psi}_j^m \) 是与 \( j \)-th 腔连接的第 \( m \)-根管末端声速势的平均值。对于所有管道具有相同的横截面积和 \( r_c \) 值,由以下关系得出:
\[
\bar{\psi}_1 = \bar{\psi}_2 = -\bar{\psi}_1^t = -\bar{\psi}_2^t = \psi,
\]
因此可以得出 \( \bar{\psi}_j^m = |\psi| \)。同时需要注意,这种近似在管道尺寸远小于腔体尺寸的情况下是适用的。

根据这些结果,可以以 \( \xi \) 表示声压:
\[
p_j^t(0) = \xi_j \bar{\psi}_j, \quad
p_j^t(l_m^t) = \xi_j \bar{\psi}_j^t, \quad
p_j^t(f_j) = e^{ika} \xi_j \bar{\psi}_j^t,
\]
\[
p_1^A = e^{ika} \xi_2 \bar{\psi}_2^t, \quad
p_2^A = e^{ika} \xi_1 \bar{\psi}_1^t,
\]
\[
p_j^\mu = \xi_j \sqrt{\frac{2}{V}}. 
\]

为了映射严格的对应关系,我们现在重点分析管道的声学连接条件。假设结构内声波以平面波形式传播,则在一个单元腔中第 \( j \)-th 腔的声压和声速可以表示为:
\[
p_m^j(l) = A_m^j e^{i\omega l/c_0} + B_m^j e^{-i\omega l/c_0},
\]
\[
\rho_0 c_0 v_m^j(l) = A_m^j e^{i\omega l/c_0} - B_m^j e^{-i\omega l/c_0}, 
\]
其中 \( m \) 表示 \( t \) 和 \( \Delta \)。

通过将 \( l = 0 \) 和 \( l = l_m^t \) 代入方程(S4),可以很容易地得到以下关系:
\[
i\rho c_0
\begin{pmatrix}
v_j^t(0) \\
v_j^t(l_m^t)
\end{pmatrix}
=
\begin{pmatrix}
\cot(\omega l_m^t / c_0) & -\csc(\omega l_m^t / c_0) \\
-\csc(\omega l_m^t / c_0) & \cot(\omega l_m^t / c_0)
\end{pmatrix}
\begin{pmatrix}
p_j^t(0) \\
p_j^t(l_m^t)
\end{pmatrix}. 
\]

对于额外的管道(标记为 \( \mu \))且一端闭合,则满足以下关系:
\[
v_j^\mu = \frac{p_j^\mu}{Z_j^\mu},
\]
其中 \( Z_j^\mu \) 是连接腔体的端部阻抗。当闭合端为声学硬边界时:
\[
Z_j^\mu = -i\rho c_0 \cot(\omega l_j^\mu / c_0),
\]
而当闭合端为声学软边界时:
\[
Z_j^\mu = i\rho c_0 \tan(\omega l_j^\mu / c_0)。
\]

进一步地,将方程(S4)-(S7)代入方程(S3),波函数方程可以以矩阵形式表示为:
\[
\omega^2 \xi = (H_0 + H_a(k)) \xi, 
\]
其中:
\[
H_0 =
\begin{pmatrix}
\omega_0 + \epsilon_1 & 0 \\
0 & \omega_0 + \epsilon_2
\end{pmatrix},
\quad
H_a(k) =
\begin{pmatrix}
\mu_1 + 2t_1\cos(ka) & \Delta_1 e^{ika} + \Delta_2 e^{-ika} \\
\Delta_1 e^{-ika} + \Delta_2 e^{ika} & \mu_2 + 2t_2\cos(ka)
\end{pmatrix}. 
\]
其中
\[
\epsilon_1 = \frac{cd^2 |\psi|^2}{2} \left[ 2\cot\left(\frac{\omega_0 l_1^t}{c_0}\right) + \cot\left(\frac{\omega_0 l_1^A}{c_0}\right) + \cot\left(\frac{\omega_0 l_2^A}{c_0}\right) \right],
\]
\[
\epsilon_2 = \frac{cd^2 |\psi|^2}{2} \left[ 2\cot\left(\frac{\omega_0 l_2^t}{c_0}\right) + \cot\left(\frac{\omega_0 l_1^A}{c_0}\right) + \cot\left(\frac{\omega_0 l_2^A}{c_0}\right) \right],
\]
\[
t_1 = -\frac{cd^2 |\psi|^2}{2} \csc\left(\frac{\omega_0 l_1^t}{c_0}\right), \quad
t_2 = -\frac{cd^2 |\psi|^2}{2} \csc\left(\frac{\omega_0 l_2^t}{c_0}\right),
\]
\[
\Delta_1 = -\frac{cd^2 |\psi|^2}{2} \csc\left(\frac{\omega_0 l_1^A}{c_0}\right), \quad
\Delta_2 = -\frac{cd^2 |\psi|^2}{2} \csc\left(\frac{\omega_0 l_2^A}{c_0}\right),
\]
\[
\mu_1 = \frac{i\rho c_0^2 d^2}{V Z_1^\mu}, \quad
\mu_2 = \frac{i\rho c_0^2 d^2}{V Z_2^\mu}. 
\]
注意:一旦声学结构确定,所有这些参数都可以直接计算。需要特别指出的是,方程(S10)表明,所有由 \( P_z \) 模式描述的关键强度参数(\( t, \Delta, \mu \))是解耦的,这与由声学系统谐振基频描述的跃迁不同,因此可以独立设计。

为了构造严格的声学Kitaev链,需要满足以下条件:
\[
l_1^t = l_2^t + h, \quad t_1 = -t_2 = -t, \quad
l_1^A = l_2^A + h,\quad \Delta_1 = -\Delta_2 = -\Delta,
\]
并且 \( Z_1^\mu = -Z_2^\mu \),当 \( \mu_1 = -\mu_2 = -\mu \) 时,这些设置自然确保 \( \epsilon_1 = \epsilon_2 = \epsilon \)。因此,方程(S9)可以简化为:
\[
H_0 =
\begin{pmatrix}
\omega_0 + \epsilon & 0 \\
0 & \omega_0 + \epsilon
\end{pmatrix},
\quad
H_a(k) =
\begin{pmatrix}
-\mu - 2t\cos(ka) & -2i\Delta\sin(ka) \\
2i\Delta\sin(ka) & \mu + 2t\cos(ka)
\end{pmatrix}. 
\]

