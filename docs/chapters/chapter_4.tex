\chapter{基于声学拓扑晶体绝缘体的彩虹捕获研究}
\section{引言}
在上一章节中,我们讨论了亚波长尺度下,腔管结构的声学拓扑绝缘体的构造方法,并观察了其高阶拓扑态。一方面,对于如上一章节提到的无自旋并且时间反演不变的TCI而言,其独特的拓扑态是由体极化来进行表征的。这种特性使得拓扑边界态在界面处有很强的鲁棒性。另一方面,我们的声学建模为在声学系统中实现这一类拓扑绝缘体提供了理论基础,我们可以根据理论模型里的相互作用,设计出对应尺寸的声学结构。如此一来,声学TCI便自然而然地为利用特定的能量(也就是特定的频率)在结构的边界或者角落之处,以一种鲁棒的方式收集声波提供了一个极为理想的平台。

然而,尽管已经有大量细致入微的研究,无论是从理论层面还是实验层面,都确凿地证明了声学系统中可以构造各种各样的拓扑态,但是,对于拓扑态本身的传播性质(例如如群速度)相关的内容,却鲜少被人们深入地探讨和研究。实际上,这些传播特性在TCI的实际应用当中,是一个至关重要且无法回避的问题。以能量收集为例,利用拓扑态来实现彩虹捕获就是一个典型的需要调控声波传播的群速度的应用场景。

所谓的彩虹捕获,指的是将具有不同频率的波进行分离,并且让它们分别在不同的空间位置上被捕获。这种现象已经在多个领域的体系中被提出,比如在电磁波系统\cite{C41-1,C41-2,C41-3,C41-4,C41-5,C41-6,C41-7,C41-8,C41-9}、弹性波系统\cite{C42-1,C42-2,C42-3,C42-4,C42-5,C42-6}以及空气声系统\cite{C43-1,C43-2,C43-3,C43-4}中。在这些相关的研究工作里,研究人员们纷纷采用了各种各样的方法,其目的就是为了能够控制不同系统中波的色散关系,从而使得波能够在空间的不同位置上实现局域化。

不仅如此,一些近期的研究工作更是已经在拓扑态中成功地实现了彩虹捕获\cite{C44-1,C44-2,C44-3,C44-4,C44-5,C44-6}。在这些研究中,隙内模式能够在不减小体带隙的情况下被有效地减慢,而且体带隙依然能够受到强无序的良好保护。不过,令人遗憾的是,现有的这些方法通常都需要在体中设置额外的区域或者对晶格进行相应的变化。例如,在半无限大的非平庸介质和半无限大的平庸的介质之间构造一个界面,而拓扑边界态出现在两种介质的交界出,亦或是在平庸的环境中嵌入非平庸的介质,使得拓扑边界态出现在嵌入的非平庸介质周界。然而,在上一章节里,我们揭示了仅需要调控非平庸的声学结构以及边界条件,即可仅用一种介质实现拓扑边界态。进一步,从实际应用的角度出发,我们还需要一种全新的方法,这种方法能够直接对边界处拓扑态的传播速度进行有效的控制。

因此,本章工作的目的是提出一种控制TCI中不同频率下声波传播速度的方法,以实现声学彩虹捕获。基于二维SSH模型,我们严格证明了经典波系统的阻抗与紧束缚模型的跳跃项和在位项之间的对应关系,这表明,无需对体结构进行变形,通过仅仅调节超材料的表面阻抗,就可以精确且独立地控制边界态的群速度。这一特性天然为拓扑保护波捕获提供了一个新的思路,使得我们能利用边界态的鲁棒性的同时,也能大大减小结构的尺寸。由此我们通过实验展示了一个亚波长拓扑彩虹集中器的原型,其理论、仿真和实验结果显示这一结构可以时不同频率的声波停驻在拓扑绝缘体边界的不同位置。

本章主要内容如下:首先在4.2节中,介绍了二维SSH模型的哈密顿量以及实际结构的拓扑边界态和角态,引出体、边、角三个位置的在位势不同;接着4.3节探讨经典波系统中表面阻抗对拓扑态的影响,包括具有附加表面阻抗的一维 SSH 模型、平庸体态和拓扑态的解析解、经典波系统的限制条件以及拓扑态存在的边界条件;然后4.4节阐述改变拓扑边界态的群速度的相关内容,并展示用以上方法改变二维SSH模型拓扑边界态群速度的可行性;4.5节则聚焦于拓扑彩虹捕获的实现,从仿真和实验实现了声学彩虹陷波;最后4.6节对上述内容进行小结。

\section{二维SSH模型的拓扑性质}

\subsection{理论模型的哈密顿量}
\begin{figure}[h!]
    \centering
    \includegraphics[width=1\textwidth]{images/fig4-1.eps} 
    \caption{(a) 二维SSH模型的示意图。(b-d)二维SSH模型的理论体能量带结构:(b) $v/w = 1$、(c) $v/w = 0.5$和(d) $v/w = 2$,“+/−”符号表示在高对称点处态的偶(+)奇(−)宇称。}
    \label{fig_4_1}
  \end{figure}  

在这部分,我们展示二维SSH紧束缚模型与声波系统之间对应关系的推导,并介绍其能带结构。首先,我们研究从如图 \ref{fig_4_1} (a)所示的典型二维SSH模型开始,其中的结构满足$C_4$对称,$w$为胞内跳跃,$v$为胞间跳跃。用上一章节介绍的电子系统和声学系统类比思路,我们使用腔体模拟原子,用管道模拟格点跳跃,对应的声学腔管结构如图 \ref{fig_4_1} (a)插图所示。其中,正方体腔体的边长位$a_c$,胞内和胞间导管的长度分别为$l_w$和$l_v$,半径分别为$r_w$和$r_v$。

在这里,我们定义在$(i,j)$晶格处的波函数(在声学系统里,可用声压作波函数)为$\Psi_{i,j} = [\psi_{i,j}^{(1)}, \psi_{i,j}^{(2)}, \psi_{i,j}^{(3)}, \psi_{i,j}^{(4)}]^T$,其中上标表示如文中图 \ref{fig_4_1} (a)所标记的原子。通过上一章节展示的等效电路方法和基尔霍夫电流定律,可以得到方程
\begin{equation}
    \begin{aligned}
    -\frac{\psi_{(i,j)}^{(1)}}{Z_c} &= \frac{\psi_{(i,j)}^{(1)} - \psi_{(i,j)}^{(3)}}{Z_w} + \frac{\psi_{(i,j)}^{(1)} - \psi_{(i,j + 1)}^{(4)}}{Z_v} + \frac{\psi_{(i,j)}^{(1)} - \psi_{(i - 1,j)}^{(3)}}{Z_v} + \frac{\psi_{(i,j)}^{(1)} - \psi_{(i,j)}^{(4)}}{Z_w}, \\
    -\frac{\psi_{(i,j)}^{(2)}}{Z_c} &= \frac{\psi_{(i,j)}^{(2)} - \psi_{(i,j + 1)}^{(4)}}{Z_v} + \frac{\psi_{(i,j)}^{(2)} - \psi_{(i + 1,j)}^{(3)}}{Z_w} + \frac{\psi_{(i,j)}^{(2)} - \psi_{(i,j)}^{(4)}}{Z_w} + \frac{\psi_{(i,j)}^{(2)} - \psi_{(i,j - 1)}^{(3)}}{Z_v}, \\
    -\frac{\psi_{(i,j)}^{(3)}}{Z_c} &= \frac{\psi_{(i,j)}^{(3)} - \psi_{(i + 1,j)}^{(4)}}{Z_v} + \frac{\psi_{(i,j)}^{(3)} - \psi_{(i,j + 1)}^{(1)}}{Z_v} + \frac{\psi_{(i,j)}^{(3)} - \psi_{(i,j)}^{(1)}}{Z_w} + \frac{\psi_{(i,j)}^{(3)} - \psi_{(i,j)}^{(2)}}{Z_w}, \\
    -\frac{\psi_{(i,j)}^{(4)}}{Z_c} &= \frac{\psi_{(i,j)}^{(4)} - \psi_{(i,j)}^{(2)}}{Z_w} + \frac{\psi_{(i,j)}^{(4)} - \psi_{(i,j)}^{(1)}}{Z_w} + \frac{\psi_{(i,j)}^{(4)} - \psi_{(i - 1,j)}^{(2)}}{Z_v} + \frac{\psi_{(i,j)}^{(4)} - \psi_{(i,j - 1)}^{(1)}}{Z_v},
    \end{aligned}
    \label{eq4-1}
\end{equation}

由于$Z_w = \mathrm{j}\omega L_w$,$Z_v = \mathrm{j}\omega L_v$且$Z_c = 1/\mathrm{j}\omega C$,我们可以分别定义$w = -1/(L_wC)$和$v = -1/(L_vC)$。然后我们将所考虑的晶格标记为0,其左、下、右和上的最近邻晶格分别标记为$i = 1,2,3,4$。相应地,方程(\ref{eq4-1})可重写为
\begin{equation}\label{eq4-2}
    \omega^2\Psi_0 = \hat{H}_0\Psi_0 + \sum_{i = 1}^{4} \hat{H}_i\Psi_i, 
\end{equation}
其中
\begin{equation}\label{eq4-3}
    \hat{H}_0 = 
    \begin{pmatrix}
    -2w - 2v & 0 & w & w \\
    0 & -2w - 2v & w & w \\
    w & w & -2w - 2v & 0 \\
    w & w & 0 & -2w - 2v
    \end{pmatrix}, 
\end{equation}
\begin{equation}\label{eq4-4}
    \hat{H}_1 = 
    \begin{pmatrix}
    0 & 0 & v & 0 \\
    0 & 0 & 0 & 0 \\
    0 & 0 & 0 & 0 \\
    0 & v & 0 & 0
    \end{pmatrix},
    \hat{H}_2 = 
    \begin{pmatrix}
    0 & 0 & 0 & 0 \\
    0 & 0 & v & 0 \\
    0 & 0 & 0 & 0 \\
    v & 0 & 0 & 0
    \end{pmatrix},
    \hat{H}_3 = 
    \begin{pmatrix}
    0 & 0 & 0 & 0 \\
    0 & 0 & 0 & 0 \\
    0 & 0 & 0 & v \\
    0 & 0 & 0 & 0
    \end{pmatrix},
    \hat{H}_4 = 
    \begin{pmatrix}
    0 & 0 & 0 & v \\
    0 & 0 & 0 & 0 \\
    0 & 0 & 0 & 0 \\
    0 & 0 & 0 & 0
    \end{pmatrix}, 
\end{equation}
进一步,通过应用傅里叶展开
\begin{equation}\label{eq4-5}
    \Psi_0 = \sum_{k} \psi_k e^{i\vec{k} \cdot \vec{r}_0},
\end{equation}
\begin{equation}\label{eq4-6}
    \Psi_i = e^{i\vec{k} \cdot \vec{r}_i} \sum_{k} \psi_k e^{i\vec{k} \cdot \vec{r}_0},
\end{equation}
其中$i = 1,2,3,4$表示晶格0的左、下、右和上的最近邻晶格,$\vec{r}_i$是从晶格0指向它们的位置矢量。方程最终可以写成如下的形式
\begin{equation}\label{eq4-7}
    \omega_k^2 \psi_k = \hat{H}_k \psi_k,
\end{equation}
其中紧束缚模型的哈密顿量可表示为:
\begin{equation}
\hat{H}_k = 
\begin{pmatrix}
-2w - 2v & 0 & w + v\mathrm{e}^{\mathrm{j}k_x} & w + v\mathrm{e}^{-\mathrm{j}k_y} \\
0 & -2w - 2v & w + v\mathrm{e}^{\mathrm{j}k_y} & w + v\mathrm{e}^{-\mathrm{j}k_x} \\
w + v\mathrm{e}^{-\mathrm{j}k_x} & w + v\mathrm{e}^{-\mathrm{j}k_y} & -2w - 2v & 0 \\
w + v\mathrm{e}^{\mathrm{j}k_y} & w + v\mathrm{e}^{\mathrm{j}k_x} & 0 & -2w - 2v
\end{pmatrix},
\label{eq4-8}
\end{equation}
相同的在位项不会影响体带的拓扑结构,并且该晶格仍可被视为保持手性对称性和\(C_4\)对称性,即:
\begin{equation}
[\Pi, \hat{H}_k + (2w + 2v)\mathbb{I}_{4\times4}] = 0, \quad [R_4, \hat{H}_k] = 0,
\label{eq4-9}
\end{equation}
其中\(\Pi\)和\(R_4\)分别是手性算符和旋转算符。

\subsection{拓扑边界态和角态}

本小节研究二维SSH模型的边界态和角态。通过求解方程(\ref{eq4-7}),不同\(v/w\)的体晶格的能带结构分别如图 \ref{fig_4_1} (b-d)所示。可以看出,在高对称点上,第一条和第二条能带间的带隙随着参数变化经过了打开,闭合再打开的过程,并且其态的宇称发了反转。具体而言,与二维Zak相相关的拓扑相变发生在\(w/v=1\)处。在此我们于图 \ref{fig_4_1}(b)中展示了临界点处的能谱。在该临界点,四个能带在\(\Gamma\)点和\(X (Y)\)点相接触。由于子晶格对称性,\ref{fig_4_1}(c)图和(d)图能带反转后的能带结构相同。能带的拓扑性质由它们在高对称点处的宇称编码,这些宇称在图中标记为“\(\pm\)”。如图 \ref{fig_4_1}所示,\(X (Y)\)点处最低\(s\)能带的宇称在能带反转后改变符号,这表明发生了拓扑相变。这种拓扑相变可由二维扩展Zak相来表征,其也是由以下表达式给出的极化\cite{C45-1}:
\begin{equation}
    \symbfit{P}=\frac{1}{2\pi}\int \mathrm{d}k_x \mathrm{d}k_y \symbfit{Tr}[\symbfit{A}(k_x,k_y),
    \label{eq4-10}
\end{equation}
其中\(\symbfit{A}=\langle\psi|i\partial_{\symbfit{k}}|\psi\rangle\)是贝里联络,积分是在第一布里渊区上进行的。反演对称性对\(\symbfit{P}\)的值施加了很强的约束,其由\(\Gamma\)点和\(X (Y)\)点处的宇称独立于规范确定,如下\cite{C45-2}:
\begin{equation}
    \mathcal{P}_i = \sum_{n}^{occ} q_i^n \bmod 2, \quad (-1)^{q_i^n} = \frac{\eta_n(X)}{\eta_n\Gamma},
    \label{eq4-11}
\end{equation}
其中\(\eta\)表示宇称,求和是对所有占据能带进行的,\(i\)代表\(x\)或\(y\)。由于\(C_4\)对称性,我们有\(\mathcal{P}_x = \mathcal{P}_y\)。基于式(\ref{eq4-11}),能带反转后对于两个拓扑非平凡带隙可得到\(\symbfit{P}=(1/2,1/2)\),这表明在带隙内结构的边界和角落处可能存在局域的拓扑态。更关键的是,对于实际声学结构的边界晶格和角晶格,如图 \ref{fig_4_1}(a)的上中晶格和上右晶格,由于内在的手性对称性破缺,有限结构哈密顿量中的相应部分可分别得到为:
\begin{equation}
\hat{H}_k^{edge} = 
\begin{pmatrix}
-2w - v & 0 & w & w \\
0 & -2w - 2v & w & w \\
w & w & -2w - v & 0 \\
w & w & 0 & -2w - 2v
\end{pmatrix},
\label{eq4-12}
\end{equation}
\begin{equation}
\hat{H}_k^{corner} = 
\begin{pmatrix}
-2w - v & 0 & w & w \\
0 & -2w - v & w & w \\
w & w & -2w & 0 \\
w & w & 0 & -2w - 2v
\end{pmatrix},
\label{eq4-13}
\end{equation}
因此,我们立即发现边界原子的不等在位势,实际上对应于广义手性对称性破缺\cite{i5},意味着在实际物理系统中边界局域态的能量移动。

除了拓扑边界态以外,上述体系还可能出现高阶拓扑角态。由$C_4$对称性保护的第$n$个能带的高阶拓扑仍可由偶极矩$\mathcal{P}^{(n)} = (\mathcal{P}_x^{(n)}, \mathcal{P}_y^{(n)})$表征。可定义一个角诱导拓扑四极矩指数为\cite{C45-3}
\begin{equation}
    \mathcal{Q}_{xy} = \sum_{n = 1}^{\text{occ}} \mathcal{P}_x^{(n)} \mathcal{P}_y^{(n)},
    \label{eq4-14}
\end{equation}
对于此处有两个占据能带的情况,当$v > w$时,$\mathcal{P}_x^{(1)} = \mathcal{P}_y^{(1)} = \mathcal{Q}_{xy} = 0.5$,这表明在具有开放边界的有限结构的边界和角落处预计会出现拓扑边界态和角落态。

\subsection{实际结构的拓扑边界态和角态}

\begin{figure}[h!]
    \centering
    \includegraphics[width=1\textwidth]{images/fig4-2.eps} 
    \caption{(a) 实际声学模型的示意图。(b) 周期性结构的能谱。(c) 有限结构的能谱。(d) 在下带隙中孤立的边界局域态。(e) 与体态简并的“零能”角局域态。}
    \label{fig_4_2}
  \end{figure}  

在这一小节中,借助有限元仿真工具COMSOL Multiphysics,我们给出了实际声学结构的拓扑态的能带结构和模式。我们的结构参数设置为:$a_c = 30$ mm、$l_w = 120$ mm、$l_v = 20$ mm、$r_w = 1.5$ mm和$r_v = 4$ mm。这里,连接腔体的晶内和晶间电感分别定义为$L_w$和$L_v$,而腔体的电容为$C$,可根据结构的几何参数获得。$\omega$是布洛赫波函数的角频率。对于体晶格,动量空间中本征值问题可写为方程(\ref{eq4-7})。在当前情况下,跳跃项和阻抗之间的严格对应关系可直接得到为$w = -1/L_wC$和$v = -1/L_vC$(参见上一章节)。这里得到的结果为$w = -2.51\times 10^5$ Hz$^2$且$v = -8.17\times 10^6$ Hz$^2$。图 \ref{fig_4_2}(b)展示了该结构周期排列时的能带图,可以看到在180 Hz到610 Hz和630 Hz到850 Hz这两个频率范围内,结构有两个主要的禁带。而对于有限大结构而言,禁带频率范围内却有特征模式,如图 \ref{fig_4_2}(c)展示。其中黑点对于着体能带的模式,禁带中的蓝点是边界态模式,而红点是埋在体能带中的角态。图 \ref{fig_4_2} (d)和(e)则给出了对应频率下的声场模式,结果显示声场能量分别集中在边界上和角落上。

理论模型的一个内在考量是,即使在开放边界条件下,手性对称性也能完美保持,因此边界局域化模式预计会被固定到零能量(能隙中间);即使在许多实验研究中,在位势也总是被视为独立的。然而,如方程(\ref{eq4-12})和(\ref{eq4-13})所示,在位势与实际物理系统中的跳跃项强耦合。因此,对于有限结构,如果在边界晶格中没有严格的补偿,手性对称性的破坏将不可避免地导致在位势的偏移。例如,对于具有绝对硬壁边界条件的$C_4$对称声学系统或具有完美磁导体类似边界条件的光学系统,表面阻抗为$Z = \infty$,这使得边界晶格中的这些项为$\varepsilon^{\text{edge}} = -2w - v$,而角落晶格中原子的在位势为$\varepsilon^{\text{corner}} = -2w$。更重要的是,在相同条件下,最外层管道的长度直接决定它们的阻抗。从这个角度来看,我们可以改变相应位置管道的长度来改变位能,进而影响拓扑态的性质,例如拓扑边界态的传播速度或者是拓扑角态的频率,这将在下一节中详述。

\section{经典波系统中表面阻抗对拓扑态的影响}

在电子系统中,我们往往把真空看作是一种平庸介质,对非平庸的拓扑绝缘体,当存在某种空间对称性时其拓扑边界态出现在其表面,并且由于对称性保护拓扑角态出现在“零能”位置。但是,从上一小节可以看到,对实际的声学系统而言,表面阻抗的存在会影响非平庸结构的边界。虽然可观测的拓扑态通常被视作源自体拓扑,且被认为对于来自内部无序或平凡环境的干扰具备鲁棒性,然而在实际的物理系统里,若缺乏严格的边界补偿,手性对称性在本质上是被破坏的,这必然会致使边界在位势发生偏移,进而对拓扑态的出现产生影响。在本部分内容中,基于声学系统,我们深入且细致地分析了拓扑态存在的条件,并给出了这些态随着由表面阻抗表征的不同补偿所产生的能量偏移情况。 

\subsection{具有附加表面阻抗的一维SSH模型}

\begin{figure}[h!]
    \centering
    \includegraphics[width=1\textwidth]{images/fig4-3.eps} 
    \caption{(a) 具有N个晶胞的一维SSH模型的示意图。(b) 等效电路图。}
    \label{fig_4_3}
  \end{figure}  

在此,不失一般性,我们从二维的结构退化到一维的结构,从一个简单的具有\(N\)个单元的常规一维SSH模型开始来说明附加表面阻抗的影响,如图 \ref{fig_4_3}(a)所示。我们分别将单元内和单元间阻抗定义为\(Z_{aa}\)和\(Z_{ab}\),原子的本征阻抗为\(Z_{c}\)。对于体中的第\(n\)个单元,原子\(a\)和\(b\)位置处的波函数可写为
\begin{equation}\label{eq4-15}
    \begin{split}
    -\frac{\psi_{n,a}}{Z_{c}} &= \frac{\psi_{n,a} - \psi_{n,b}}{Z_{w}} + \frac{\psi_{n,a} - \psi_{n - 1,b}}{Z_{v}},\\
    -\frac{\psi_{n,b}}{Z_{c}} &= \frac{\psi_{n,b} - \psi_{n,a}}{Z_{w}} + \frac{\psi_{n,b} - \psi_{n + 1,a}}{Z_{v}},
    \end{split}
\end{equation}
因此,经典波系统中的单元内和单元间跳跃项可分别定义为\(w = \epsilon Z_{c} / Z_{aa}\)和\(v = \epsilon Z_{c} / Z_{ab}\),其中\(\epsilon\)与波函数\([\psi_{n,a},\psi_{n,b}]^T\)的频率\(\omega\)有关。把式(\ref{eq4-15})重新整理,可重写为
\begin{equation}\label{eq4-16}
    \begin{split}
    (e + w + v)\psi_{n,a} &= w\psi_{n,b} + w\psi_{n - 1,b},\\
    (e + w + v)\psi_{n,b} &= w\psi_{n,a} + w\psi_{n + 1,a},
    \end{split}
\end{equation}
对于边界单元,即第一个单元和第\(N\)个单元,我们在末端的原子上施加一个额外阻抗\(Z_{add}=Z_{v}+Z\),这在实际声学结构中非常常见,因为实际结构边界处的表面阻抗总是需要仔细考量的。相应单元中的波函数则可表示为
\begin{equation}\label{eq4-17}
    \begin{split}
    (\epsilon + w + \Delta_0)\psi_{1,a} &= w\psi_{1,b},\\
    (\epsilon + w + \Delta_0)\psi_{N,b} &= w\psi_{N,a},
    \end{split}
    \end{equation}
其中\(\Delta_0 = \epsilon Z_{c} / Z_{add} = vZ_{v} / (Z_{v} + Z)\)。值得注意的是,在实际物理系统中,原子可由纯电容\(C\)表征,原子间的相互作用可由纯电感\(L_{\alpha} (\alpha = w,v)\)表征。因此,我们得到关系\(Z_{c} = 1 / i\omega C\)和\(Z_{\alpha} = i\omega L_{\alpha}\)。此外,从\(\Delta_0\)的定义可以容易地得出\(Z = i\omega L\)应归因于纯电感,如图 \ref{fig_4_3}(b)所示。这是由于腔管结构的表面往往是控制导管来控制表面阻抗,否则实际结构可能无法和电子系统中线性的本征值问题很好地对应上,而这在本工作中未被考虑。结果,式(\ref{eq4-17})最终可重写为
\begin{equation}\label{eq4-18}
    \begin{split}
    (\epsilon + w + v + \Delta)\psi_{1,a} &= w\psi_{1,b},\\
    (\epsilon + w + v + \Delta)\psi_{N,b} &= w\psi_{N,a},
    \end{split}
\end{equation}
值得一提的是,由于在实际结构中\(L \in [0,\infty]\),我们得到\(\Delta \in [-v,0]\)。

\subsection{平庸体态和拓扑态的解析解}

实际上,系统的哈密顿量可直接从式(\ref{eq4-15})和(\ref{eq4-16})导出。然而,这种方法难以区分平庸体态和拓扑态。在此,我们使用一种直接方法,通过获得解析解来讨论拓扑态的存在。我们使用第\(n\)个单元中波函数的通解为\cite{i5}
\begin{equation}\label{eq4-19}
    \begin{split}
        \psi_{n,a}=A_{1}\gamma^{n}+A_{2}\gamma^{-n},\\
        \psi_{n,b}=B_{1}\gamma^{n}+B_{2}\gamma^{-n},
    \end{split}
\end{equation}
其中\(\{A_{1},A_{2},B_{1},B_{2}\}\)是正向和反向平面波的幅值系数。因此,若\(\gamma=\xi e^{i\theta}(\theta\in[-\pi,\pi])\)为虚数则表示体中具有周期性场分布的平凡态,而\(\gamma\)为实数(\(\gamma\neq\pm1\))表示从边界向体中指数衰减的拓扑态。我们将通解(\ref{eq4-19})代入式(\ref{eq4-16}),得到
\begin{equation}\label{eq4-20}
    \begin{split}
        {P}_{1}\gamma^{n}-\mathcal{P}_{2}\gamma^{-n}=0,\\
        {P}_{3}\gamma^{n}-\mathcal{P}_{4}\gamma^{-n}=0,
    \end{split}
\end{equation}
其中
\begin{equation}\label{eq4-21}
    \begin{split}
        {P}_{1}=\epsilon'A_{1}-wB_{1}-vB_{1}\gamma^{-1},\\
        {P}_{2}=\epsilon'A_{2}-wB_{2}-vB_{2}\gamma,\\
        {P}_{3}=\epsilon'B_{1}-wA_{1}-vA_{1}\gamma^{-1},\\
        {P}_{4}=\epsilon'B_{2}-wA_{2}-vA_{2}\gamma^{-1},
    \end{split}
\end{equation}
其中\(\epsilon' = w + v + \epsilon\)。然后我们可以得到\({P}_{i}=0\ (i = 1,2,3,4)\)的关系,以满足每个体单元中的波函数。结果,我们得到
\begin{equation}\label{eq4-22}
    \epsilon'\begin{bmatrix}B_{1}\\A_{1}\\A_{2}\\B_{2}\end{bmatrix}=\begin{bmatrix}\mathcal{D}_{2\times2}&\mathcal{O}\\\mathcal{O}&\mathcal{D}_{2\times2}\end{bmatrix}\begin{bmatrix}B_{1}\\A_{1}\\A_{2}\\B_{2}\end{bmatrix},
\end{equation}
其中
\begin{equation}\label{eq4-23}
    \mathcal{D}_{2\times2}=\begin{bmatrix}0&w + v\gamma\\w + v\gamma^{-1}&0\end{bmatrix},
\end{equation}
对于矩阵\(\mathcal{D}_{2\times2}\),存在两个不相等的非零特征值,即
\begin{equation}\label{eq4-24}
    \begin{split}
        \epsilon_{1}' = -\sqrt{(w + v\gamma)(w + v\gamma^{-1})},\\
        \epsilon_{2}' = -\epsilon_{1}'
    \end{split}
\end{equation}
因此,对于式(\ref{eq4-22}),应当有两对双重特征值,分别对应两对独立的特征向量。然后我们将与\(\mathcal{D}_{2\times2}\)的特征值\(\epsilon'\)对应的特征向量定义为\(\symbfit{u} = [u_{1}\ u_{2}]^{T}\),式(\ref{eq4-22})的特征向量空间最终得到
\begin{equation}\label{eq4-25}
    \begin{bmatrix}B_{1}\\A_{1}\\A_{2}\\B_{2}\end{bmatrix}=\alpha\begin{bmatrix}u_{1}\\u_{2}\\0\\0\end{bmatrix}+\beta\begin{bmatrix}0\\0\\u_{1}\\u_{2}\end{bmatrix}=\begin{bmatrix}\alpha u_{1}\\\alpha u_{2}\\\beta u_{1}\\\beta u_{2}\end{bmatrix},
\end{equation}
其中\(\alpha\)和\(\beta\)是任意系数。因此,式(\ref{eq4-25})可重写为
\begin{equation}\label{eq4-26}
    \begin{split}
        \psi_{n,a}=\alpha u_{2}\gamma^{n}+\beta u_{1}\gamma^{-n},\\
        \psi_{n,b}=\alpha u_{1}\gamma^{n}+\beta u_{2}\gamma^{-n},
    \end{split}
\end{equation}
为了方便讨论,这里特征向量的基选择为\(u_{1}=w + v\gamma\)和\(u_{2}=\epsilon'\)。将式(\ref{eq4-26})代入式(\ref{eq4-20}),我们得到
\begin{equation}\label{eq4-27}
    \mathcal{F}\begin{bmatrix}x\\y\end{bmatrix}=\begin{bmatrix}0\\0\end{bmatrix},
\end{equation}
其中
\begin{equation}\label{eq4-28}
    \mathcal{F}=\begin{bmatrix}\{(\epsilon'+\Delta)u_{2}-wu_{1}\}\gamma&\{(\epsilon'+\Delta)u_{1}-wu_{2}\}\gamma^{-1}\\\{(\epsilon'+\Delta)u_{1}-wu_{2}\}\gamma^{N}&\{(\epsilon'+\Delta)u_{2}-wu_{1}\}\gamma^{-N}\end{bmatrix},
\end{equation}
方程存在的非平凡解要求\(\mathrm{det}(\mathcal{F}) = 0\),所以得到
\begin{equation}\label{eq4-29}
    w^{2}F(\gamma)=\{(\epsilon'+\Delta)u_{2}-wu_{1}\}^{2}\gamma^{-N + 1}-\{(\epsilon'+\Delta)u_{1}-wu_{2}\}^{2}\gamma^{N - 1}=0,
\end{equation}
将式(\ref{eq4-24})代入式(\ref{eq4-29}),\(\gamma\)可以很好地确定,并且由于\(F(\gamma)=F(-\gamma)\)的性质,很明显该方程有\(N\)对相反的根。同时,系数\([\alpha\ \beta]^{T}\)可定义为
\begin{equation}\label{eq4-30}
    \begin{bmatrix}\alpha\\\beta\end{bmatrix}=\begin{bmatrix}-\{(\epsilon'+\Delta)u_{2}+wu_{1}\}\gamma^{-1}\\\{(\epsilon'+\Delta)u_{1}-wu_{2}\}\gamma\end{bmatrix},
\end{equation}
结果,所有系数\(\{\alpha,\beta,\gamma\}\)都可以在\(\Delta\)定义的情况下直接得到。

\subsection{经典波系统的限制条件}

我们注意到\(\epsilon_{1}'\)和\(\epsilon_{2}'\)都可以确定\(\gamma\)的解析解。然而,在实际物理系统中,\(\Delta\in[-v,0]\)这一事实表明只有\(\epsilon_{1}'\)具有明确的物理意义,而\(\epsilon_{2}'\)在本工作中不做讨论。另一方面,我们认为\(\gamma=\pm1\)是一对不满足式(\ref{eq4-24})的异常解。然而,这两种情况可以从式(\ref{eq4-16})和(\ref{eq4-18})中精确求解。

(i)\(\gamma = 1\)情况。我们假设\(\psi_{n,a}=A\)且\(\psi_{n,b}=B\),分别将这些函数代入式(\ref{eq4-16}),然后我们立即得到\(\epsilon'=\pm(w + v)\)且\(A/B=\pm1\)。此外,式(\ref{eq4-18})得出
\begin{equation}\label{eq4-31}
    \epsilon' = w + v, \Delta = -v;
\end{equation}

(ii)\(\gamma = -1\)情况。我们假设\(\psi_{n,a}=(-1)^{n}A\)且\(\psi_{n,b}=(-1)^{n}B\),分别将这些函数代入式\ref{eq4-16},然后我们得到\(\epsilon'=\pm(w - v)\)且\(A/B=\pm1\)。此外,式(\ref{eq4-18})得出
\begin{equation}\label{eq4-32}
    \epsilon' = -w + v, \Delta = -v,
\end{equation}
如上所述,\(\gamma=\pm1\)是\(Z=\infty\)条件下的一对解。此外,对于体态\(\gamma=\xi e^{i\theta}\),很容易证明\(\xi^{2}=1\)。

\subsection{拓扑态存在的边界条件}

\begin{figure}[h!]
    \centering
    \includegraphics[width=1\textwidth]{images/fig4-4.eps} 
    \caption{当(a) $\Delta = 0$(声学绝对软壁边界条件)、(c) $\Delta = -0.8v$以及(e) $\Delta = -v$(声学绝对硬壁边界条件)时,具有不同复$\gamma$的无量纲$F(\gamma)$,分别如所示。当(b) $\Delta = 0$、(d) $\Delta = -0.8v$以及(f) $\Delta = -v$时,具有不同实$\gamma$的$F(\gamma)$,分别如所示。(d)和(f)中的箭头指示了具有不同$\Delta$时$\gamma$的移动方向。}
    \label{fig_4_4}
\end{figure} 

\begin{figure}[h!]
    \centering
    \includegraphics[width=1\textwidth]{images/fig4-5.eps} 
    \caption{当(a) $\Delta = 0$(声学绝对软壁边界条件)、(c) $\Delta = -0.5v$以及(e) $\Delta = -v$(声学绝对硬壁边界条件)时,由哈密顿量(蓝色圆圈)和解析方法(黑色点)得到的能谱,分别如所示。(b)、(d)、(f)分别为(a)、(c)和(e)中拓扑态(用黑色圆圈圈出)的相应场分布。能量单位为$Hz^2$,$\Psi$是无量纲的。}
    \label{fig_4_5}
\end{figure} 

如上所述,在实际物理系统中,\(\Delta\)的范围是从\(-v\)到\(0\)。值得注意的是,两个极端条件,即\(\Delta = -v\)和\(\Delta = 0\),分别对应于表面阻抗\(Z\)为\(\infty\)和\(0\)。例如,在经典波系统中,\(Z=\infty\)表示光学系统中类似完美磁导体的边界或声学系统中绝对硬壁边界,而\(Z = 0\)表示光学系统中类似完美电导体的边界或声学系统中绝对软壁边界。在下文中,我们基于声学系统讨论边界条件。

(i)绝对硬壁边界条件(\(Z=\infty\))——将\(\Delta = -v\)代入式(\ref{eq4-29})得出
\begin{equation}
    \gamma^{2N}=1,
\end{equation}
然后我们得到解为\(\gamma = e^{i\frac{2\pi}{N}i}\),其中\(i = 1,2,\cdots,N - 1\)。再加上根\(\gamma = 1\),我们最终得到所有\(N\)个正根,值得注意的是所有这些根都表示体态。进一步,解\(\gamma=\pm1\)实际上可以被视为与体态混合的简并拓扑态(如下所述)。

(ii)绝对软壁边界条件(\(Z = 0\))——在这种情况下,我们将\(\Delta = 0\)代入式(\ref{eq4-29}),得到
\begin{equation}\label{eq4-34}
(w + v\gamma^{-1})\gamma^{N + 1}-(w + v\gamma)\gamma^{-N - 1}=0,
\end{equation}
式(\ref{eq4-34})有\(2N + 2\)个根。除了如上所述的\(\gamma=\pm1\),我们最终得到所有\(2N\)个根。注意,当\(N\rightarrow\infty\)时,只有当\(v/w > 1\)时,才有两个极端解\(\gamma = \{-v/w, -w/v\}\),这是对应于拓扑态在非平凡系统中的典型解。此外,如正文所述,广义手性对称性得以完美保留,并且“零能”角态在主带隙中被完美钉扎。

(iii)更一般的条件(\(0 < Z < \infty\))——目前大多数关于经典波系统中拓扑绝缘体实现的工作总是构建一个拓扑平凡域,将其作为拓扑材料外部的“环境”。我们认为,尽管这种方法可以保证拓扑绝缘体的厄米性,但它仍然会导致内在的手性对称性破缺,并且仍然可以解释拓扑态不可观测的原因。也就是说,拓扑态的能量将不可避免地向下偏移,并且可能与体态简并或移入下带隙(例如,\(C_{4}\)结构)。

为了更清楚地解释这一点,当\(\Delta = 0\)、\(\Delta = -0.8v\)和\(\Delta = -v\)时\(F(\gamma)\)的复根和实根分别在图 \ref{fig_4_4} (a-f)中给出。可以看出,当\(Z\)从\(0\)(绝对软边界)变为\(\infty\)(绝对硬边界)时,所有实根都偏移到\(-1\)。相应地,图 \ref{fig_4_5} (a-f)分别显示了当\(\Delta = 0\)、\(\Delta = -0.5v\)和\(\Delta = -v\)时的能谱和理论计算的拓扑态。通过哈密顿量和解析方法求解的具有不同\(\Delta\)的偏移拓扑态分别在图 \ref{fig_4_5} (a-f)中用空心圆圈和实心点描绘。

如上所述,我们已经证明,只有当附加表面阻抗为\(Z = 0\)(绝对软壁边界条件)时,广义手性对称性才得以保留,而非零的\(Z\)将不可避免地导致边界处拓扑态的能量偏移。特别地,当\(Z = \infty\)(绝对硬壁边界条件)时,这些拓扑态将退化为体态中。因此,我们还可以得出结论,只要厄米拓扑绝缘体的非平庸拓扑得以保留,拓扑态将始终存在,并且在实际物理系统中仅影响结构表面阻抗的所谓无序不应影响这些态的存在。关键的是,这些结果可以推广到遵循不同对称性的其他结构,比如二维的SSH模型。

\section{改变拓扑边界态的群速度}

\begin{figure}[h!]
    \centering
    \includegraphics[width=1\textwidth]{images/fig4-6.eps} 
    \caption{(a) 在$y$方向截断且在x方向具有周期性的带状结构。最外层的管道用蓝色标记。(b) 带状结构的能谱。(c) 当$ka = 0$时拓扑边界态的声压场分布。(d) 具有不同l的带状结构的边界态能谱。(e) 279 Hz时,以$l$为自变量的边界态群速度的函数。}
    \label{fig_4_6}
\end{figure} 

在4.1节中我们研究了二维SSH模型的拓扑边界态的性质,而在4.2节中我们以一维SSH模型表面阻抗的影响。在本节中,我们将说明利用表面阻抗调整二维SSH模型边界态的具体方法。在前述研究中,我们发现内在的手性对称性破缺是致使拓扑态频率发生移动并改变其群速度的关键要素。尽管拓扑态的出现由体拓扑所决定,然而,鉴于边界态和角落态具有这样的特性:这些被困于边界晶格内的态呈指数衰减进入体中,故而拓扑态的能量(频率)主要取决于边界晶格表面原子的在位势。也就是说,拓扑态与平庸体态相互独立,即便它们可能处于简并状态。由此可见,这一特殊性质表明,通过控制在位势,即经典波系统中边界晶格的表面阻抗,能够精准且自由地对边界态和角落态的频率加以移动。

为了依据上述所探讨的理论模型进行阐释,我们构建了一个在$x$方向截断且在y方向具备周期性的带状结构,其晶胞如图 \ref{fig_4_6} (a)所示。我们通过改变最外层管道的长度$l$,人为地向上边界和下边界添加额外的在位势,详情如图 \ref{fig_4_6} (a)所示。体晶格的非平庸性质决定了在上边界和下边界会有无隙边界态,如图 \ref {fig_4_6} (b)所示。两个能隙中拓扑边界态的声压场分布如图 \ref{fig_4_6} (c)所示。然而,边界处在位势的差异致使其频率出现偏移。相应地,不同$l$时第一能隙中边界态的频率如图 \ref{fig_4_6} (d)所示。不难看出,边界态将始终存在于第一能隙中,但其频率会随$l$的变化而移动。类似于凝聚态物理,这种声学现象是由在位势对拓扑态的影响所引发,这在上一章节中通过一维SSH模型拓扑态的解析解得到了清晰的证实。此结果亦表明拓扑平庸环境并非不可或缺。

通过这种方式,能够精准地调谐边界态以捕获具有特定频率的波,且不会对体拓扑产生影响。这种特性彰显了TCIs作为拓扑保护的频率集中器的广阔应用前景。实际上,在图 \ref{fig_4_6}(d)中,我们画出来不同$l$时边界态的能带曲线,而这一系列的曲线给出了不同$l$对应的边界态的色散关系,据此可通过$c_g = d\omega/dk$计算边界态的群速度$c_g$。例如,图 \ref{fig_4_6}(e)展示了279Hz时边界态的群速度作为$l$的函数。值得注意的是,当$l$接近80 mm时,279 Hz处的群速度逐渐趋近于0。在此情形下,声波处于驻波状态,这意味着声波的能量被局域于特定位置。对于不同频率,群速度$c_g = 0$对应着不同的$l$。这为我们实现声学彩虹中不同频率的分离提供了理论支撑。二维SSH模型的另一个重要特征在于,只有当$v/w > 1$时才存在非平庸拓扑。当$|v|$远大于$|w|$时,能隙更宽,边界态呈指数衰减进入体中的速度更快;另外,晶格内原子的耦合振荡可近似忽略。因此,对于较大的$v/w$,成对的晶间原子在位势的偏移几乎不会对其他对产生影响,这意味着被局域于相应晶格内的拓扑态可进行独立调谐。

\section{拓扑彩虹捕获的实现}

\begin{figure}[h!]
    \centering
    \includegraphics[width=1\textwidth]{images/fig4-7.eps} 
    \caption{(a) 在一个7×7结构中,最外层管道的长度$l$随其位置而变化。(b) 彩虹集中器的本征频率谱。(c-f) 频率为(c) 268 Hz、(d) 408 Hz、(e) 534 Hz和(f) 615 Hz的边界局域态的移动。}
    \label{fig_4_7}
\end{figure} 

\begin{figure}[h!]
    \centering
    \includegraphics[width=1\textwidth]{images/fig4-8.eps} 
    \caption{(a) 实验样品的示意图。插图展示了放置扬声器或麦克风的孔。(b) 不同位置的测量强度谱。(c) 268Hz、(d) 408Hz、(e) 534Hz和(f) 615Hz时的声场结果。蓝点是声源的位置。}
    \label{fig_4_8}
\end{figure} 

在这一节里,我们基于表面阻抗改变拓扑边界态的原理,实现了声彩虹陷波。彩虹捕获,即把波的不同频率分量分离至不同空间位置,是通过调控复合材料中的色散来设计实现的。近期,一些其他研究借助散射型光子晶体中的合成维度,成功打造了光子晶体的拓扑彩虹集中器\cite{C44-3}。相较而言,在我们的设计中,声学TCI依托共振系统,这一特性使得结构尺寸能够远小于波长。此外,我们的模型精准对应离散的二维SSH模型,从而让整个过程更为便捷。我们能够依据所需的频率范围,确定哈密顿量中的各项参数,诸如跳跃项和在位势等。而后,依据理论模型中的这些参数,可计算出所需声学结构的阻抗,最终确定实际样品的几何尺寸,例如腔体的体积以及管道的长度与半径。

我们构建了一个有限的7×7结构,其中最外层管道的长度依其位置而变化,如图 \ref{fig_4_7}所示,以此来展现该结构在能隙中特定频率下驻留声波的能力。此结构的本征频率如图 \ref{fig_4_7} (b)所示。从中可以看出,在150至620 Hz的禁带中,一系列边界态赫然出现。在我们的设计里,声波停止的位置会随频率而改变,如图 \ref{fig_4_7} (c-f)所示。以上仿真结果和我们的设计预期一致。

最后,我们运用3D打印技术制作了如图 \ref{fig_4_8} (a)所示的样品,从实验的角度进一步验证我们的观点。为了将声源安置于特定位置或测量特定位置的声场,我们在相应位置的腔体上开孔,将扬声器或麦克风置入其中。其中声波由声源(Hivi B2S)激发,样品内的声压振幅由直径为1/4英寸的Brüel\&Kjær type-4944麦克风测量。所有数据均由分析仪(Brüel\&Kjær PULSE Type 3160)处理。当声源被放置于边界的不同位置后,通过扫频来测量相邻腔体的频率响应。图 \ref{fig_4_8} (b)展示了不同位置的测量强度谱。对于特定位置而言,倘若声波在某一频率下的群速度为零,那么该位置处声波的能量便会被局域化,进而在该位置和该频率处会出现一个强度峰值。能量强度峰值会随着频率$f$和位置编号$n$而移动,进而形成图 \ref{fig_4_8} (b)中的亮条纹。这条亮条纹应当与由图 \ref{fig_4_8}(b)中所有满足$c_g(n,f)=0$的点$(n,f)$所构成的曲线相吻合,其中$c_g$被视为频率$f$和位置编号$n$的函数。满足$c_g(n,f)=0$的曲线也可从用于求解本征模的模拟计算中获取。在模拟计算中,每个本征模都对应一个频率$f$和一个位置编号$n$,其中声场能量密度达到最大,这些点$(n,f)$便形成了图 \ref{fig_4_8} (b)中的蓝线。这条蓝线处于图 \ref{fig_4_8}(b)的亮条纹之中,这与我们的预期完全相符。为了确定实际结构的本征模,我们在相应位置(蓝点)以及相应频率(268、408、534和615 Hz)处激发声源,并测量所有腔体中的声场,如图 \ref{fig_4_8} (c-f)所示。图 \ref{fig_4_8} (c-f)清晰地表明,测量所得的声场与图\ref{fig_4_7} (c-f)中的模拟结果高度一致。

\begin{figure}[h!]
    \centering
    \includegraphics[width=1\textwidth]{images/fig4-9.eps} 
    \caption{(a) 最外层管道的长度$l$随其位置而变化。声源放置在位置1处的腔体中。插图展示了实验配置。(b) 不同频率的声波传播。(c) 不同频率下的声压场图。}
    \label{fig_4_9}
\end{figure} 

依据图 \ref{fig_4_8} (b)可知,一旦表面阻抗在边界上呈梯度连续分布,便可实现拓扑保护的宽带彩虹捕获\cite{C41-1,C41-2,C41-3}。声学彩虹对不同频率声波的局域化能力,在固定声源的激发下得以充分展现。我们采用5×15个晶胞来设计一个非平庸界面,其中最外层管道的长度$l$与腔体中位置的关系如图 \ref{fig_4_9} (a)所示。声源被放置于腔体中的位置1。我们测量了边界处腔体在275至295Hz频率范围内的声能。不同频率的声波在晶格边界处的传播距离如图 \ref{fig_4_9} (b)所示。从中能够看出,不同频率的声波传播距离各不相同。例如,我们对整个结构在不同频率(275、285和295 Hz)下的声场分布进行了测量,如图 \ref{fig_4_9} (c)所示。正如我们所见,频率越高,声波传播得越远。这一特性使得不同频率的声波能够被有效分离,进而形成声学彩虹。

与其他研究工作相比,我们直接对边界态的群速度加以改变,仅在边界上进行操作,而无需改变体晶格。在调节彩虹捕获的过程中,大部分整个结构无需变动。彩虹捕获发生于非平庸结构的边界处,且无需额外的区域。此外,我们的模型基于共振而非散射,这对亚波长结构的设计大有裨益。模型的所有参数均可通过声学参数来计算,这有助于在给定预期工作频率的情况下,顺利设计出相应结构。

\section{小结}

总而言之,在这一章节里,我们围绕拓扑边界态和表面阻抗的关系,提出拓扑边界态的频率和群速度可通过在位势来控制,并基于声学拓扑集中器的原型实现了宽带彩虹捕获。我们将经典波系统中的概念与凝聚态物理理论相联系,证明了在边界处调节声学拓扑绝缘体的表面阻抗是控制拓扑态的有效方式,这在理论模型中对应于调节边界处的在位势。我们从一维的SSH模型出发,详细推导解释了附加表面阻抗对拓扑态频率和模式的影响。以此推广到二维系统,我们通过改变结构最外侧的管长,实现了附加表面阻抗随着空间位置变化的设计。沿着这个思路,我们实现了边界态群速随着空间位置和频率的变化而变化的效果,最终得到了宽带彩虹捕获声波的效果。有限元仿真和实验结构都符合我们的预期,说明了我们理论的有效性。除此之外,我们获得了结构的几何参数与理论模型中的哈密顿量之间的严格数学关系,这有助于更高效地设计特定的声学器件。基于这些考量,我们实现了拓扑彩虹捕获的边界态的拓扑性质使其相较于传统方法更具鲁棒性,而结构的共振特性使得亚波长和宽带操作更易于实现。我们希望这项工作能对与空间调制相关的声学器件有所助益,并为拓扑材料的设计和应用提供一个多功能平台。