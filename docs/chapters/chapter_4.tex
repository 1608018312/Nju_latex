\chapter{基于$C_4$对称的声学拓扑晶体绝缘体的声传播与彩虹捕获研究}
\section{引言}

对于那些无自旋并且时间反演不变的拓扑晶体绝缘体而言,其独特的拓扑态是由体极化来进行表征的。这种特性使得它在经典波系统中相对来说更容易得以实现,如此一来,便自然而然地为利用特定的能量(也就是特定的频率)在结构的边界或者角落之处,以一种鲁棒的方式俘获波提供了一个极为理想的平台。

然而,尽管已经有大量细致入微的研究,无论是从理论层面还是实验层面,都确凿地证明了拓扑晶体绝缘体(TCIs)中存在着拓扑态,但是,像拓扑态的群速度这类与传播特性相关的内容,却鲜少被人们深入地探讨和研究。可实际上,这些传播特性在拓扑晶体绝缘体的实际应用当中,却是一个至关重要且无法回避的问题。

举个例子来说,利用拓扑态来实现彩虹俘获就是一个典型的应用场景。所谓的彩虹俘获,指的是将具有不同频率的波进行分离,并且让它们分别被俘获在不同的空间位置上。这种现象已经在多个领域的系统中被提出,比如在电磁波系统\cite{C41-1,C41-2,C41-3,C41-4,C41-5,C41-6,C41-7,C41-8,C41-9}、弹性波系统\cite{C42-1,C42-2,C42-3,C42-4,C42-5,C42-6}以及空气声系统\cite{C43-1,C43-2,C43-3,C43-4}中。在这些相关的研究工作里,研究人员们纷纷采用了各种各样的方法,其目的就是为了能够控制不同系统中波的色散关系,从而使得波能够在空间的不同位置上实现局域化。

不仅如此,一些近期的研究工作更是已经在拓扑态中成功地实现了彩虹俘获\cite{C44-1,C44-2,C44-3,C44-4,C44-5,C44-6}。在这些研究中,隙内模式能够在不减小体带隙的情况下被有效地减慢,而且体带隙依然能够受到强无序的良好保护。不过,令人遗憾的是,现有的这些方法通常都需要在体中设置额外的区域或者对晶格进行相应的变化。从实际应用的角度出发,我们迫切地需要一种全新的方法,这种方法能够直接对边界处拓扑态的传播速度进行有效的控制。