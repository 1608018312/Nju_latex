\chapter{声学彩虹}
\section{引言}

\section{一阶模式的声学哈密顿量}

为了审慎地展示Kitaev链的严格声学对应性,我们详细展示了完整的推导过程。

我们从一个单声学腔开始,该腔体与单元腔中连接的管道相连,如主文图1(a)所示(图S1中也提供了更详细的信息)。该腔体对应的经典格林函数定义为 \( G(\vec{r}, \vec{r}') \)。

当声波传播频率 \( \omega \) 位于特定模式下(即 \( P_z \)-偶极模式),其声速势场的归一化分布为:
\[
\psi(\vec{r}) = \sqrt{\frac{2}{w^2 h}} \sin\left(\frac{\pi x}{h}\right),
\]
如主文所示,此时 \( G(\vec{r}, \vec{r}') \) 可以简化为:
\[
G(\vec{r}, \vec{r}') = \sum_j \frac{c_0^2}{\omega_j^2 - \omega^2} \psi_j(\vec{r}) \psi_j(\vec{r}')
\approx \frac{c_0^2}{\omega_0^2 - \omega^2} \psi(\vec{r}) \psi(\vec{r}'). 
\]

其中,\( \psi(\vec{r}) \) 是位置 \( \vec{r} \) 处的声速势,\( \omega_0 \) 是 \( P_z \)-偶极模式的固有频率,该模式表现为腔内的驻波。需要强调的是,当 \( \omega \) 接近 \( \omega_0 \) 时,上述近似是成立的。

因此,腔体中声压的场分布可以表示为:
\[
p(\vec{r}) = -i \rho \omega \oint_{\partial V} G(\vec{r}, \vec{r}') \vec{v}(\vec{r}') \mathrm{d}S'
= -\frac{i \rho c_0^2 \omega}{\omega_0^2 - \omega^2} \oint_{\partial V} \psi(\vec{r}') \vec{v}(\vec{r}') \mathrm{d}S', 
\]
其中 \( \vec{v}(\vec{r}') \) 表示声学速度,\( S' \) 表示腔体的表面积。

由于声学硬边界条件使得 \( \vec{v} = 0 \),方程(S2)中的积分仅在管道的两端不为零。因此,有效面积 \( S' \) 可表示为:
\[
S' = S^t + 2S^A + S^\mu,
\]
其中上标分别表示对应的管道区域,如图S1(b)中的虚线区域所示。

---

为了进一步区分单元腔体中的两个腔体的参数,我们引入下标 \( j = 1, 2 \),并定义Kitaev链中波函数的关键参数 \( \xi = [\xi_1, \xi_2]^T \)。它们与Kitaev链中的波函数相关联,可以定义为:
\[
\xi_1 = \frac{p_1(\vec{r})}{v_1(\vec{r})} = \frac{-i \rho c_0^2}{\omega_0^2 - \omega^2} \oint_{\partial V_1} \psi^*(\vec{r}') \vec{v}_1(\vec{r}') \mathrm{d}S',
\]
\[
\xi_2 = \frac{p_2(\vec{r})}{v_2(\vec{r})} = \frac{-i \rho c_0^2}{\omega_0^2 - \omega^2} \oint_{\partial V_2} \psi^*(\vec{r}') \vec{v}_2(\vec{r}') \mathrm{d}S'. 
\]

将方程(S2)代入并结合体积和表面特性,可以分别表示为:
\[
\xi_1 = -\frac{c_0^2 d^2}{2(\omega_0^2 - \omega^2)} \left[ -\nu_1(0) \psi_1^A + \nu_1(0) \psi_1^A e^{-ika} + \nu_1^A(\psi_2^A e^{-ika}) \right],
\]
\[
\xi_2 = -\frac{c_0^2 d^2}{2(\omega_0^2 - \omega^2)} \left[ -\nu_2(0) \psi_2^A + \nu_2(0) \psi_2^A e^{-ika} + \nu_2^A(\psi_1^A e^{-ika}) \right]. 
\]

其中 \( V = w^2 h \) 是腔体的体积,\( \bar{\psi}_j^m \) 是与 \( j \)-th 腔连接的第 \( m \)-根管末端声速势的平均值。对于所有管道具有相同的横截面积和 \( r_c \) 值,由以下关系得出:
\[
\bar{\psi}_1 = \bar{\psi}_2 = -\bar{\psi}_1^t = -\bar{\psi}_2^t = \psi,
\]
因此可以得出 \( \bar{\psi}_j^m = |\psi| \)。同时需要注意,这种近似在管道尺寸远小于腔体尺寸的情况下是适用的。

根据这些结果,可以以 \( \xi \) 表示声压:
\[
p_j^t(0) = \xi_j \bar{\psi}_j, \quad
p_j^t(l_m^t) = \xi_j \bar{\psi}_j^t, \quad
p_j^t(f_j) = e^{ika} \xi_j \bar{\psi}_j^t,
\]
\[
p_1^A = e^{ika} \xi_2 \bar{\psi}_2^t, \quad
p_2^A = e^{ika} \xi_1 \bar{\psi}_1^t,
\]
\[
p_j^\mu = \xi_j \sqrt{\frac{2}{V}}. 
\]

为了映射严格的对应关系,我们现在重点分析管道的声学连接条件。假设结构内声波以平面波形式传播,则在一个单元腔中第 \( j \)-th 腔的声压和声速可以表示为:
\[
p_m^j(l) = A_m^j e^{i\omega l/c_0} + B_m^j e^{-i\omega l/c_0},
\]
\[
\rho_0 c_0 v_m^j(l) = A_m^j e^{i\omega l/c_0} - B_m^j e^{-i\omega l/c_0}, 
\]
其中 \( m \) 表示 \( t \) 和 \( \Delta \)。

通过将 \( l = 0 \) 和 \( l = l_m^t \) 代入方程(S4),可以很容易地得到以下关系:
\[
i\rho c_0
\begin{pmatrix}
v_j^t(0) \\
v_j^t(l_m^t)
\end{pmatrix}
=
\begin{pmatrix}
\cot(\omega l_m^t / c_0) & -\csc(\omega l_m^t / c_0) \\
-\csc(\omega l_m^t / c_0) & \cot(\omega l_m^t / c_0)
\end{pmatrix}
\begin{pmatrix}
p_j^t(0) \\
p_j^t(l_m^t)
\end{pmatrix}. 
\]

对于额外的管道(标记为 \( \mu \))且一端闭合,则满足以下关系:
\[
v_j^\mu = \frac{p_j^\mu}{Z_j^\mu},
\]
其中 \( Z_j^\mu \) 是连接腔体的端部阻抗。当闭合端为声学硬边界时:
\[
Z_j^\mu = -i\rho c_0 \cot(\omega l_j^\mu / c_0),
\]
而当闭合端为声学软边界时:
\[
Z_j^\mu = i\rho c_0 \tan(\omega l_j^\mu / c_0)。
\]

进一步地,将方程(S4)-(S7)代入方程(S3),波函数方程可以以矩阵形式表示为:
\[
\omega^2 \xi = (H_0 + H_a(k)) \xi, 
\]
其中:
\[
H_0 =
\begin{pmatrix}
\omega_0 + \epsilon_1 & 0 \\
0 & \omega_0 + \epsilon_2
\end{pmatrix},
\quad
H_a(k) =
\begin{pmatrix}
\mu_1 + 2t_1\cos(ka) & \Delta_1 e^{ika} + \Delta_2 e^{-ika} \\
\Delta_1 e^{-ika} + \Delta_2 e^{ika} & \mu_2 + 2t_2\cos(ka)
\end{pmatrix}. 
\]
其中
\[
\epsilon_1 = \frac{cd^2 |\psi|^2}{2} \left[ 2\cot\left(\frac{\omega_0 l_1^t}{c_0}\right) + \cot\left(\frac{\omega_0 l_1^A}{c_0}\right) + \cot\left(\frac{\omega_0 l_2^A}{c_0}\right) \right],
\]
\[
\epsilon_2 = \frac{cd^2 |\psi|^2}{2} \left[ 2\cot\left(\frac{\omega_0 l_2^t}{c_0}\right) + \cot\left(\frac{\omega_0 l_1^A}{c_0}\right) + \cot\left(\frac{\omega_0 l_2^A}{c_0}\right) \right],
\]
\[
t_1 = -\frac{cd^2 |\psi|^2}{2} \csc\left(\frac{\omega_0 l_1^t}{c_0}\right), \quad
t_2 = -\frac{cd^2 |\psi|^2}{2} \csc\left(\frac{\omega_0 l_2^t}{c_0}\right),
\]
\[
\Delta_1 = -\frac{cd^2 |\psi|^2}{2} \csc\left(\frac{\omega_0 l_1^A}{c_0}\right), \quad
\Delta_2 = -\frac{cd^2 |\psi|^2}{2} \csc\left(\frac{\omega_0 l_2^A}{c_0}\right),
\]
\[
\mu_1 = \frac{i\rho c_0^2 d^2}{V Z_1^\mu}, \quad
\mu_2 = \frac{i\rho c_0^2 d^2}{V Z_2^\mu}. 
\]
注意:一旦声学结构确定,所有这些参数都可以直接计算。需要特别指出的是,方程(S10)表明,所有由 \( P_z \) 模式描述的关键强度参数(\( t, \Delta, \mu \))是解耦的,这与由声学系统谐振基频描述的跃迁不同,因此可以独立设计。

为了构造严格的声学Kitaev链,需要满足以下条件:
\[
l_1^t = l_2^t + h, \quad t_1 = -t_2 = -t, \quad
l_1^A = l_2^A + h,\quad \Delta_1 = -\Delta_2 = -\Delta,
\]
并且 \( Z_1^\mu = -Z_2^\mu \),当 \( \mu_1 = -\mu_2 = -\mu \) 时,这些设置自然确保 \( \epsilon_1 = \epsilon_2 = \epsilon \)。因此,方程(S9)可以简化为:
\[
H_0 =
\begin{pmatrix}
\omega_0 + \epsilon & 0 \\
0 & \omega_0 + \epsilon
\end{pmatrix},
\quad
H_a(k) =
\begin{pmatrix}
-\mu - 2t\cos(ka) & -2i\Delta\sin(ka) \\
2i\Delta\sin(ka) & \mu + 2t\cos(ka)
\end{pmatrix}. 
\]