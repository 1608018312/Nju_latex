\chapter{声学边界的影响}
\section{拓扑绝缘体}

\section{零阶模式的声学哈密顿量}

在文章中,哈密顿量的推导较为详细,主要涉及以下几个步骤和公式:

文章中首先通过声学-电学类比来定义声学系统的 lumped 参数模型。在该模型中,腔体被表示为电容 \( C \),波导被表示为电感 \( L_v \) 和 \( L_w \)。具体公式如下:
\[
C = \frac{V}{\rho c^2}
\]
\[
L_v = \frac{\rho (l_v + 1.7r_v)}{\pi r_v^2}, \quad L_w = \frac{\rho (l_w + 1.7r_w)}{\pi r_w^2}
\]
其中,\( V \) 是腔体的体积,\( \rho \) 是空气的密度,\( c \) 是声速,\( r_v \) 和 \( r_w \) 分别是波导的半径,\( l_v \) 和 \( l_w \) 是它们的长度。


通过应用基尔霍夫电流定律,文章得到了描述声学波动的方程。对于周期性结构,腔体的声压在每个晶格点处满足如下方程:
\[
- (2w + 2v) u_1^0 + w u_2^0 + w u_3^0 + v u_2^6 + v u_3^1 = \omega^2 u_1^0
\]
\[
- (2w + 2v) u_2^0 + w u_3^0 + w u_1^0 + v u_3^2 + v u_1^3 = \omega^2 u_2^0
\]
\[
- (2w + 2v) u_3^0 + w u_1^0 + w u_2^0 + v u_1^4 + v u_2^5 = \omega^2 u_3^0
\]
其中,\( u_n^m \) 表示第 \( n \) 个腔体在第 \( m \) 个晶格上的声压,\( w = -\frac{1}{L_v C} \),\( v = -\frac{1}{L_w C} \) 是与腔体和波导的物理参数相关的系数,\( \omega \) 是角频率。


通过上述方程,可以将哈密顿量表示为矩阵形式。在周期性结构下,哈密顿量 \( H_0(k) \) 以矩阵的形式表示为:
\[
H_0(k) =
\begin{pmatrix}
-2w - 2v & w + v e^{i k \cdot (a_1 + a_2)} & w + v e^{i k \cdot a_1} \\
w + v e^{-i k \cdot (a_1 + a_2)} & -2w - 2v & w + v e^{-i k \cdot a_2} \\
w + v e^{-i k \cdot a_1} & w + v e^{i k \cdot a_2} & -2w - 2v
\end{pmatrix}
\]
这里,\( k \) 是布里渊区内的波矢,\( a_1 \) 和 \( a_2 \) 是晶格常数。


考虑到实际声学系统的边界条件,文章进一步推导了边界对哈密顿量的影响。对于有限结构,硬边界和软边界条件分别影响哈密顿量的对角项。

- **硬边界条件**:对于硬边界,腔体的对角项变为:
  \[
  - (2w + v)
  \]
  
- **软边界条件**:对于软边界,腔体的对角项变为:
  \[
  - (2w + 2v)
  \]


文章进一步通过计算系统的拓扑极化,来揭示哈密顿量与拓扑相之间的关系。极化 \( p \) 的计算通过下式给出:
\[
e^{-i \pi p} = \prod_{n \in \text{occ}} \theta_n(K) \theta_n(\Gamma)
\]
其中,\( \theta_n(k) = \langle u_n(k) | R_3 | u_n(k) \rangle \) 是通过三重对称操作 \( R_3 \) 计算得到的本征值,\( n \) 是占据的能带,\( K \) 和 \( \Gamma \) 是高对称点。


通过上述推导,文章揭示了软边界条件如何保留系统的拓扑特性,特别是角落态的存在,这些角落态与电子系统中的零能态相对应。

总结而言,哈密顿量的推导通过声学电学类比、基尔霍夫电流定律以及周期性结构的布洛赫波函数方法,详细描述了声学拓扑晶体绝缘体的物理特性,同时揭示了边界条件对拓扑态的关键作用。