\chapter{绪论}
\section{引言}
声学,作为物理学的一个重要分支,研究声波的产生、传播及与物质的相互作用,具有几千年的历史。从古代到现代,声学的发展经历了从经验性观察到系统化理论研究的演变。最早的声学研究主要来源于对自然现象的观察。古希腊毕达哥拉斯(Pythagoras)通过对弦的振动进行实验,发现了音高与弦长、张力和质量之间的关系,为声学提供了初步的实验基础。亚里士多德(Aristotle)提出声音是通过介质传播的这一理论,为后来的声波传播研究提供了启示。中国古代也对音律和声波传播有过观察,如《礼记》中的“宫商角徵羽”五音系统,反映了中国古人对声音与音高的初步认知。进入17世纪,科学革命推动了声学研究进入了一个新的阶段。伽利略(Galileo Galilei)通过对弦乐器的研究揭示了振动与声音之间的关系。罗伯特·胡克(Robert Hooke)在其《弹性理论》中深入探讨了弹性体的振动特性,并提出了弹性波的传播理论,这对声波传播的理解产生了重要影响。进入19世纪,雷利爵士(Lord Rayleigh)对声学做出了具有里程碑意义的贡献。其著作《声音的理论》(The Theory of Sound, 1877)详细阐述了声波的传播规律,并提出了声波在不同介质中传播的速度、反射、折射等现象的数学描述。雷利的工作奠定了现代声学的数学基础,并对波动理论的普及产生了重要影响。20世纪,声学进入了更加多样化和精细化的研究阶段。随着实验技术的不断发展,声学研究逐渐拓展到多个方向,包括超声学、建筑声学、声学成像和音频声学等领域。声学理论得到了前所未有的发展,尤其是在声学超材料(Acoustic Metamaterials)的研究中。

超材料是指通过人工设计的结构,其在某些条件下表现出自然材料没有的异常物理性质。声学超材料的研究起源于能带理论的提出和光子晶体的成功发现。能带理论最早由Felix Bloch提出\cite{a1},用以描述电子在周期性晶体中的运动,揭示了周期性结构如何形成能带和禁带。这一理论的成功应用,激发了人们对周期性结构在其他波动现象中的潜力的探索。1987年,John Yablonovitch提出了光子晶体的概念\cite{a2},通过周期性结构控制光波的传播,并形成光学带隙。光子晶体的出现展示了通过人工设计材料微观结构,可以调节波动传播的方向、速度及频率,为超材料的研究开辟了新的方向\cite{a3,a4,a5}。受到光子晶体启发,声学领域的研究者开始探索如何借鉴类似的理念来调控声波的传播。1993年,Kushwaha等人第一次明确提出了声子晶体(Phononic Crystals)的概念\cite{b1},不久以后Martinez等人通过实验验证了声子晶体的禁带特性\cite{b2,b3}。随后,Liu等人提出了局域共振声子晶体的概念\cite{b4},展示了局域共振声子晶体在低频声波的有效控制能力,为声波调控提供了新的思路。声学超材料因其独特的声波控制能力,广泛应用于多个领域,如声学负参数材料\cite{c11,c12,c13,c14,c15,c16,c17,c18},反常声传输\cite{c21,c22,c23,c24,c25,c26,c27,c28,c29},声学超透镜\cite{c31,c32,c33,c34,c35,c36,c37,c38,c39},声隐身\cite{c41,c42,c43,c44,c45,c46},声学轨道角动量\cite{c51,c52,c53,c54,c55,c56,c57,c58},声学非互易\cite{c61,c62,c63,c64,c65,c66},声学黑洞\cite{c71,c72,c73,c74,c75,c76},通风降噪\cite{c81,c82,c83,c84,c85}等等。