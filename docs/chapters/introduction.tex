\chapter{绪论}

\section{引言}
声学,作为物理学的一个重要分支,研究声波的产生、传播及与物质的相互作用,具有几千年的历史。从古代到现代,声学的发展经历了从经验性观察到系统化理论研究的演变。最早的声学研究主要来源于对自然现象的观察。古希腊毕达哥拉斯(Pythagoras)通过对弦的振动进行实验,发现了音高与弦长、张力和质量之间的关系,为声学提供了初步的实验基础。亚里士多德(Aristotle)提出声音是通过介质传播的这一理论,为后来的声波传播研究提供了启示。中国古代也对音律和声波传播有过观察,如《礼记》中的“宫商角徵羽”五音系统,反映了中国古人对声音与音高的初步认知。进入17世纪,科学革命推动了声学研究进入了一个新的阶段。伽利略(Galileo Galilei)通过对弦乐器的研究揭示了振动与声音之间的关系。罗伯特·胡克(Robert Hooke)在其《弹性理论》中深入探讨了弹性体的振动特性,并提出了弹性波的传播理论,这对声波传播的理解产生了重要影响。进入19世纪,雷利爵士(Lord Rayleigh)对声学做出了具有里程碑意义的贡献。其著作《声音的理论》(The Theory of Sound, 1877)详细阐述了声波的传播规律,并提出了声波在不同介质中传播的速度、反射、折射等现象的数学描述。雷利的工作奠定了现代声学的数学基础,并对波动理论的普及产生了重要影响。20世纪,声学进入了更加多样化和精细化的研究阶段。随着实验技术的不断发展,声学研究逐渐拓展到多个方向,包括超声学、建筑声学、声学成像和音频声学等领域。声学理论得到了前所未有的发展,尤其是在声学超材料(Acoustic Metamaterials)的研究中。

超材料是指通过人工设计的结构,其在某些条件下表现出自然材料没有的异常物理性质。声学超材料的研究起源于能带理论的提出和光子晶体的成功发现。能带理论最早由Felix Bloch提出\cite{a1},用以描述电子在周期性晶体中的运动,揭示了周期性结构如何形成能带和禁带。这一理论的成功应用,激发了人们对周期性结构在其他波动现象中的潜力的探索。1987年,John Yablonovitch提出了光子晶体的概念\cite{a2},通过周期性结构控制光波的传播,并形成光学带隙。光子晶体的出现展示了通过人工设计材料微观结构,可以调节波动传播的方向、速度及频率,为超材料的研究开辟了新的方向\cite{a3,a4,a5}。受到光子晶体启发,声学领域的研究者开始探索如何借鉴类似的理念来调控声波的传播。1993年,Kushwaha等人第一次明确提出了声子晶体(Phononic Crystals)的概念\cite{b1},不久以后Martinez等人通过实验验证了声子晶体的禁带特性\cite{b2,b3}。随后,Liu等人提出了局域共振声子晶体的概念\cite{b4},展示了局域共振声子晶体在低频声波的有效控制能力,为声波调控提供了新的思路。声学超材料因其独特的声波控制能力,广泛应用于多个领域,如声学负参数材料\cite{c11,c12,c13,c14,c15,c16,c17,c18},反常声传输\cite{c21,c22,c23,c24,c25,c26,c27,c28,c29},声学超透镜\cite{c31,c32,c33,c34,c35,c36,c37,c38,c39},声隐身\cite{c41,c42,c43,c44,c45,c46},声学轨道角动量\cite{c51,c52,c53,c54,c55,c56,c57,c58},声学非互易\cite{c61,c62,c63,c64,c65,c66},声学黑洞\cite{c71,c72,c73,c74,c75,c76},通风降噪\cite{c81,c82,c83,c84,c85}等等。

近年来,拓扑物理学的概念被引入到声学研究中,结合声学超材料的周期性结构,为声波在复杂环境中的可靠操控提供了新的思路:声学拓扑绝缘体的边界态可以在拓扑保护下实现稳定传播,甚至在存在缺陷或障碍时,依然能够保持声波传输的完整性。一方面,声人工结构以其高度的设计灵活性,宏观可观测性和实验研究的便利性,被视作观测量子效应的重要平台。另一方面,声学拓扑绝缘体独特的结构和性能不仅拓展了声学研究的边界,还为新型声学功能材料的设计和应用奠定了理论基础。


\section{拓扑绝缘体概述}
\subsection{拓扑学与物理}
拓扑学是数学中的一个重要分支,研究几何空间或物体在连续变形下保持不变的性质\cite{d1}。这种变形允许拉伸、压缩和扭曲,但不允许切割或粘合,因此拓扑学更注重对象的整体性质,而非几何细节。拓扑学的核心概念是拓扑不变量,它是一种描述拓扑空间本质特征的量,在连续变形下保持不变。

以图\ref{fig_1_1}三叶结和简单环为例,这两个形状分别代表了具有不同拓扑不变量的结构。三叶结是一种复杂的拓扑结构,其"结点"特性无法通过连续变形去除,而简单环则没有这样的"结点"。从拓扑学的视角来看,三叶结和简单环之间无法通过连续变形转化,只有切断重新连接才能实现这一转换。这一不可变性正是由拓扑不变量决定的。

拓扑不变量的定义依赖于数学工具,例如同伦类、基本群和同调群等。这些工具能够描述空间的连通性、孔洞数量和高维特征等。例如,在一个二维平面上,具有不同孔洞数量的形状可以通过拓扑不变量区分:一个圆没有孔洞,而一个甜甜圈有一个孔洞,因此它们的拓扑不变量不同。

拓扑不变量的意义不仅限于几何空间的分类,它在拓扑学中提供了统一的语言,用于研究不同系统中保持稳定的性质。通过分析拓扑不变量,可以揭示复杂系统中的内在联系。例如,三叶结的"结点"可以用拓扑学中的链环数来量化,而简单环的拓扑结构则显得更加直观单一。正是这种对拓扑特性的精准描述,使得拓扑学在理论研究中占据了重要地位。

总之,拓扑学以拓扑不变量为核心,通过研究对象在连续变形下的不变性质,为理解几何空间的本质提供了深刻的洞见。这一领域不仅具有高度的数学抽象性,也为探索空间、形状及其内部结构的基本规律奠定了理论基础。值得注意的是,拓扑学不仅限于纯数学领域,其思想也被广泛引入到物理学中,用于解释自然界中许多深层次的规律和现象,从而为研究拓扑物理打开了大门。
\begin{figure}[h!]
    \centering
    \includegraphics[width=0.8\textwidth]{images/fig1-1.png} 
    \caption{拓扑相变的直观说明。三叶结(左)和简单环(右)分别代表不同的绝缘材料:三叶结是拓扑绝缘体而简单环是普通绝缘体。由于无法通过连续变形将一种形状转变为另一种,因此在两者之间的过渡必须有一个表面,这个表面可以被视为“被切断的结”。\cite{d1}}
    \label{fig_1_1}
\end{figure}


\subsection{量子霍尔效应}
量子霍尔效应(Quantum Hall Effect)是凝聚态物理中最重要的拓扑现象之一,由Klaus von Klitzing于1980年首次发现\cite{d2}(如图\ref{fig_1_2}所示)。这一效应是在二维电子气系统中通过实验观察到的:当强磁场垂直作用于二维导体表面,并且系统处于低温条件下时,霍尔电导呈现完全量子化的特性。其值由公式
\[
\sigma_{xy} = \frac{e^2}{h} n
\]
确定,其中 \( e \) 是电子电荷,\( h \) 是普朗克常数,\( n \) 是一个整数,称为拓扑不变量。霍尔电导的量子化表明其对材料的微观细节或杂质分布不敏感,而由系统的拓扑性质决定。

这一现象的理论基础是电子在磁场中运动时形成的朗道能级,其能量为:
\[
E_n = \hbar \omega_c \left( n + \frac{1}{2} \right),
\]
其中 \( \hbar \) 是约化普朗克常数,\( \omega_c = \frac{eB}{m} \) 为回旋频率,\( n \) 是朗道能级的量子数。这些离散能级限制了电子在二维平面中的运动,使得系统的输运性质呈现量子化行为。

1982年,Thouless、Kohmoto、Nightingale和Nijs(TKNN)将整数量子霍尔效应与拓扑不变量联系起来\cite{d3}。他们发现,霍尔电导的量子化可以通过布里渊区中电子态的几何特性来解释,并提出了以下公式:
\[
\sigma_{xy} = \frac{e^2}{h} \frac{1}{2\pi} \int_{BZ} \Omega(\mathbf{k}) \, d^2k
\]
其中 \( \Omega(\mathbf{k}) \) 是贝里曲率,表示为贝里联络的旋度。贝里联络的定义为:
\[
\mathbf{A}(\mathbf{k}) = i \langle u(\mathbf{k}) | \nabla_{\mathbf{k}} | u(\mathbf{k}) \rangle
\]
这里 \( |u(\mathbf{k})\rangle \) 是布里渊区中的周期波函数。TKNN工作首次明确了霍尔电导的拓扑起源,并将陈数 \( C \) 引入这一框架:
\[
C = \frac{1}{2\pi} \int_{BZ} \Omega(\mathbf{k}) \, d^2k
\]
这个整数拓扑不变量直接决定了量子霍尔效应中的量子化电导。

1984年,迈克尔·贝里(Michael Berry)提出了贝里相(Berry Phase)的理论\cite{d4},这是一个与路径相关的几何相位,系统的波函数在参数空间中演化时会积累这一相位。贝里相为理解电子态的几何和拓扑特性提供了基础,贝里联络和贝里曲率的引入进一步深化了对拓扑效应的描述。

1988年,Haldane在以上工作的基础上提出了陈绝缘体(Chern Insulator)的理论\cite{d5}。他设计了一个二维模型,其中通过引入周期性磁通破坏时间反演对称性,使得系统可以在无外加磁场的条件下表现出量子化霍尔效应。陈绝缘体的电导仍然由陈数 \( C \) 确定,但其拓扑特性完全由晶格的能带结构决定,而不依赖于外部磁场。这一模型不仅验证了量子霍尔效应的拓扑本质,也为非磁性拓扑材料的研究奠定了理论基础。

量子霍尔效应及其拓展模型如陈绝缘体,揭示了凝聚态物理中拓扑学与电子输运之间的深刻联系。这些工作从实验发现到理论突破,构建了一个将几何、拓扑与物理现象结合的完整框架,为后续拓扑物质的研究铺平了道路。

\begin{figure}[h!]
    \centering
    \includegraphics[width=0.8\textwidth]{images/fig1-2.png} 
    \caption{量子霍尔效应的说明:(a) 原子绝缘体状态;(b) 简单的绝缘体能带结构;(c) 表示绝缘体的球面(亏格 \( g=0 \))。(d) 量子霍尔态,表现为电子在磁场 \( B \) 中的回旋运动;(e) 朗道能级,显示为离散的能带结构,对应量子化的电子状态;(f) 表示量子霍尔态的圆环形面(亏格 \( g=1 \))。\cite{r11}}
    \label{fig_1_2}
\end{figure}


\subsection{量子自旋霍尔效应}

\begin{figure}[h!]
    \centering
    \includegraphics[width=0.8\textwidth]{images/fig1-3.eps} 
    \caption{量子霍尔效应和量子自旋霍尔效应对比:(a) 量子霍尔效应的边界态和能带结构;(b) 量子自旋霍尔效应的边界态和能带结构。\cite{r11}}
    \label{fig_1_3}
\end{figure}

在量子霍尔效应和陈绝缘体理论的基础上,量子自旋霍尔效应(Quantum Spin Hall Effect, QSHE)作为一种不依赖外加磁场的拓扑量子态逐渐成为研究热点。与传统量子霍尔效应不同,量子自旋霍尔效应通过自旋轨道耦合实现电子自旋与运动方向的联系,从而在系统边缘形成受拓扑保护的无散射传输通道。该效应揭示了非磁性体系中拓扑物态的新特征,并为进一步研究拓扑绝缘体奠定了理论与实验基础。

量子自旋霍尔效应的初步理论框架由Murakami、Nagaosa和Zhang于2004年提出\cite{e1}。他们通过分析具有强自旋轨道耦合的二维电子系统,指出这种耦合会在布里渊区中产生Berry曲率,从而驱动电子表现出类似量子霍尔效应的行为,但其霍尔电导由自旋分量贡献。这一理论表明,自旋向上和自旋向下的电子可以在系统边缘分别沿相反方向传播,而无需外加磁场。

2005年,Kane和Mele进一步扩展了这一理论\cite{e2,e3},提出了自旋霍尔绝缘体的Z₂拓扑不变量,用于描述二维体系中量子自旋霍尔效应的稳定性和边缘态的拓扑保护。他们通过计算Berry曲率和Z₂拓扑数,证明了具有强自旋轨道耦合的体系可以支持量子自旋霍尔态。这种状态具有量子化的边界导电性,而系统内部则保持绝缘。

2006年,Bernevig、Hughes和Zhang预测了一个实际的量子自旋霍尔效应体系——HgTe/CdTe量子阱结构\cite{e4}。当量子阱的厚度超过临界值时,体系会发生拓扑相变,进入量子自旋霍尔态。这一理论预测在König等人的实验中得到了验证\cite{e5}。他们通过低温输运实验发现,这种体系的边缘态表现出无散射传输和量子化导电的特性,即便在没有外加磁场的条件下亦是如此。

量子自旋霍尔效应的拓扑特性由Z₂拓扑不变量决定,反映了系统布里渊区中能带结构的几何特征。与传统量子霍尔效应的整数拓扑不变量(如陈数)不同,Z₂不变量强调体系的时间反演对称性,使其特别适用于非磁性系统的描述。量子自旋霍尔效应不仅拓展了拓扑物理的研究范围,也为拓扑绝缘体、拓扑超导体等新型物态的研究开辟了道路。


\subsection{拓扑晶体绝缘体和高阶拓扑态}
在量子霍尔效应和量子自旋霍尔效应中,时间反演对称性被认为是实现特定拓扑态的关键因素。然而,自然界中晶体材料的空间对称性(如镜面对称性、旋转对称性等)广泛存在。那么受对称性保护的拓扑态能否出现在这些体系中呢?这一问题促使研究者将视角从时间反演对称性拓展到更一般的对称性保护拓扑态,从而引出了拓扑晶体绝缘体(Topological Crystalline Insulator, TCI)的概念。

作为拓扑物态研究中的一个重要方向,拓扑晶体绝缘体的概念由Liang Fu在2011年首次提出\cite{f1}。不同于传统的时间反演对称性保护的拓扑绝缘体,TCI的拓扑性质由晶体的镜面对称性和旋转对称性等空间对称性保护。这种拓扑保护使得TCI在晶体的高对称面上能够存在受保护的表面态。例如,Hsieh等人通过第一性原理计算,发现SnTe材料中存在镜面对称性保护的表面态,这些表面态分布在高对称面(如{001}和{111}面)上\cite{f2}。这些表面态的拓扑特性与体系的晶体对称性紧密相关,若镜面对称性被破坏,这些表面态可能退化或消失,这凸显了晶体对称性在TCI中的重要作用。

TCI的理论框架在随后的研究中得到了扩展,特别是在高阶拓扑绝缘体(Higher-Order Topological Insulator, HOTI)的研究中展现了更加丰富的物理现象\cite{f3}。Benalcazar等人提出了基于多极矩描述的HOTI理论,指出高阶拓扑特性不仅表现在二维的表面,还可以出现在一维边界甚至零维角落。例如,在具有Cn旋转对称性的晶体中,其角落电荷可以通过以下公式描述:
\[
Q_{\text{corner}} = \frac{e}{n},
\]
其中 \(n\) 是旋转对称性的阶数,\(Q_{\text{corner}}\) 是量子化的角落电荷。这一理论进一步揭示了角落电荷的分数化性质,其量子化特性是由系统的对称性和能带结构共同决定的。

Schindler等人通过研究铋(Bi)材料,进一步揭示了三维HOTI的物理特性\cite{f4}。他们发现,铋在边界上的二维表面表现为绝缘体,而在角落处表现为受拓扑保护的局域化态。这种局域化态来源于填充异常(filling anomaly),即能带中电子数与晶体对称性要求的电子数之间的不匹配。填充异常的数学本质可以通过Wannier中心与晶格位置的不匹配来解释。Wannier中心的分布决定了体系的高阶拓扑特性,并与角落电荷的分数化直接相关。

TCI的拓扑特性还可以通过布里渊区中的拓扑不变量来量化。例如,体系的晶格极化可以用以下公式定义:
\[
P^{(n)} = p_1 a_1 + p_2 a_2,
\]
其中 \(P^{(n)}\) 表示晶体的极化,\(p_1\) 和 \(p_2\) 是布里渊区中不可缩环上的Berry相位,\(a_1\) 和 \(a_2\) 是晶格的基矢量。通过这一公式,可以预测高阶TCI中的角落态和边界态的分布。

此外,Slager等人通过分析空间群的对称性,系统性地分类了不同的拓扑绝缘体\cite{f5}。他们提出了一种基于空间群对称性的拓扑绝缘体分类方法,将TCI与晶体的对称性紧密联系在一起。这一分类方法不仅适用于传统的TCI,还能够描述高阶拓扑绝缘体的拓扑特性,从而为拓扑物态的理论研究提供了完整的框架。

TCI的研究为拓扑物态的理解提供了新的视角,不仅揭示了晶体对称性如何影响拓扑特性,还为设计具有特定拓扑性质的材料提供了理论依据。这些研究在理论和实验上均取得了显著进展,并为未来探索更复杂的拓扑物态奠定了基础。 


\section{经典波系统中寻找类拓扑效应}

由于波动方程的相似性,光和声作为经典波动形式,近年来通过引入拓扑物理学的核心概念,展现出许多奇异的物质拓扑相位现象,开辟了新的研究方向。经典波系统中的拓扑绝缘体是通过构造具有特定对称性和拓扑特征的波导或晶体结构来实现的。这些系统的拓扑性质使得它们在边界或表面上支持无反射的边界态,展现出独特的物理现象。

Haldane 和 Raghu 在 2008 年首次提出了光子晶体中的拓扑绝缘体模型\cite{g1},展示了如何通过引入非平庸的拓扑相位来实现光子的单向传播。这一模型的核心在于,通过引入时间反演对称性破缺的机制,例如磁性光学材料,可以在光子晶体的能带结构中打开拓扑能隙,使得光子的传播方向被拓扑保护。具体而言,光子晶体的周期性结构使得光子的传播特性受到调制,从而形成带隙结构。这种设计允许光子在带隙中只能以单向的形式传播,同时避免了散射和反射的影响。随后,2009 年 Wang 等人成功在实验中验证了这一理论\cite{g2}。他们利用二维磁性光子晶体,在实验中观察到了稳健的单向边界态,这种边界态不仅可以有效防止背向散射,而且在一定程度上对缺陷和杂质不敏感,为光子学中的拓扑物理研究提供了实验支持。

为了在光学领域实现拓扑模型并探索其应用,其中一个主要问题是在光学领域缺乏大的磁光响应。解决这一难题的方法之一是将光子的内部自由度视为类自旋,并寻找类比于量子自旋霍尔系统的模型,即整体时间反演对称性未被破坏,但每个赝自旋感受到人工磁场。2011 年,Hafezi 等人通过在光学系统中引入类似于电子系统中的自旋自由度,提出了一种模拟量子自旋霍尔效应的光学类比模型\cite{g3}。具体而言,他们设计了一种基于环形谐振腔阵列的光学结构,这种结构通过人为构造的光学路径,产生了类似于电子自旋与轨道耦合的现象,从而在光子系统中实现了类自旋霍尔效应。他们的研究表明,自旋自由度的引入可以在光子系统中产生稳健的拓扑边界态,这些边界态不仅能够有效地抵抗缺陷和无序的影响,还能够实现无损耗的单向传播。第二种方法建立在Floquet拓扑绝缘体的理论基础上\cite{g4,g5},通过时间周期性的外部调制,使系统形成一个有效的时间无关哈密顿量,从而实现时间反演对称性的动态破缺。这一理论从凝聚态物理中延伸至光学系统,并在2013年由Hafezi 等和Rechtsman 等分别通过独立实验成功验证\cite{g6,g7}。其中,Hafezi 等人利用时间周期性调制光学结构实现了稳定的拓扑边界态,而Rechtsman 等人通过设计二维波导阵列系统,观察到了与Floquet拓扑绝缘体相关的单向边界态,这些实验结果为理论提供了有力支持。此外,另一种基于时间依赖性调制的方式,即“拓扑泵”理论\cite{g8},通过对系统施加时间演化的调制,可以实现量子态的拓扑保护转移。这一方法最终在2012年由Kraus 等通过实验验证,他们成功实现了基于光子学的拓扑泵现象,进一步丰富了光学拓扑物理的实验体系\cite{g9}。这些理论与实验的结合,不仅推动了光学系统中拓扑效应的深入研究,也为未来开发新型光学器件提供了理论指导和实践支持\cite{h1,h2,h3,h4,h5,h6,h7,h8,h9,h10,h11,h12,h13,h14,h15,h16}。

\section{声学拓扑绝缘体的研究现状}