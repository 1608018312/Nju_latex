%%%%%%%%%%%%%%%%%%%%%%%%%%%%%%%%%%%%%%%%%%%%%%%%%%%%%%%%%%%%%%%%%%%%%%
% njuthesis 示例模板 v1.4.2 2024-11-08
% https://github.com/nju-lug/NJUThesis
%
% 贡献者
% Yu XIONG @atxy-blip   Yichen ZHAO @FengChendian
% Song GAO @myandeg     Chang MA @glatavento
% Yilun SUN @HermitSun  Yinfeng LIN @linyinfeng
% Yukai Chou @Muzimuzhi
%
% 许可证
% LaTeX Project Public License(版本 1.3c 或更高)
%%%%%%%%%%%%%%%%%%%%%%%%%%%%%%%%%%%%%%%%%%%%%%%%%%%%%%%%%%%%%%%%%%%%%%

%---------------------------------------------------------------------
% 一些提升使用体验的小技巧:
%   1. 请务必使用 UTF-8 编码编写和保存本文档
%   2. 请务必使用 XeLaTeX 或 LuaLaTeX 引擎进行编译
%   3. 不保证接口稳定,写作前一定要留意版本号
%   4. 以百分号(%)开头的内容为注释,可以随意删除
%---------------------------------------------------------------------

%---------------------------------------------------------------------
% 请先阅读使用手册:
% http://mirrors.ctan.org/macros/unicodetex/latex/njuthesis/njuthesis.pdf
%---------------------------------------------------------------------

\documentclass[
    % 模板选项(注意右端逗号):
    %
    % type = bachelor|master|doctor|postdoc, % 文档类型,默认为本科生
    % degree = academic|professional,        % 学位类型,默认为学术型
    %
    % nl-cover,   % 是否需要国家图书馆封面,默认关闭
    % decl-page,  % 是否需要诚信承诺书或原创性声明,默认关闭
    %
    %   页面模式,详见手册说明
    % draft,                  % 开启草稿模式
    anonymous,              % 开启盲审模式
    % minimal,                % 开启最小化模式
    %
    %   单双面模式,默认为适合印刷的双面模式
    % oneside,                % 单面模式,无空白页
    % twoside,                % 双面模式,每一章从奇数页开始
    %
    %   字体设置,不填写则自动调用系统预装字体,详见手册
    % fontset = win|mac|macoffice|fandol|none,
    type = doctor,
  ]{njuthesis}

% 模板选项设置,包括个人信息、外观样式等
% 较为冗长且一般不需要反复修改,我们把它放在单独的文件里
\input{njuthesis-setup.def}

% 自行载入所需宏包
\usepackage{bm}
\usepackage{amsmath}
\usepackage{textcomp}
% \usepackage{subcaption} % 嵌套小幅图像,比 subfig 和 subfigure 更新更好
% \usepackage{siunitx} % 标准单位符号
% \usepackage{physics} % 物理百宝箱
% \usepackage[version=4]{mhchem} % 绘制分子式
% \usepackage{listings} % 展示代码
% \usepackage{algorithm,algorithmic} % 展示算法伪代码

% 在导言区随意定制所需命令
% \DeclareMathOperator{\spn}{span}
% \NewDocumentCommand\mathbi{m}{\textbf{\em #1}}

% 开始编写论文
\begin{document}

%---------------------------------------------------------------------
%	封面、摘要、前言和目录
%---------------------------------------------------------------------

% 生成封面页
\maketitle

% 模板默认使用 \flushbottom,即底部平齐
% 效果更好,但可能出现 underfull \vbox 信息
% 以下命令用于抑制这些信息
\raggedbottom

\begin{abstract}
  声学作为经典波动系统的重要分支,其与凝聚态物理中拓扑物态的交叉研究为新型声学功能材料的设计与应用开辟了全新路径。本文聚焦于声学拓扑绝缘体的理论建模、拓扑态调控及功能应用,结合电声类比、格林函数分析、紧束缚模型等方法对声学拓扑结构进行理论分析,系统探究了多种声学拓扑系统的构建机制,并在实验上验证了拓扑边界态的鲁棒传播、彩虹俘获效应及类量子拓扑态的经典模拟。研究不仅深化了对经典波系统中拓扑现象的理解,还为声学器件的设计与应用提供了创新思路。

  在第一章中,我们从声学与拓扑物理的交叉背景出发,梳理了声学超材料与拓扑绝缘体的研究进展。声子晶体与局域共振理论的发展为声波调控奠定了基础,而拓扑物理的引入则赋予声学系统独特的鲁棒边界态。通过对比电子系统与声学系统的相似性,我们揭示了声学拓扑绝缘体的核心特征:受拓扑保护的边界态对缺陷与无序的免疫性。在此基础上,提出了本文的研究框架,即通过声学腔管结构的等效建模,探索高阶拓扑态、彩虹捕获及量子拓扑效应的经典实现。
  
  在第二章中,我们着重于建立了声学拓扑晶体绝缘体的理论模型的涉及方法。基于电声类比方法,将亚波长腔管结构等效为集总参数电路,推导了声容、声质量与声阻的数学表达,构建了声压-体积速度的等效电路模型。进一步,针对多腔耦合系统,利用刚性腔的简正模式展开与格林函数积分方法,建立了弱耦合条件下的声场分布理论。该模型为后续章节的拓扑态研究提供了普适性分析工具,揭示了声学参数(如腔体尺寸、导管阻抗)与紧束缚模型中跳跃项、在位势的严格对应关系。
  
  在第三章中,我们以Kagome晶格为研究对象,实现了高阶声学拓扑绝缘体的设计与验证。通过声-电类比严格推导了二维Kagome晶格的声学哈密顿量,阐明了其拓扑相变机制:在无限周期结构中,能带反转诱导非平庸拓扑;在有限结构中,边界条件(硬/软边界)直接影响边缘晶格在位势,仅软边界可支持受保护的高阶角态。有限元模拟表明,Kagome声学晶格在三角排列下可稳定激发角态,且其频率特性与理论预测一致。研究突破了传统拓扑绝缘体对“真空包裹”条件的依赖,为任意频率拓扑态的设计提供了通用方法。
  
  在第四章中,我们提出基于$C_4$对称的声学拓扑晶体绝缘体,实现了拓扑边界态群速度的主动调控与声波彩虹俘获。通过构建二维SSH模型,揭示了表面阻抗对紧束缚模型在位势的调控作用:调节边界声阻可独立改变拓扑态的群速度,而无需修改体结构。基于此,设计了一种亚波长拓扑彩虹俘获器原型,通过渐变的表面阻抗使不同频率声波局域于边界特定位置。实验与仿真结果表明,该结构在500-1500 Hz范围内实现了宽带声能分离,且拓扑保护特性显著提升了俘获效率与鲁棒性。这一成果为声学传感、能量收集及信息处理提供了新范式。
  
  在第五章中,我们首次在经典波系统中实现了Kitaev链的声学模拟,并观测到类马约拉纳零模。通过设计一维共振腔链,引入交替调节的“化学势”(腔体共振频率)与“超导配对势”(腔间耦合相位),构造了严格对应于无自旋p波超导链的声学哈密顿量。数值与实验结果表明,拓扑非平庸相下,链两端出现局域化零能模,其波函数分布满足马约拉纳费米子的自共轭特性。进一步,通过“声学键盘”结构动态调控耦合参数,实现了零模的创建、传输与融合,验证了其准粒子行为。该工作为宏观尺度探索非阿贝尔统计特性提供了经典平台。
  
  第六章则总结了全文研究内容,并展望了未来方向。理论方面,需进一步探索三维声学拓扑态及强耦合系统的非线性效应;应用方面,拓扑彩虹俘获器在噪声控制与声通信中潜力显著,而Kitaev链的声学实现为拓扑量子计算的经典模拟奠定了基础。此外,如何将拓扑声学与智能材料、可编程超表面结合,实现动态可重构器件,是下一阶段的重要课题。
\end{abstract}

\begin{abstract*}
  As an important branch of the classical wave system, the interdisciplinary research between acoustics and topological states in condensed matter physics has opened up new avenues for the design and application of novel acoustic functional materials. This paper focuses on the theoretical modeling, topological state regulation, and functional applications of acoustic topological insulators. By combining methods such as electro-acoustic analogy, Green's function analysis, and tight-binding model, a theoretical analysis of acoustic topological structures is carried out. The construction mechanisms of various acoustic topological systems are systematically explored, and the robust propagation of topological boundary states, the rainbow trapping effect, and the classical simulation of quantum topological states are experimentally verified. The research not only deepens the understanding of topological phenomena in classical wave systems but also provides innovative ideas for the design and application of acoustic devices.

  In Chapter 1, starting from the interdisciplinary background of acoustics and topological physics, we review the research progress of acoustic metamaterials and topological insulators. The development of phonon crystals and local resonance theories has laid the foundation for sound wave regulation, and the introduction of topological physics endows acoustic systems with unique robust boundary states. By comparing the similarities between electronic systems and acoustic systems, the core characteristics of acoustic topological insulators are revealed: the immunity of topologically protected boundary states to defects and disorders. On this basis, the research framework of this paper is proposed, that is, through the equivalent modeling of acoustic cavity-tube structures, to explore the classical realization of higher-order topological states, rainbow trapping, and quantum topological effects.
  
  In Chapter 2, we focus on establishing the methods involved in the theoretical model of acoustic topological crystal insulators. Based on the electro-acoustic analogy method, the sub-wavelength cavity-tube structure is equivalent to a lumped parameter circuit. The mathematical expressions of acoustic capacitance, acoustic mass, and acoustic resistance are derived, and an equivalent circuit model of sound pressure - volume velocity is constructed. Furthermore, for the multi-cavity coupling system, using the normal mode expansion of the rigid cavity and the Green's function integration method, a theoretical model of the sound field distribution under weak coupling conditions is established. This model provides a universal analytical tool for the topological state research in subsequent chapters and reveals the strict correspondence between acoustic parameters (such as cavity size, duct impedance) and the hopping terms and on-site potentials in the tight-binding model.
  
  In Chapter 3, taking the Kagome lattice as the research object, we achieve the design and verification of higher-order acoustic topological insulators. Through the acoustic - electrical analogy, the acoustic Hamiltonian of the two-dimensional Kagome lattice is strictly derived, and its topological phase transition mechanism is elucidated: in the infinite periodic structure, the band inversion induces a non-trivial topology; in the finite structure, the boundary conditions (hard/soft boundaries) directly affect the on-site potential of the edge lattice, and only the soft boundary can support the protected higher-order corner states. Finite element simulations show that the Kagome acoustic lattice can stably excite corner states in a triangular arrangement, and its frequency characteristics are consistent with the theoretical predictions. The research breaks through the dependence of traditional topological insulators on the "vacuum encapsulation" condition and provides a general method for the design of topological states at any frequency.
  
  In Chapter 4, we propose an acoustic topological crystal insulator based on $C_4$ symmetry and achieve the active regulation of the group velocity of topological boundary states and the rainbow trapping of sound waves. By constructing a two-dimensional SSH model, the regulation effect of surface impedance on the on-site potential of the tight-binding model is revealed: adjusting the boundary acoustic resistance can independently change the group velocity of the topological state without modifying the bulk structure. Based on this, a prototype of a sub-wavelength topological rainbow trap is designed, and different frequency sound waves are localized at specific positions on the boundary through the gradually changing surface impedance. Experimental and simulation results show that this structure achieves broadband acoustic energy separation in the range of 500-1500 Hz, and the topological protection characteristics significantly improve the trapping efficiency and robustness. This achievement provides a new paradigm for acoustic sensing, energy harvesting, and information processing.
  
  In Chapter 5, we realize the acoustic simulation of the Kitaev chain in the classical wave system for the first time and observe the Majorana-like zero mode. By designing a one-dimensional resonant cavity chain and introducing alternately adjusted "chemical potential" (cavity resonant frequency) and "superconducting pairing potential" (coupling phase between cavities), an acoustic Hamiltonian strictly corresponding to the spinless p-wave superconducting chain is constructed. Numerical and experimental results show that in the topologically non-trivial phase, localized zero-energy modes appear at both ends of the chain, and their wave function distributions satisfy the self-conjugate characteristics of Majorana fermions. Furthermore, through the "acoustic keyboard" structure to dynamically regulate the coupling parameters, the creation, transmission, and fusion of zero modes are realized, verifying their quasiparticle behavior. This work provides a classical platform for exploring non-Abelian statistics at the macroscopic scale.
  
  Chapter 6 summarizes the research content of the full text and looks forward to future directions. Theoretically, it is necessary to further explore three-dimensional acoustic topological states and the nonlinear effects of strongly coupled systems; in terms of application, the topological rainbow trap has significant potential in noise control and acoustic communication, and the acoustic realization of the Kitaev chain lays the foundation for the classical simulation of topological quantum computing. In addition, how to combine topological acoustics with intelligent materials and programmable metasurfaces to realize dynamically reconfigurable devices is an important topic for the next stage. 
\end{abstract*}

% 生成目录
\tableofcontents
% 生成图片清单
% \listoffigures
% 生成表格清单
% \listoftables

%---------------------------------------------------------------------
%	正文部分
%---------------------------------------------------------------------
\mainmatter

% 符号表
% 语法与 description 环境一致
% 两个可选参数依次为说明区域宽度、符号区域宽度
% 带星号的符号表(notation*)不会插入目录
% \begin{notation}[10cm]
%   \item[DFT] 密度泛函理论 (Density functional theory)
%   \item[DMRG] 密度矩阵重正化群 (Density-Matrix Reformation-Group)
% \end{notation}

% 建议将论文内容拆分为多个文件
% 即新建一个 chapters 文件夹
% 把每一章的内容单独放入一个 .tex 文件
% 然后在这里用 \include 导入,例如
\chapter{绪论}
\section{引言}
声学,作为物理学的一个重要分支,研究声波的产生、传播及与物质的相互作用,具有几千年的历史。从古代到现代,声学的发展经历了从经验性观察到系统化理论研究的演变。最早的声学研究主要来源于对自然现象的观察。古希腊毕达哥拉斯(Pythagoras)通过对弦的振动进行实验,发现了音高与弦长、张力和质量之间的关系,为声学提供了初步的实验基础。亚里士多德(Aristotle)提出声音是通过介质传播的这一理论,为后来的声波传播研究提供了启示。中国古代也对音律和声波传播有过观察,如《礼记》中的“宫商角徵羽”五音系统,反映了中国古人对声音与音高的初步认知。进入17世纪,科学革命推动了声学研究进入了一个新的阶段。伽利略(Galileo Galilei)通过对弦乐器的研究揭示了振动与声音之间的关系。罗伯特·胡克(Robert Hooke)在其《弹性理论》中深入探讨了弹性体的振动特性,并提出了弹性波的传播理论,这对声波传播的理解产生了重要影响。进入19世纪,雷利爵士(Lord Rayleigh)对声学做出了具有里程碑意义的贡献。其著作《声音的理论》(The Theory of Sound, 1877)详细阐述了声波的传播规律,并提出了声波在不同介质中传播的速度、反射、折射等现象的数学描述。雷利的工作奠定了现代声学的数学基础,并对波动理论的普及产生了重要影响。20世纪,声学进入了更加多样化和精细化的研究阶段。随着实验技术的不断发展,声学研究逐渐拓展到多个方向,包括超声学、建筑声学、声学成像和音频声学等领域。声学理论得到了前所未有的发展,尤其是在声学超材料(Acoustic Metamaterials)的研究中。

超材料是指通过人工设计的结构,其在某些条件下表现出自然材料没有的异常物理性质。声学超材料的研究起源于能带理论的提出和光子晶体的成功发现。能带理论最早由Felix Bloch提出\cite{a1},用以描述电子在周期性晶体中的运动,揭示了周期性结构如何形成能带和禁带。这一理论的成功应用,激发了人们对周期性结构在其他波动现象中的潜力的探索。1987年,John Yablonovitch提出了光子晶体的概念\cite{a2},通过周期性结构控制光波的传播,并形成光学带隙。光子晶体的出现展示了通过人工设计材料微观结构,可以调节波动传播的方向、速度及频率,为超材料的研究开辟了新的方向\cite{a3,a4,a5}。受到光子晶体启发,声学领域的研究者开始探索如何借鉴类似的理念来调控声波的传播。1993年,Kushwaha等人第一次明确提出了声子晶体(Phononic Crystals)的概念\cite{b1},不久以后Martinez等人通过实验验证了声子晶体的禁带特性\cite{b2,b3}。随后,Liu等人提出了局域共振声子晶体的概念\cite{b4},展示了局域共振声子晶体在低频声波的有效控制能力,为声波调控提供了新的思路。声学超材料因其独特的声波控制能力,广泛应用于多个领域,如声学负参数材料\cite{c11,c12,c13,c14,c15,c16,c17,c18},反常声传输\cite{c21,c22,c23,c24,c25,c26,c27,c28,c29},声学超透镜\cite{c31,c32,c33,c34,c35,c36,c37,c38,c39},声隐身\cite{c41,c42,c43,c44,c45,c46},声学轨道角动量\cite{c51,c52,c53,c54,c55,c56,c57,c58},声学非互易\cite{c61,c62,c63,c64,c65,c66},声学黑洞\cite{c71,c72,c73,c74,c75,c76},通风降噪\cite{c81,c82,c83,c84,c85}等等。
\chapter{声学边界的影响}
\section{拓扑绝缘体}
\chapter{基于Kagome晶格的声学哈密顿量推导及边界条件对拓扑态的影响研究}

\section{引言}

\section{Kagome晶格与声学结构类比}

\begin{figure}[h!]
  \centering
  \includegraphics[width=1\textwidth]{images/fig3-1.eps} 
  \caption{以Kagome结构为例,电子系统与声学系统存在显著的相似性。}
  \label{fig_3_1}
\end{figure}

在物理学中,电子系统与声学系统存在显著的相似性。电子系统里,格点、相互作用、能量、电荷分布是关键要素;而在声学系统中,可类比为腔体、导管、频率、声场分布。以图\ref{fig_3_1}的Kagome晶格为例,以下是电子系统和声学系统中几个要素的相似性详细解释:
\begin{itemize}
  \item 格点与腔体:在电子系统的晶格模型中,格点是电子所处的位置,电子在这些格点上具有一定的能量状态和量子特性。格点的排列和相互关系决定了电子的能带结构等重要物理性质。而在声学系统中,我们可以把单个腔体看作是一个共振单元,它也具有共振频率和对应的共振模式。而使用周期排列的声学腔体,可以模仿电子晶格空间排列,构造相同的空间对称性。二者在离散化的单个单元共振和多个单元的空间排列上,存在一定的相似性。
  \item 电子系统的相互作用与声学导管:在电子系统的晶格模型中,电子在晶格中的跳跃等过程可以看作是一种相互作用,它决定了电子在不同格点之间的转移和能量传递。在声学系统中,我们可以在腔体之间连接导管,为不同腔体提供相互作用,并通过调节导管的参数实现对这种相互作用的大小和正负符号的调节。二者在单元和单元的相互作用上存在一定的相似性。
  \item 能量与频率:在电子系统中,薛定谔方程用于求解电子能量的本征值问题,其能量以离散本征值呈现,相应本征向量描述电子运动状态 。当处于周期结构时,电子会形成能带,其特性受晶格周期性影响。在声学系统里,波动方程的求解同样属于本征值问题,频率(或者频率的平方)作为本征向量,其离散取值决定声波传播模式。在周期结构的声学体系中,声波也会形成类似的能带结构,该结构与声学单元的周期性排列密切相关。二者在本征值问题架构及周期结构下形成能带的特性上,展现出显著的相似性。 
  \item 电荷分布与声场分布:在电子系统中,电荷分布反映了电子于空间的分布态势,它与电子的能量状态、相互作用及外部电场紧密相连,电荷分布的不均匀会引发电场等物理效应,对电子系统的电学性质和物理行为产生影响。在声学系统里,声场分布体现了声波在空间中的强度、相位分布状况,腔体和导管的结构以及声波传播特性致使声场在空间呈现不同分布模式。二者相似之处在于,它们均是描述各自系统内物理量在空间的分布,且这种分布都会显著影响系统与其他物质的相互作用,以及整个系统的性能表现。特别地,在拓扑绝缘体的研究中,我们可以观测二者的场分布,从而观测其拓扑性质。 
\end{itemize}

从类比的相似性出发,我们构造了如图\ref{fig_3_1}所示的声学结构。图中周期结构以C$_{3}$对称性排列,其中$a_1$和$a_2$是两个方向晶格基矢的大小,用于描述晶格的周期性结构。右侧展示了单个晶格的三维结构:一个由三个相同的六边形亥姆霍兹谐振器通过波导管连接而成的三角晶格,并考虑$C_3$对称性和平移不变性,即所有谐振器是相同的,其边长为$d$,高度为$H$。$l_w$和$l_v$分别对应胞内耦合的导管和胞间导管的长度,$r_w$和$r_v$是分别对应胞内耦合的导管和胞间导管的半径。从直觉上而言,当连接两个腔体的导管的长度更短,横截面积更大,此时两个腔体之间的相互作用更大,对应电子系统中更大的格点间的跳跃。

\section{Kagome晶格的声学哈密顿量}

\begin{figure}[h!]
  \centering
  \includegraphics[width=1\textwidth]{images/fig3-2.eps} 
  \caption{Kagome声学系统体晶格的等效电路:
  (a)元胞Kagome晶格结构,其中晶格0居中,周围相邻的晶格分别编号为1到6。(b)晶格0的等效电路图。
  }
  \label{fig_3_2}
\end{figure}

如上一节所述,电子系统和声学系统有着相似性,我们可以构造声学腔管结构来类别电子Kagome晶格,也可以通过直觉调控其胞间和胞内的相互作用。在本节中,从声电类比方法出发,我们在亚波长尺度下从传统声学的角度严格推导出具有二维周期排列Kagome晶格的声共振系统的哈密顿量,从而揭示理论模型中的胞内或胞间跳跃与实际物理系统中的声学参数之间的联系。对于有限大结构,由于体,边,角上晶格的相邻晶格数量不同,我们也分别作了讨论。

首先,我们可以通过声学-电学类比来定义声学系统的 lumped 参数模型。如图\ref{fig_3_2}(a)所示,对于在周期结构中的晶格0而言,其周围有六个最近邻晶格,编号为1到6。单个晶格0的等效电路图如图\ref{fig_3_2}所示。在该电路模型中,腔体被表示为电容 \( C \),波导被表示为电感 \( L_v \) 和 \( L_w \)。具体公式如下:
\begin{equation} \label{eq3-1}
  C = \frac{V}{\rho c^2},
\end{equation}
\begin{equation} \label{eq3-2}
  L_v = \frac{\rho (l_v + 1.7r_v)}{\pi r_v^2}, \quad L_w = \frac{\rho (l_w + 1.7r_w)}{\pi r_w^2},
\end{equation}
其中,\( V \) 是腔体的体积,\( \rho \) 是空气的密度,\( c \) 是声速,\( r_v \) 和 \( r_w \) 分别是波导的半径,\( l_v \) 和 \( l_w \) 是它们的长度。

通过应用基尔霍夫电流定律,我们可以得到了描述声学波动的方程。对于周期性结构,腔体的声压在每个晶格点处满足如下方程:
\begin{subequations}\label{eq3-3}
  \begin{align}
  -(2w + 2v)u_{1}^{0} + wu_{2}^{0} + wu_{3}^{0} + vu_{2}^{6} + vu_{3}^{1} &= \omega^{2}u_{1}^{0}\label{eq:sub1}\\
  -(2w + 2v)u_{2}^{0} + wu_{3}^{0} + wu_{1}^{0} + vu_{3}^{2} + vu_{1}^{3} &= \omega^{2}u_{2}^{0}\label{eq:sub2}\\
  -(2w + 2v)u_{3}^{0} + wu_{1}^{0} + wu_{2}^{0} + vu_{1}^{4} + vu_{2}^{5} &= \omega^{2}u_{3}^{0}\label{eq:sub3}
  \end{align}
\end{subequations}
其中,$u_m^n$表示第$m$个晶格的第$n$个空腔中的声压,且$v = -1/L_vC$,$w = -1/L_wC$。对于一个周期性结构,它保证了$u_m^n$可以被描述为布洛赫波函数,方程(3)可以重写为矢量形式:
\begin{equation}\label{eq3-4}
  \mathcal{H}_{0}\mathbf{u} = \omega^{2}\mathbf{u},
\end{equation}
其中,$\mathbf{u} = [u_0^1\ u_0^2\ u_0^3]^{\mathrm{T}}$,并且$\mathcal{H}_{0}$可以表示为:
\begin{equation}\label{eq3-5}
  \mathcal{H}_{0}(\mathbf{k}) = 
  \begin{bmatrix}
  -2w - 2v & w + ve^{j\mathbf{k}\cdot(\mathbf{a}_{1}+\mathbf{a}_{2})} & w + ve^{j\mathbf{k}\cdot\mathbf{a}_{1}} \\
  w + ve^{-j\mathbf{k}\cdot(\mathbf{a}_{1}+\mathbf{a}_{2})} & -2w - 2v & w + ve^{-j\mathbf{k}\cdot\mathbf{a}_{2}} \\
  w + ve^{-j\mathbf{k}\cdot\mathbf{a}_{1}} & w + ve^{j\mathbf{k}\cdot\mathbf{a}_{2}} & -2w - 2v
  \end{bmatrix}
\end{equation}
其中,$\mathbf{k}$是布洛赫波矢,$\mathbf{a}_{1}$、$\mathbf{a}_{2}$表示晶格常数。显然可以看出,求解周期结构的共振频率及其相应的特征模式可归结于求解$\mathcal{H}_{0}$的本征值问题,这与电子系统的哈密顿量相对应,$w$和$v$对应于Kagome晶格的胞内跳跃和胞间跳跃。我们把$\mathcal{H}_{0}$称为无限大周期排列的Kagome结构的声学哈密顿量。

\begin{figure}[h!]
  \centering
  \includegraphics[width=1\textwidth]{images/fig3-3.eps} 
  \caption{Kagome声学系统边界上晶格的等效电路:
  (a)边界上Kagome晶格结构,其中晶格0为边上晶格,周围相邻的晶格分别编号为1到4。(b)绝对硬边界时晶格0的等效电路图。(b)绝对软边界时晶格0的等效电路图。
  }
  \label{fig_3_3}
\end{figure}

我们注意到,对于周期系统,该结构相应哈密顿量的对角项(式\ref{eq3-5})是相同的。然而,若考虑有限大系统的边界,由于边、角晶格相邻晶格的数目体晶格相邻晶格数目不同,这会导致哈密顿量的对角项。有限结构哈密顿量中对角项的差异会对拓扑性质产生巨大影响,这是由于广义手征对称性的差异所致,而广义手征对称性保护着拓扑态。我们以边界晶格为例说明这种有限大系统边界上哈密顿量随边界条件的改变,如图\ref{fig_3_3}所示,我们假设在有限结构的边缘存在一个晶格0。与图\ref{fig_3_2}(a)相比,图\ref{fig_3_3}(a)中的相邻晶格5和6被与之相连的最外层管道的硬边界或软边界所取代。在集总电路模型中,硬边界相当于断路情况,而软边界相当于接地,分别如图\ref{fig_3_3}(b)和图\ref{fig_3_3}(c)所示。相应地,对于硬边界情况,边缘晶格的方程可如下获得:
\begin{subequations}\label{eq3-6}
  \begin{align}
  -(2w + v)u_{0}^{1} + wu_{0}^{2} + wu_{0}^{3} + vu_{1}^{3} &= \omega^{2}u_{0}^{1}, \\
  -(2w + 2v)u_{0}^{2} + wu_{0}^{3} + wu_{0}^{1} + vu_{2}^{3} + vu_{3}^{1} &= \omega^{2}u_{0}^{2}, \\
  -(2w + v)u_{0}^{3} + wu_{0}^{1} + wu_{0}^{2} + vu_{4}^{1} &= \omega^{2}u_{0}^{3}, \
  \end{align}
\end{subequations}
而软边界情况的公式为:
\begin{subequations}\label{eq3-7}
  \begin{align}
  -(2w + 2v)u_{0}^{1} + wu_{0}^{2} + wu_{0}^{3} + vu_{1}^{3} &= \omega^{2}u_{0}^{1}, \\
  -(2w + 2v)u_{0}^{2} + wu_{0}^{3} + wu_{0}^{1} + vu_{2}^{3} + vu_{3}^{1} &= \omega^{2}u_{0}^{2}, \\
  -(2w + 2v)u_{0}^{3} + wu_{0}^{1} + wu_{0}^{2} + vu_{4}^{1} &= \omega^{2}u_{0}^{3}. 
  \end{align}
\end{subequations}
将方程\ref{eq3-6}与方程\ref{eq3-7}进行比较,可以明显看出边界条件的影响恰好反映在哈密顿量对角项的差异上。总体而言,软边界使得所有晶格(包括体晶格和边界晶格)的对角项都等于$-(2w + 2v)$。从这个角度来看,应用绝对软边界的有限大声学系统可以被视为是更接近电子系统的类比。

\section{Kagome晶格体能带与拓扑相变}

\begin{figure}[h!]
  \centering
  \includegraphics[width=1\textwidth]{images/fig3-4.eps} 
  \caption{Kagome声学系统的体能带:
  (a)$w=v$(b)绝对硬边界时晶格0的等效电路图。(b)绝对软边界时晶格0的等效电路图。
  }
  \label{fig_3_4}
\end{figure}

上一节中,我们推导了这种腔管结构的声学方程。接下来,我们研究这种晶格周期排列时的拓扑性质。在图\ref{fig_3_4}中,谐振器的边长和高度分别为$d = 10$(mm)和$h = 25$(mm)。波导管的长度为$l_w = l_v = 2.5$(mm),$V$是腔体的体积。空气的质量密度和相应的声速分别定义为$\rho = 1.23$(kg/m³)和$c = 343$(m/s)。一般来说,带反转发生在胞间跳跃的变化处,此模型中的胞间跳跃和胞内跳跃由胞间管半径$r_w$和胞内管半径$r_v$分别确定。

使用COMSOL Multiphysics进行模拟,图\ref{fig_3_4}描绘了当$r_w = r_v = 0.55$(mm),$r_w = 0.75$(mm)且$r_v = 0.3$(mm),$r_v = 0.3$(mm)且$r_v = 0.75$(mm)时的能带结构。由$v = -1/L_vC$,$w = -1/L_wC$,这三种情况分别对应$v/w=1$,$v/w<1$和$v/w>1$。三个图中的平带来自于相消干涉,这导致了紧致局域态,而另外两条能带反映了拓扑性质。可以明显看出,当$r_w = r_v$时,存在一个受对称性保护的狄拉克锥,这表明拓扑相变的临界点。对于$C_3$对称晶格,体极化$p_l$定义为:
\begin{equation} \label{eq3-8}
  e^{-i\pi p_l} = \prod_{n \in \text{occ}} \frac{\theta_n(\mathbf{K})}{\theta_n(\Gamma)},
\end{equation}
其中$\theta_n(\mathbf{k}) = (u_n(\mathbf{k})|R_{3}|u_n(\mathbf{k}))$是通过将三重对称算子$R_3$(旋转$2\pi/3$)应用于晶格高对称点处相应的布洛赫波函数$u_n(\mathbf{k})$计算得到的,“occ”表示占据带。由于只有一个能带位于带隙下方,体极化可表示为:
\begin{equation} \label{eq3-9}
  (p_1,p_2) = 
  \begin{cases}
  (-1/3,-1/3), & w < v \\
  (0,0), & w > v
  \end{cases}
\end{equation}
因此,当$w < v$时的非零极化表明了能带的非平凡拓扑相,而当$w > v$时则表示平凡相。由于所有原子被认为是相同的,边界诱导的填充异常预计将沿着非平凡体极化情况被诱导,这由拓扑边缘态的存在表示,此时边界是开放的。同时,当所有原子在热力学极限下相同时,也可以诱导出由低维拓扑角态表征的角诱导填充异常。

到目前为止,我们讨论的是所有原子在热力学极限下相同的情况。然而,在下一节中,我们将展示当涉及到封闭经典系统时,系统的onsite能量(体现为哈密顿量的对角项)总是受到边界的影响。

\section{Kagome晶格的边界态和角态}

\section{边界条件对拓扑态的影响}
\chapter{基于$C_4$对称的声学拓扑晶体绝缘体的声传播与彩虹捕获研究}
\section{引言}

对于那些无自旋并且时间反演不变的拓扑晶体绝缘体而言,其独特的拓扑态是由体极化来进行表征的。这种特性使得它在经典波系统中相对来说更容易得以实现,如此一来,便自然而然地为利用特定的能量(也就是特定的频率)在结构的边界或者角落之处,以一种鲁棒的方式俘获波提供了一个极为理想的平台。

然而,尽管已经有大量细致入微的研究,无论是从理论层面还是实验层面,都确凿地证明了拓扑晶体绝缘体(TCIs)中存在着拓扑态,但是,像拓扑态的群速度这类与传播特性相关的内容,却鲜少被人们深入地探讨和研究。可实际上,这些传播特性在拓扑晶体绝缘体的实际应用当中,却是一个至关重要且无法回避的问题。

举个例子来说,利用拓扑态来实现彩虹俘获就是一个典型的应用场景。所谓的彩虹俘获,指的是将具有不同频率的波进行分离,并且让它们分别被俘获在不同的空间位置上。这种现象已经在多个领域的系统中被提出,比如在电磁波系统\cite{C41-1,C41-2,C41-3,C41-4,C41-5,C41-6,C41-7,C41-8,C41-9}、弹性波系统\cite{C42-1,C42-2,C42-3,C42-4,C42-5,C42-6}以及空气声系统\cite{C43-1,C43-2,C43-3,C43-4}中。在这些相关的研究工作里,研究人员们纷纷采用了各种各样的方法,其目的就是为了能够控制不同系统中波的色散关系,从而使得波能够在空间的不同位置上实现局域化。

不仅如此,一些近期的研究工作更是已经在拓扑态中成功地实现了彩虹俘获\cite{C44-1,C44-2,C44-3,C44-4,C44-5,C44-6}。在这些研究中,隙内模式能够在不减小体带隙的情况下被有效地减慢,而且体带隙依然能够受到强无序的良好保护。不过,令人遗憾的是,现有的这些方法通常都需要在体中设置额外的区域或者对晶格进行相应的变化。从实际应用的角度出发,我们迫切地需要一种全新的方法,这种方法能够直接对边界处拓扑态的传播速度进行有效的控制。
\chapter{Kitaev}

\section{引言(未完成}

在前面的章节里我们阐述了拓扑绝缘体在声学系统中的建模方法,并且以彩虹捕获为例说明了拓扑物态在声学应用中的潜力。除此之外,声学系统由于其宏观且易于实现的优势,成为观测新型量子拓扑物态的一个极佳的平台。这一章节里,我们以Kitaev链为例,在声学系统中实现了类似的结构,并且观测到了与马约拉纳零模类似的声学模态。

一方面,拓扑绝缘体具有多重拓扑相这一概念,吸引了众多研究者的广泛关注。在这以后,由于能够操控特殊的鲁棒电子态,这些电子态可抵御微观无序并支持无损能量传输,不同分类的拓扑绝缘体和拓扑超导体已被认为在量子技术方面具有重大潜力。尤其在拓扑量子计算领域,拓扑超导体有望为一类非阿贝尔任意子(也就是马约拉纳费米子)提供颇具价值的解决方案。这些马约拉纳费米子或许能以非局域且本质上无退相干的方式,将量子信息嵌入到量子计算机的实验合成中\cite{r31,r32,r33}。2001年,Kitaev提出了一个著名的模型——无自旋一维(1D)玩具模型,即无自旋p波超导N位链。该链能够支持未配对的端马约拉纳零模,这些零模在零能量处简并,而这是因为该系统在拓扑上具有非平凡的特性\cite{r4}。这一模型激发了大量关于在超流体和超导体中实现这种一维超导导线的研究\cite{r51,r52,r53,r54,r55,r56,r57,r58,r59,r510},但由于假设的无自旋费米子的缺乏以及p波配对的罕见性,要合成实际的基塔耶夫链,至今仍是一个持续存在的挑战。

另一方面,正如前文所述,经典波系统凭借其灵活性和可重构性,长期以来都被当作设计拓扑能带以及观测新奇量子现象的重要平台。但是,如果第三章采样的电声类比方法对拓扑绝缘体进行建模时,我们会发现这种方法亚波长尺度下无法构造两种符号相反的相互作用,即电子的正跳跃和负跳跃。而这是Kitaev提出它的玩具模型中必须的。因此,我们需要在声学系统里找到新的方法解决这个问题。值得关注的是,尽管Kitaev链的机械对应物已经在一些实验工作中有所研究\cite{C54},但在经典波系统中,能否实现具有可观测马约拉纳样零模的Kitaev链类似物,依旧是一个尚未解决的问题。

在本章中工作目的是将无自旋p波Kitaev链的概念从理论和实验两方面引入到声学系统中,并且在与原子系统的严格对应中,揭示了所创建的马约拉纳样零模的性质。与其他情况不同的是,在所设想的一维声线中,马约拉纳样零模无需额外的外场来利用超导近邻效应,而且能够通过声刺激直接被激发。通过巧妙地设计基于共振声学系统的Kitaev链的一维线网络,体声系统的拓扑相变以及未配对马约拉纳零模的出现都可以被精准观测到。更为重要的是,实验不仅验证了这种奇异拓扑态的存在,还展示了所呈现结构的“键盘”特性,也就是链的个体可调配置。这一特性使得在保持体隙的同时,能够对每个位点的拓扑进行局部控制,还可以对类似马约拉纳费米子的类似物进行灵活操控。这些令人着迷的结果意义非凡,很可能为经典波系统开辟新的途径,有助于在宏观尺度上对新型类量子材料展开探索。

本章内容如下:...

\section{kitaev链性质}

让我们将注意力集中在描述无自旋p波超导N位链的基塔耶夫玩具模型的最小哈密顿量H上,其形式为[12]:

\[H = -\mu \sum_{n = 1}^{N} c_{n}^{\dagger} c_{n} - \sum_{n = 1}^{N - 1} (t c_{n + 1}^{\dagger} c_{n} + \Delta c_{n} c_{n + 1} + \text{H.c.}), \quad (1)\]

其中\(t > 0\),\(\mu\)和\(\Delta\)分别表示最近邻跳跃强度、化学势和p波配对振幅。\(c_{n}\)是第\(n\)个位点的无自旋费米子算符。一旦施加周期性边界条件,动量空间中的博戈留波夫 - 德热纳哈密顿量可写为

\[H = \frac{1}{2} \sum_{k} C_{k}^{\dagger} \mathcal{H}(k) C_{k}, \quad (2)\]

其中\(C_{k}^{\dagger} = [c_{k}^{\dagger}, c_{-k}^{\dagger}]\),且\(\mathcal{H}(k) = (-2t \cos k - \mu) \tau_{z} + 2\Delta \sin k \tau_{y}\),其中\(\tau\)是泡利矩阵。进一步,通过对角化\(\mathcal{H}(k)\),可立即得到体能量本征值\(E(k) = \pm \sqrt{(2t \cos k + \mu)^{2} + 4\Delta^{2} \sin^{2} k}\)。相应地,当\(\vert \mu \vert = 2t\)时,在\(k = 0\)处发生体隙闭合,这恰好表明拓扑相变的临界点。

此外,虽然众所周知凝聚态物质系统由电子组成,这些电子总是对应于成对的马约拉纳费米子,但事实证明,在基塔耶夫链中起关键作用的未成对马约拉纳费米子可以通过哈密顿量以一些特殊方式实现。为了理解链中拓扑相变以及未成对马约拉纳费米子的出现,我们现在用马约拉纳算符\(\gamma_{n,a}\)和\(\gamma_{n,b}\)(其中\(\gamma_{n,\alpha} = \gamma_{n,\alpha}^{\dagger}\))替换无自旋费米子\(c_{n}\),即\(c_{n} = (\gamma_{n,a} - i\gamma_{n,b}) / 2\),那么式(1)可重写为

\[H = \frac{i}{2} \left\{ \sum_{n = 1}^{N - 1} [(-\Delta - t) \gamma_{n,b} \gamma_{n + 1,a} + (-\Delta + t) \gamma_{n + 1,b} \gamma_{n,a}] + \mu \sum_{n = 1}^{N} \gamma_{n,a} \gamma_{n,b} \right\}. \quad (3)\]

此外,谨慎地考虑了两种极限情况。在第一种情况下,我们设\(\Delta = t = 0\)且\(\mu \neq 0\),此时\(H\)简化为\(H = (i / 2) \mu \sum_{n = 1}^{N} \gamma_{n,a} \gamma_{n,b}\),这表明马约拉纳模式\(\gamma_{n,a}\)和\(\gamma_{n,b}\)总是在同一位置成对,对应于平凡相。关键的是,在第二种极限情况中,当\(\Delta = t\)且\(\mu = 0\)时,\(H = it \sum_{n = 1}^{N - 1} \gamma_{n,b} \gamma_{n + 1,a}\),这表明马约拉纳费米子从相邻位点成对,这自然使得\(\gamma_{1,a}\)和\(\gamma_{N,b}\)不成对,并由于粒子 - 空穴对称性在零能量处充当端马约拉纳模式简并。同时,波函数对\([1, 0]^T\)和\([0, 1]^T\)在高对称点处反转也表明了拓扑相变。结果是,\(\vert \mu \vert < 2t\)表示具有部分填充能带对的非平凡超导态,而\(\vert \mu \vert > 2t\)对应于没有马约拉纳费米子出现的平凡拓扑。

\section{声学哈密顿量推导}

为了审慎地展示Kitaev链的严格声学对应性,我们详细展示了完整的推导过程。

我们从一个单声学腔开始,该腔体与单元腔中连接的管道相连,如主文图1(a)所示(图S1中也提供了更详细的信息)。该腔体对应的经典格林函数定义为 \( G(\vec{r}, \vec{r}') \)。

当声波传播频率 \( \omega \) 位于特定模式下(即 \( P_z \)-偶极模式),其声速势场的归一化分布为:
\[
\psi(\vec{r}) = \sqrt{\frac{2}{w^2 h}} \sin\left(\frac{\pi x}{h}\right),
\]
如主文所示,此时 \( G(\vec{r}, \vec{r}') \) 可以简化为:
\[
G(\vec{r}, \vec{r}') = \sum_j \frac{c_0^2}{\omega_j^2 - \omega^2} \psi_j(\vec{r}) \psi_j(\vec{r}')
\approx \frac{c_0^2}{\omega_0^2 - \omega^2} \psi(\vec{r}) \psi(\vec{r}'). 
\]

其中,\( \psi(\vec{r}) \) 是位置 \( \vec{r} \) 处的声速势,\( \omega_0 \) 是 \( P_z \)-偶极模式的固有频率,该模式表现为腔内的驻波。需要强调的是,当 \( \omega \) 接近 \( \omega_0 \) 时,上述近似是成立的。

因此,腔体中声压的场分布可以表示为:
\[
p(\vec{r}) = -i \rho \omega \oint_{\partial V} G(\vec{r}, \vec{r}') \vec{v}(\vec{r}') \mathrm{d}S'
= -\frac{i \rho c_0^2 \omega}{\omega_0^2 - \omega^2} \oint_{\partial V} \psi(\vec{r}') \vec{v}(\vec{r}') \mathrm{d}S', 
\]
其中 \( \vec{v}(\vec{r}') \) 表示声学速度,\( S' \) 表示腔体的表面积。

由于声学硬边界条件使得 \( \vec{v} = 0 \),方程(S2)中的积分仅在管道的两端不为零。因此,有效面积 \( S' \) 可表示为:
\[
S' = S^t + 2S^A + S^\mu,
\]
其中上标分别表示对应的管道区域,如图S1(b)中的虚线区域所示。

---

为了进一步区分单元腔体中的两个腔体的参数,我们引入下标 \( j = 1, 2 \),并定义Kitaev链中波函数的关键参数 \( \xi = [\xi_1, \xi_2]^T \)。它们与Kitaev链中的波函数相关联,可以定义为:
\[
\xi_1 = \frac{p_1(\vec{r})}{v_1(\vec{r})} = \frac{-i \rho c_0^2}{\omega_0^2 - \omega^2} \oint_{\partial V_1} \psi^*(\vec{r}') \vec{v}_1(\vec{r}') \mathrm{d}S',
\]
\[
\xi_2 = \frac{p_2(\vec{r})}{v_2(\vec{r})} = \frac{-i \rho c_0^2}{\omega_0^2 - \omega^2} \oint_{\partial V_2} \psi^*(\vec{r}') \vec{v}_2(\vec{r}') \mathrm{d}S'. 
\]

将方程(S2)代入并结合体积和表面特性,可以分别表示为:
\[
\xi_1 = -\frac{c_0^2 d^2}{2(\omega_0^2 - \omega^2)} \left[ -\nu_1(0) \psi_1^A + \nu_1(0) \psi_1^A e^{-ika} + \nu_1^A(\psi_2^A e^{-ika}) \right],
\]
\[
\xi_2 = -\frac{c_0^2 d^2}{2(\omega_0^2 - \omega^2)} \left[ -\nu_2(0) \psi_2^A + \nu_2(0) \psi_2^A e^{-ika} + \nu_2^A(\psi_1^A e^{-ika}) \right]. 
\]

其中 \( V = w^2 h \) 是腔体的体积,\( \bar{\psi}_j^m \) 是与 \( j \)-th 腔连接的第 \( m \)-根管末端声速势的平均值。对于所有管道具有相同的横截面积和 \( r_c \) 值,由以下关系得出:
\[
\bar{\psi}_1 = \bar{\psi}_2 = -\bar{\psi}_1^t = -\bar{\psi}_2^t = \psi,
\]
因此可以得出 \( \bar{\psi}_j^m = |\psi| \)。同时需要注意,这种近似在管道尺寸远小于腔体尺寸的情况下是适用的。

根据这些结果,可以以 \( \xi \) 表示声压:
\[
p_j^t(0) = \xi_j \bar{\psi}_j, \quad
p_j^t(l_m^t) = \xi_j \bar{\psi}_j^t, \quad
p_j^t(f_j) = e^{ika} \xi_j \bar{\psi}_j^t,
\]
\[
p_1^A = e^{ika} \xi_2 \bar{\psi}_2^t, \quad
p_2^A = e^{ika} \xi_1 \bar{\psi}_1^t,
\]
\[
p_j^\mu = \xi_j \sqrt{\frac{2}{V}}. 
\]

为了映射严格的对应关系,我们现在重点分析管道的声学连接条件。假设结构内声波以平面波形式传播,则在一个单元腔中第 \( j \)-th 腔的声压和声速可以表示为:
\[
p_m^j(l) = A_m^j e^{i\omega l/c_0} + B_m^j e^{-i\omega l/c_0},
\]
\[
\rho_0 c_0 v_m^j(l) = A_m^j e^{i\omega l/c_0} - B_m^j e^{-i\omega l/c_0}, 
\]
其中 \( m \) 表示 \( t \) 和 \( \Delta \)。

通过将 \( l = 0 \) 和 \( l = l_m^t \) 代入方程(S4),可以很容易地得到以下关系:
\[
i\rho c_0
\begin{pmatrix}
v_j^t(0) \\
v_j^t(l_m^t)
\end{pmatrix}
=
\begin{pmatrix}
\cot(\omega l_m^t / c_0) & -\csc(\omega l_m^t / c_0) \\
-\csc(\omega l_m^t / c_0) & \cot(\omega l_m^t / c_0)
\end{pmatrix}
\begin{pmatrix}
p_j^t(0) \\
p_j^t(l_m^t)
\end{pmatrix}. 
\]

对于额外的管道(标记为 \( \mu \))且一端闭合,则满足以下关系:
\[
v_j^\mu = \frac{p_j^\mu}{Z_j^\mu},
\]
其中 \( Z_j^\mu \) 是连接腔体的端部阻抗。当闭合端为声学硬边界时:
\[
Z_j^\mu = -i\rho c_0 \cot(\omega l_j^\mu / c_0),
\]
而当闭合端为声学软边界时:
\[
Z_j^\mu = i\rho c_0 \tan(\omega l_j^\mu / c_0)。
\]

进一步地,将方程(S4)-(S7)代入方程(S3),波函数方程可以以矩阵形式表示为:
\[
\omega^2 \xi = (H_0 + H_a(k)) \xi, 
\]
其中:
\[
H_0 =
\begin{pmatrix}
\omega_0 + \epsilon_1 & 0 \\
0 & \omega_0 + \epsilon_2
\end{pmatrix},
\quad
H_a(k) =
\begin{pmatrix}
\mu_1 + 2t_1\cos(ka) & \Delta_1 e^{ika} + \Delta_2 e^{-ika} \\
\Delta_1 e^{-ika} + \Delta_2 e^{ika} & \mu_2 + 2t_2\cos(ka)
\end{pmatrix}. 
\]
其中
\[
\epsilon_1 = \frac{cd^2 |\psi|^2}{2} \left[ 2\cot\left(\frac{\omega_0 l_1^t}{c_0}\right) + \cot\left(\frac{\omega_0 l_1^A}{c_0}\right) + \cot\left(\frac{\omega_0 l_2^A}{c_0}\right) \right],
\]
\[
\epsilon_2 = \frac{cd^2 |\psi|^2}{2} \left[ 2\cot\left(\frac{\omega_0 l_2^t}{c_0}\right) + \cot\left(\frac{\omega_0 l_1^A}{c_0}\right) + \cot\left(\frac{\omega_0 l_2^A}{c_0}\right) \right],
\]
\[
t_1 = -\frac{cd^2 |\psi|^2}{2} \csc\left(\frac{\omega_0 l_1^t}{c_0}\right), \quad
t_2 = -\frac{cd^2 |\psi|^2}{2} \csc\left(\frac{\omega_0 l_2^t}{c_0}\right),
\]
\[
\Delta_1 = -\frac{cd^2 |\psi|^2}{2} \csc\left(\frac{\omega_0 l_1^A}{c_0}\right), \quad
\Delta_2 = -\frac{cd^2 |\psi|^2}{2} \csc\left(\frac{\omega_0 l_2^A}{c_0}\right),
\]
\[
\mu_1 = \frac{i\rho c_0^2 d^2}{V Z_1^\mu}, \quad
\mu_2 = \frac{i\rho c_0^2 d^2}{V Z_2^\mu}. 
\]
注意:一旦声学结构确定,所有这些参数都可以直接计算。需要特别指出的是,方程(S10)表明,所有由 \( P_z \) 模式描述的关键强度参数(\( t, \Delta, \mu \))是解耦的,这与由声学系统谐振基频描述的跃迁不同,因此可以独立设计。

为了构造严格的声学Kitaev链,需要满足以下条件:
\[
l_1^t = l_2^t + h, \quad t_1 = -t_2 = -t, \quad
l_1^A = l_2^A + h,\quad \Delta_1 = -\Delta_2 = -\Delta,
\]
并且 \( Z_1^\mu = -Z_2^\mu \),当 \( \mu_1 = -\mu_2 = -\mu \) 时,这些设置自然确保 \( \epsilon_1 = \epsilon_2 = \epsilon \)。因此,方程(S9)可以简化为:
\[
H_0 =
\begin{pmatrix}
\omega_0 + \epsilon & 0 \\
0 & \omega_0 + \epsilon
\end{pmatrix},
\quad
H_a(k) =
\begin{pmatrix}
-\mu - 2t\cos(ka) & -2i\Delta\sin(ka) \\
2i\Delta\sin(ka) & \mu + 2t\cos(ka)
\end{pmatrix}. 
\]

\section{实际结构}

我们现在关注上述讨论的基塔耶夫链的声学对应物,晶格常数为\(a\)的一维线结构如图1(a)所示。这里每个晶胞由两个单独的长方体声学腔组成,其长度和宽度分别为\(w\)和高度\(h\)。晶胞通过四根弯曲的管子复杂地连接(在图1(a)中分别用红色、黄色、蓝色和绿色标记),这些管子具有相同的边长\(d\)和有效长度\(l_{1}^{t}\)、\(l_{2}^{t}\)、\(l_{1}^{\Delta}\)和\(l_{2}^{\Delta}\)。此外,两根额外的管子(分别用紫色和橙色标记),其长度\(l_{1}^{\mu}\)和\(l_{2}^{\mu}\)可调,分别与每个晶胞中的两个腔体相连。空气的声速和密度分别为\(c_{0}\)和\(\rho_{0}\),并且最外层的管子都用声学硬边界封闭。

为了合成类似的马约拉纳费米子,我们现在考虑在系统中以\(P_{z}\)模式传播的声波[图1(a)],它对应于频率为\(\omega_{0}\)的偶极驻波。相应地,第\(j\)个(\(j = 1,2\))每个晶胞腔体中的声速势的正态分布可以写为\(\psi_{j}(\vec{r}) = \sqrt{2 /(w^{2} h)} \sin(\pi r_{z} / h)\),其中\(r_{z}\)是\(\vec{r}\)在\(z\)方向的分量。进一步,通过定义参数\(\xi = [\xi_{1}, \xi_{2}]^{T}\),其中\(\xi_{j} = p_{j}(\vec{r}) / \psi_{j}(\vec{r})\)(\(p_{j}(\vec{r})\)是相应的声压),那么具有传播频率\(\omega\)的\(\xi\)表示的声场分布满足

\(\omega \xi = [\mathcal{H}_{0} + \mathcal{H}_{a}(k)] \xi\),\((4)\)

其中

\(\mathcal{H}_{0} = \begin{pmatrix} \omega_{0} + \epsilon_{1} & 0 \\ 0 & \omega_{0} + \epsilon_{2} \end{pmatrix}\),

\(\mathcal{H}_{a}(k) = \begin{pmatrix} \mu_{1} + 2t_{1} \cos(ka) & \Delta_{1} e^{ika} + \Delta_{2} e^{-ika} \\ \Delta_{1} e^{-ika} + \Delta_{2} e^{ika} & \mu_{2} + 2t_{2} \cos(ka) \end{pmatrix}\),\((5)\)

其中\(\epsilon_{1} = (c_{0} d^{2}|\vec{\psi}^{\prime}|^{2} / 2)[2 \cot(\omega_{0} l_{1}^{t} / c_{0}) + \cot(\omega_{0} l_{1}^{\Delta} / c_{0}) + \cot(\omega_{0} l_{1}^{\mu} / c_{0})]\)表示由连接到相应腔体的管子引起的总微扰,\(\epsilon_{2}\)具有相同形式但下标相反。关键的是,关键的对应关系被明确确定为\(t_{j} = -(c_{0} d^{2}|\vec{\psi}^{\prime}|^{2} / 2) \csc(\omega_{0} l_{j}^{t} / c_{0})\),\(\Delta_{j} = -(c_{0} d^{2}|\vec{\psi}^{\prime}|^{2} / 2) \csc(\omega_{0} l_{j}^{\Delta} / c_{0})\),以及\(\mu_{j} = i(\rho_{0} c_{0}^{2} d^{2}) /(w^{2} h Z_{j}^{\mu})\),其中\(|\vec{\psi}^{\prime}|\)表示\(\psi(\vec{r})\)在管子两端的平均值,\(Z_{j}^{\mu} = -i \rho_{0} c_{0} \cot(\omega_{0} l_{j}^{\mu} / c_{0})\)是额外管子的阻抗(见补充材料[50]的I节)。值得注意的是,所有参数自然解耦,振幅以及符号因此可以独立操纵。此外,一旦设置\(l_{1}^{t} = l_{2}^{t} + h\)和\(l_{1}^{\Delta} = l_{2}^{\Delta} + h\),我们得到\(\epsilon_{1} = \epsilon_{2}\),\(t_{1} = -t_{2}\),以及\(\Delta_{1} = -\Delta_{2}\),并立即发现\(\mathcal{H}_{a}(k)\)严格等同于\(\mathcal{H}(k)\)(在式(2)中),而\(\mathcal{H}_{0}\)仅在实践中贡献一个谱移。因此,基塔耶夫链最终可以在声学系统中实现。

根据上述讨论,我们现在设置\(w = 2.5\)厘米,\(h = 7.5\)厘米,\(d = 0.5\)厘米,\(l_{1}^{t} = 18.25\)厘米,\(l_{2}^{t} = 10.5\)厘米,\(l_{1}^{\Delta} = 18.95\)厘米,\(l_{2}^{\Delta} = 11.2\)厘米(因此\(l = 184\)和\(l^{\Delta} = 180\)),并通过仅控制\(l^{\mu}\)在声学基塔耶夫链中展示以下拓扑相变。图1(b) - 1(d)分别描绘了晶胞中\(\vert \mu \vert < 2t\)、\(\vert \mu \vert = 2t\)和\(\vert \mu \vert > 2t\)的过程,相应的理论和数值能带结构分别在图1(e) - 1(g)中给出,这为系统中的拓扑相变提供了良好的证据(见补充材料[50]的II节)。特别是,作为系统中关键的已识别拓扑特征,预计在基塔耶夫链的两端会出现一对不成对的马约拉纳零模式。两种极限情况下基塔耶夫链的图示分别在图2(a)和2(b)中给出,七个位点的声学对应物的能量谱分别在图2(c)和2(d)中给出。可以清楚地看到,与拓扑平凡时的完全体隙相比,在非平凡相中,两个简并的马约拉纳零模式恰好被钉扎在频率\(\omega_{0} + \epsilon\)处。关键的是,这种奇异模式的声场分布如图2(e)和2(f)所示,表明它们是不成对的。以下将详细进行实验。

\section{实验}

首先,我们考虑\(\vert \mu \vert = 2t\)时的能隙闭合条件。经过线性展开后,哈密顿量\(\mathcal{H}(k)\)具有如下形式:

\(\mathcal{H}(k) = m\tau_{z} + 2\Delta k\tau_{y}\),\((6)\)

其中\(m = -\mu - 2t\)是质量项。结果是,\(m = 0\)表示一个临界点,\(m\)的正负号因此表示不同的拓扑相,这提醒我们马约拉纳样零模式可以作为具有不同拓扑的两个畴之间的畴壁态出现。如图3(a)所示,我们构建了一个具有六个平凡(\(m > 0\))和六个非平凡(\(m < 0\))晶胞的声学样品作为畴壁。为了识别马约拉纳零模式,每个位点腔体都开有一个孔(当不在测量时用声学密封),并在样品外靠近畴壁处放置一个宽带声学激励源。图3(b)显示了测量的强度谱,很明显在约2430Hz处有一个光谱孤立的峰值,这恰好对应于图2(b)中预测的马约拉纳零模式频率。同时,马约拉纳样零模式的测量空间分布如图3(c)所示,展示了马约拉纳波函数的局域性(见补充材料[50]的III节)。因此,这些实验结果证实了声学基塔耶夫链中存在经典的马约拉纳样零模式。

最后,在理论模型中,只要化学势得到调节,马约拉纳费米子就可以被操控,这为“声学系统”带来了灵感。我们进行了基塔耶夫“键盘”的实验,以验证声学马约拉纳样零模式的准粒子特性[10]。由于在所呈现的结构中\(\mu\)和\(t\)是解耦的,我们现在定义一个拓扑相为“开(ON)”,平凡相为“关(OFF)”,如图4(a)所示,并且可以通过调节\(l^{\mu}\)轻松切换门(见图4(b))。相应地,给定的门可以被局部控制,这允许类似马约拉纳费米子被创建、传输和自由融合。图4(c)概述了一个跨越12个位点的基塔耶夫“键盘”的设置,这对应于三个独立实验,测量的空间强度分布分别在图4(d) - 4(f)中给出。因此,一个不成对的马约拉纳样零模式从位点2被驱动到位点6(见图4(d)),并且声学模式可以在传输过程中要么融合为一个有限能量模式(图4(e)中的位点5和位点8),要么通过控制门在位点5和位点8处被创建(见图4(f))(见补充材料[50]的IV节)。测量结果直接表明声学马约拉纳样零模式可以充当准粒子。此外,与一些现有的模型(如声学Su - Schrieffer - Heeger模型)相比,在我们的模型中只需要调节管子的长度,并且我们可以在单个样品中轻松实现各种功能。这一优势源于基塔耶夫链,它只需要调节化学势来操控马约拉纳费米子。
\chapter{总结与展望}
\section{研究总结}
\section{未来研究展望}



%---------------------------------------------------------------------
%	参考文献
%---------------------------------------------------------------------

% 生成参考文献页
\printbibliography

%---------------------------------------------------------------------
%	致谢
%---------------------------------------------------------------------

\begin{acknowledgement}
  时光匆匆,如白驹过隙,我的博士生涯即将画上句号。在即将告别校园之际,心中满是感慨与感恩。
  
  首先,我要衷心感谢我的导师程建春老师和邹欣晔老师。从论文的选题、研究方案的设计,到具体的实验操作、论文的撰写与修改,每一个环节都凝聚着导师的心血和智慧。课题组严谨的治学态度、敏锐的学术洞察力和科学的研究方法,如明灯照亮我在学术道路上前行。老师不仅在学术上给予我悉心指导,更在生活中关心我的成长,教会我如何做学问、如何做人。您的教诲如春风化雨,让我受益终身。师恩如山,我将永远铭记。
  
  感谢课题组的各位老师,梁彬老师,杨京老师等,你们精彩的授课和前沿的学术报告拓宽了我的学术视野,为我的研究奠定了坚实的理论基础。你们在学术上的严谨和对学生的关爱,让我深刻体会到了学者的风范和师者的责任。
  
  感谢实验室的刘广升,刘铭颢,彭尧吟,杨彰昭,韩冬雨,陈晋恒,李鑫,杨文杰,张志磊,李石峰,卢杰煜,冯楚尧,裴宁我,王昶淳,张龙发,蔡增昕,许铭轩,喻杰,左玲铭等各位同学,在实验过程中,我们相互帮助、相互鼓励,共同攻克了一个又一个难题。那些在实验室里并肩作战的日子,是我博士生涯中最宝贵的回忆。你们的陪伴和支持,让我在科研的道路上不再孤单。
  
  感谢我的同门师兄弟姐妹们,我们一起讨论学术问题,分享生活中的喜怒哀乐。在这个温暖的大家庭里,我感受到了浓浓的亲情和友情。你们的鼓励和支持,让我有勇气面对研究中的困难和挑战。
  
  感谢我的家人,尤其是我的父母。你们含辛茹苦地将我养大,为我提供了良好的学习和生活条件。在我求学的道路上,你们始终默默地支持我、鼓励我,是我最坚强的后盾。你们的爱和付出,我无以为报,唯有努力奋斗,不辜负你们的期望。

  
  同时,我也要感谢参与我论文评审和答辩的各位专家学者,你们提出的宝贵意见和建议,让我的论文更加完善。
  
  回顾博士生涯,有欢笑,有泪水,有成功的喜悦,也有失败的沮丧。但正是这些经历,让我不断成长和进步。我深知,我的每一点成绩都离不开大家的帮助和支持。
  
  在未来的日子里,我将带着这份感恩之心,继续努力前行。我会牢记母校的教诲,以更加饱满的热情和更加严谨的态度投入到工作和学习中,为社会做出自己的贡献。
  
  再次感谢所有关心、支持和帮助过我的人! 
\end{acknowledgement}

%---------------------------------------------------------------------
%	学术简历
%---------------------------------------------------------------------

% 详见手册中“成果列表”一节
\njuchapter{学术成果}
{
\setstretch{1.5} % 局部设置 1.2 倍行距
\njupaperlist[攻读博士学位期间发表的学术论文]{my_1,my_2,my_3,my_4,my_5,my_6,my_7}
}
%---------------------------------------------------------------------
%	附录部分
%---------------------------------------------------------------------

% 附录部分使用单独的字母序号
% \appendix

% 可以在这里插入补充材料
% \chapter{正文中涉及的数据及源代码}
% \dots

% 完工
\end{document}
